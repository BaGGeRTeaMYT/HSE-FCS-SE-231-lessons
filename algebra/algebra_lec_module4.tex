\documentclass[12pt, letterpaper, twoside]{article}
\usepackage[T2A]{fontenc}
\usepackage{amsfonts}
\usepackage{amsmath}
\usepackage{mathabx}
\usepackage{graphicx}

\title{Лекции по алгебре 4 модуль.}
\author{Андрей Тищенко}
\date{}

\begin{document}
    \maketitle
    \[\textbf{Лекция 3 апреля}\]
    \[\text{Квадратичные формы}\]
    \begin{enumerate}
        \item[\text{Определение:}] Многочлен второй степени от $n$ переменных, то есть выражение вида
        \[q\overset{x\in \mathbb{R}^n}{(x_1,\dots,\ x_n)} = \sum_{i = 1}^{n} a_{i\, i}x_i^2 + 2\sum_{1\leq i < j \leq n} a_{i\, j}x_{i}x_{j}\]
        Где $a_{i\, j}\in \mathbb{R}$, называют квадратичной формой.
        \item[\text{Замечание:}] Многочлен $q(x)$ называется однородным степени $k$, если
        \[\forall \alpha\quad q(\alpha x) = \alpha^k q(x)\]
        \item[\text{Замечание:}] Квадратичная форма - это отображение $q: V\longrightarrow \mathbb{R}$ (вектор в число)
    \end{enumerate}
    Рассмотрим $n$-мерное вектороное пространство $V$ над $\mathbb{R}$. Зафиксируем в нём базис $e_1,\dots,\ e_n$:\\
    Тогда у любого $x\in V$ есть набор координат в этом базисе $x_1,\dots,\ x_n$.\\
    То есть $\forall x\in V: x=x_1e_1+\dots+x_{n} e_{n}$\\
    Пусть $x^e = \begin{pmatrix}
        x_1\\
        \vdots\\
        x_n
    \end{pmatrix}\Rightarrow q(x)$ можно представить в виде $q(x) = (x^e)^{T}A x^{e}$, где $A = (a_{i\, j})$ матрица квадратичной формы $q(x)$ в базисе $e_1,\dots,\ e_n$,\\
    $a_{i\, j}$ - коэффициенты квадратичной формы.
    \begin{enumerate}
        \item[\text{Пример:}] В $\mathbb{R}^3$
        \[q(x) = x_1^2 + 8 x_1 x_3 = x_1^2 + 4 x_1 x_3 + 4x_3 x_1 = \begin{pmatrix}
            x_1 & x_2 & x_3
        \end{pmatrix}\cdot \begin{pmatrix}
            1 & 0 & 4\\
            0 & 0 & 0\\
            4 & 0 & 0
        \end{pmatrix}\cdot \begin{pmatrix}
            x_1\\
            x_2\\
            x_3
        \end{pmatrix}\]
        \item[\text{Замечание:}] Матрица квадратичной формы всегда симметрическая. То есть \[A^T = A\]
        \item[\text{Замечание:}] По любой билинейной форме можно построить квадратичную форму, взяв $q(x) = b(x,\ x)$. Тогда $a_{i\, j} = \dfrac{b_{i\, j} + b_{j\, i}}{2}$
        \item[\text{Пример:}] $b(x,\ y) = x_1 y_1 + e x_1 y_3 + 5 x_3 y_1\Rightarrow q(x) = b(x,\ x) = x^2_1 + 8 x_1 x_3$
        \item[\text{Определение:}] Билинейная форма называется $\underline{\text{симметрической}}$, если \[b(x,\ y) = b(y,\ x),\ \text{например, скалярное произведение}\] Называется $\underline{\text{кососиметрической}}$, если \[b(x,\ y) = -b(y,\ x)\]
        \item[\text{Пример:}] Кососиметрическая билинейная форма с матрицей $B = \begin{pmatrix}
            0 & -1\\
            1 & 0
        \end{pmatrix}\Rightarrow\\\Rightarrow B^T = -B$ 
        \item[\text{Замечание:}] По любой квадратичной форме можно построить симметрическую билинейную форму. Это называется $\underline{\text{поляризацией}}$ квадратичной формы.\[b(x,\ y) = \frac{1}{2}\big[q(x + y) - q(x) - q(y)\big]\] Полярная билинейная форма к $q(x)$ \big(имеет ту же матрицу, что и $q(x)$, $b(x,\ x) = q(x)$\big)
        \item[\text{Утверждение:}] При переходе от базиса $e$ к базису $e'$ в линейном пространстве $V$ матрица квадратичной формы меняется так:
        \[A' = C^{T}\cdot A\cdot C,\text{ "Стас" без рофлов, реально Стасямба конкретная}\]
        $A'$ - матрица квадартичной формы в новом базисе  $e'$\\
        $C$ - матрица перехода от базиса $e$ к базису $e'$
        \item[\text{Доказательство:}] Свзять координат вектора:\\
        $x = Cx'$, так как $x' = C^{-1} x$ - формула изменения координат вектора при замене базиса.\\
        Тогда $\forall x\quad q(x) = x^{T} A x = (Cx')^{T} A (C x') = (x')^T C^T A C x' = (x')^{T} A' x'$, значит $A' = C^{T}AC$ \big(Можно в качестве $x$ брать все векторы канонического базиса $(0,\dots 0,\ \underset{i}{1},\ 0,\dots,\ 0)$ и показать совпадение матричных элементов\big)
        \item[\text{Определение:}] Если квадратичная форма в некотором базисе записана в виде $q(x) = x^{T} A x$, то есть если $A$ - матрица квадратичной формы в некотором базисе, то $\operatorname{Rg} A$ называется рангом квадратичной формы $q(x)$.
    \end{enumerate}
    Почему это определение корректно? То есть почему $\operatorname{Rg} A$ не зависит от базиса.
    \begin{enumerate}
        \item[Лемма:] Пусть $A,\ U\in M_n(\mathbb{R}),\ \det U\neq 0$. Тогда $\operatorname{Rg} A\cdot U = \operatorname{Rg A} = \operatorname{Rg} U\cdot A$, то есть при умножении на невырожденную матрицу ранг не меняется.
        \item[Доказательство:] $\operatorname{Rg} A\cdot U \leq \operatorname{Rg} A$, так как столбцы матрицы $AU$ есть линейные комбинации столбцов матрицы $A$.\\
        Ранг матрицы по теореме о ранге матрицы равен максимальному числу линейно независимых столбцов не могло вырасти, так как все столбцы $AU$ линейно выражаются через столбцы исходной матрицы.\\
        Покажем $\operatorname{Rg} A\cdot U\geq \operatorname{Rg} A$. \[\operatorname{Rg} A = \operatorname{Rg} A(U\cdot U^{-1}) = \operatorname{Rg} (AU)U^{-1} \leq \operatorname{Rg} (AU)\]
        $\operatorname{Rg} U\cdot A = \operatorname{Rg} (UA)^T = \operatorname{Rg} A^T U^T = \operatorname{Rg} A^T = \operatorname{Rg} A = \operatorname{Rg} A U$
        \item[Утверждение:] (об инвариантности ранга квадратичной формы)\\
        Пусть $q(x)$ - квадратичная форам на линейном пространстве V.\\
        Пусть $a = (a_1,\dots,\ a_n)$ и $b = (b_1,\dots,\ b_n)$ - базисы в $V$.\\
        Пусть $A$ - матрица квадратичной формы в базисе $a$\\
        Пусть $B$ - матрицы квадратичной формы в базисе $b$\\
        Тогда $\operatorname{Rg} A = \operatorname{Rg} B$ и ранг квадратичной формы корректно определен.
        \item[Доказательство:] Было доказано, что $B = C^T AC\Rightarrow$ по лемме, так как мы умножаем матрицу $A$ на матрицы $C^T$ слева и на $C$ справа, то $\operatorname{Rg} B = \operatorname{Rg} A$, ч.т.д.\\
        \item[Определение:] квадратичную форму $q(x)$ будем назвать $\underline{\text{положительно определённой}}$, если \[\forall x \neq 0\quad q(x) > 0\]
        $\underline{\text{отрицательно определённой}}$, если \[\forall x\neq 0\quad q(x) < 0\]
        $\underline{\text{знакопеременной}}$, если \[\exists x,\ y\in V: q(x) < 0 < q(y)\] 
        \item[Пример:] $q_1(x) = x_1^2 + 2x^2_2 + 5x_3^2$ на $\mathbb{R}^3$ - положительно определена\\
        $q_2(x) = x_1^2 - x_3^2$ - знакопеременна $\left(y = \begin{pmatrix}
            1 & 0 & 0
        \end{pmatrix},\ x =\begin{pmatrix}
            0 & 0 & 1
        \end{pmatrix}\Rightarrow q(x) < 0 < q(y)\right)$.\\
        $q_3(x) = -x^2_1 - 2x_2^2 - 3x_3^2$ - отрицательно определена на $\mathbb{R}^3$,\\
        но $q_3'(x) = -x^2_1 - 3x_3^2$ - не является отрицательно определённой, так как $q'_3\begin{pmatrix}
            0\\
            1\\
            0
        \end{pmatrix} = 0$ - это неположительно определённая квадратная форма.
        \item[Теорема:] (Критерий Сильвестра положительной определённости)\\
        Пусть $A$ - матрица квадратичной формы $q(x)$ в некотором базисе. Тогда
        \[q(x)\text{ положительно определена}\Leftrightarrow \begin{matrix}\text{последовательность главных угловых}\\ \text{миноров в A строго положительна} \end{matrix}\]
        То есть $\begin{cases}
            \Delta_1 = a_{1\, 1} > 0\\
            \Delta_2 = \begin{vmatrix}
                a_{1\, 1} & a_{1\, 2}\\
                a_{2\, 1} & a_{2\, 2}
            \end{vmatrix} > 0\\
            \dots\\
            \Delta_{n} = \det A > 0
        \end{cases}$\\
        \item[Следствие:]
        \[\text{Квадратичная форма отрицательно определена}\Leftrightarrow \begin{cases}
            \Delta_1 < 0\\
            \Delta_2 > 0\\
            \dots\\
            (-1)^n\Delta_n > 0
        \end{cases}\]
        То есть знаки главных угловых миноров чередуются, начиная с минуса.
        \item[Доказательство:] Так как $A$ - отрицательно определена $\Leftrightarrow -A$ положительно определена\\
        $\det (-A)=  (-1)^n\det A$, ч.т.д.
        \item[Пример:] $q(x) = -x_1^2 - x_2^2 - \dots - x_n^2$ - отрицательно определённая\\
        $A = \begin{pmatrix}
            -1 & 0 & \dots & 0\\
            0 & -1 & \dots & 0\\
            \vdots & \vdots & \ddots & \vdots\\
            0 & 0 & \dots & -1
        \end{pmatrix}$
        \item[Определение:] Квадратичную форму $q(x) = \alpha_1 x_1^2 + \dots + \alpha_n x_n^2$, где $\alpha_i\in\mathbb{R},\ i = \overline{1,\ n}$, то есть в квадратичной форме нет попарных произведений вида $Cx_ix_j$, называют квадратичной формой каноничесмкого вида.\\
        Если $\alpha_i\in \{-1,\ 0,\ 1\}$, то канонический вид называют нормальным.
        \item[Замечание:] Матрица квадратичной формы в каноническом виде является диагональной.
    \end{enumerate}
\end{document}