\documentclass[12pt, letterpaper, twoside]{article}
\usepackage[T2A]{fontenc}
\usepackage{amsfonts}
\usepackage{amsmath}
\usepackage{mathabx}
\usepackage{graphicx}
\usepackage{hyperref}
\usepackage{listings}
\usepackage{color}
\usepackage{nicefrac}

\definecolor{dkgreen}{rgb}{0,0.6,0}
\definecolor{gray}{rgb}{0.5,0.5,0.5}
\definecolor{mauve}{rgb}{0.58,0,0.82}

\lstset{frame=tb,
  language=C++,
  aboveskip=3mm,
  belowskip=3mm,
  showstringspaces=false,
  columns=flexible,
  basicstyle={\small\ttfamily},
  numbers=none,
  numberstyle=\tiny\color{gray},
  keywordstyle=\color{blue},
  commentstyle=\color{dkgreen},
  stringstyle=\color{mauve},
  breaklines=true,
  breakatwhitespace=true,
  tabsize=3
}

\title{Групповая динамика 1 модуль.}
\author{Андрей Тищенко}
\date{2024/2025}

\begin{document}
    \maketitle
    \begin{center}
        \textbf{Лекция 3 сентября.}
    \end{center}
    \section*{Формула оценки}
    $\text{Накоп} = 0.25\text{ТЗ} + 0.65\text{ДЗ} + 0.1\text{НК}$, где \\
    ТЗ - оценка за документ "Техническое задание"\\
    ДЗ - общая оценка за домашние задания в модуле, считается как среднее арифметическое всех ДЗ, с округлением в сторону большего, до 4 знака после запятой.\\
    НК - оценка за проведенный нормоконтроль командой заказчиком документа "Техническое задание", представленного командой-исполнителем.
    \section{Лекция}
    \subsection*{Что такое команда?}
    Группа людей, объединённых общей целью, команда выполняет совместную работу (работают вместе, а не рядом).
    \subsection*{Зачем собирают команды?}
    Команда эффективнее одиночки.\\
    Синегрия - суммирующий эффект взаимодействия двух или более человек, характеризующийся тем, что их действие существенно превосходит эффект каждого отдельного человека. Знания и умения нескольких человек могут организовываться таким образом, что они взаимно усиливаются.\\
    Целок больше суммы отдельных частей (Аристотель, АУФ).
    \subsection*{Как образовываются команды}
    Тест Белбина для формирования команд. Определил понятие $\textit{эффективная команда}$.\\
    Роли в команде (психологические портреты скипнуты, потому что приколов и без этого достаточно):
    \begin{enumerate}
        \item Мотиватор (авторитетный чувак, крутые шутки про жопу)
        \item Исполнитель (executor)
        \item Педант (***ик)
        \item Координатор (F3 в майнкрафте)
        \item Душа команды (какие-то рофлы про свадьбу, я чуть не сдох)
        \item Исследователь (ануса)
        \item Генераторы идей (идея повысить процент троечников на предмете, думайте)
        \item Аналитик (могут вкидывать что-то неожиданное. Банан!)
        \item Специалист (знает, как накормить жирного белого червя, скорее всего оплата почасовая)
    \end{enumerate}
    \subsection*{Как сформировать эффективную команду}
    \begin{itemize}
        \item должны быть закрыты все роли
        \item Тестирование должно быть честным
        \item Мотиватор и координатор должен быть либо один человек, либо они не должны быть в сильный ролях одновременно
        \item Сильные и поддерживающие роли не должны пересекаться. 
    \end{itemize}
    \subsection*{Ещё невероятной инфы}
    Тест Хони-Мамфорда. Для выявления стилей обучения, а так же для проверки занимаемой должности.\\
    Ещё один списочек:
    \begin{enumerate}
        \item Деятель (не придумал)
        \item Мыслитель (Лена Головач Александрийский)
        \item Теоретик (Гуевреев Яж из Рима)
        \item Прагматик ($\#pragma\ warning$)
    \end{enumerate}
\end{document}