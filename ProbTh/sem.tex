\documentclass[12pt, a4paper]{article}
\usepackage[utf8]{inputenc}
\usepackage[T2A]{fontenc}
\usepackage[russian]{babel}
\usepackage{amsfonts}
\usepackage{amsmath}
\usepackage{mathabx}
\usepackage{graphicx}
\usepackage{hyperref}
\usepackage{listings}
\usepackage{color}
\usepackage[margin=0.25in]{geometry}
\usepackage{pgfplots}
\pgfplotsset{width=10cm,compat=1.9}
\usepackage{nicefrac}
\graphicspath{../Images}
\usepackage{tikz}
\usepackage{cancel}

\usepackage{fontspec}
\setmainfont{Times New Roman}

\newcommand{\cov}{\operatorname{cov}}

\title{Теория вероятностей.}
\author{Андрей Тищенко \href{https://t.me/AndrewTGk}{@AndrewTGk}}
\date{2024/2025}

\begin{document}
    \maketitle
    \begin{center}
        \textbf{Семинар 6 сентября.}
    \end{center}
    \subsection*{Теория}
    Классическое определение вероятности:\\
    Количество исходов конечно, они взаимоисключающие и равновозможные.\\
    $P(A) = \frac{|A|}{|\Omega|}\in [0,\ 1]$.
    \subsection*{Задача 0}
    Казино Монте-Карло с 37 слотами. 18 красных, 18 чёрных и 0. Тогда вероятность 
    выиграть при ставке на красное будет $\frac{18}{37}$. Ставка удваивается.\\
    Можно играть на дюжины $[1,\ 12],\ [13,\ 24],\ [25,\ 36]$, тогда вероятность 
    выигрыша $\frac{12}{37}$. Ставка при этом утраивается.
    \subsection*{Задача 1}
    У взломщика есть связка из 10 ключей. С какой вероятностью он откроет дверь, 
    перебрав ровно половину ключей.\\
    Ключ равновероятно может находиться на любой позиции в его связке. На пятом месте он 
    будет в единственном случае, тогда $P = \frac{1}{10}$
    \subsection*{Задача 2}
    Студент выучил 20 билетов из тридцати. С какой вероятностью 
    ему достанется выученный билет, если он заходит первым? Вторым?
    Если первым: $\frac{20}{30}$\\
    Если вторым: $\frac{19}{29}\frac{2}{3} + \frac{20}{29}\frac{1}{3} = \frac{2}{3}$
    \subsection*{Задача 3}
    Есть буквы М, О, С, К, В, А. Какова вероятность получить 
    слово МОСКВА при случайном расположении этих букв.\\
    $P(A) = \frac{1}{6!} = \frac{1}{720}$\\
    Условие такое же, но буквы А, Б, Р, А, К, А, Д, А, Б, Р, А и нужно получить АБРАКАДАБРА.\\
    $P(A) = \frac{5!\,2!\,2!}{11!}$
    \subsection*{Задача 4}
    Есть два брата и 10 мест за круглым столом. Какова вероятность 
    размещения братьев напротив друг-друга.\\
    Одного фиксируем, у второго 9 вариантов $P(A) = \frac{1}{9}$\\
    Рассматриваем положение обоих $P(A) = \frac{10\cdot 8!}{10!} = \frac{1}{9}$
    \subsection*{Задача 5}
    10 человек, 10 мест, меджу двумя конкретными должно быть 3 человека.
    $P(A) = \frac{6\cdot 2\cdot 8!}{10!}$\\
    Выбираем $m$ элементов из $N$, с учётом порядка:\\
    $A^m_N = N(N - 1)\dots(N - m + 1) = \frac{N!}{(N - m)!}$\\
    Без учёта порядка:\\
    $C_N^m = \frac{A^m_N}{m!} = \frac{N!}{(N - m)!m!}$
    \subsection*{Задача 6}
    Угадываем номер телефона, знаем все цифры, кроме последних трёх, но известно, что они разные:\\
    $P(A) = \frac{1}{A^3_{10}} = \frac{1}{720}$
    \subsection*{Задача 7}
    Какова вероятность выиграть в спортлото (49 видов спорта, 6 выигрышных, нужно собрать все 6).\\
    $P(A) = \frac{1}{C^6_{49}}\approx \frac{1}{14\,000\,000}$.\\
    $P(\overline{A}) = 1 - P(A)$
    \subsection*{Задача 8}
    Выбираются 3 цифры, хотим, чтобы их произведение было чётным.\\
    Посчитаем вероятность нечётности произведения: $P(\overline{A}) = \frac{C_5^3}{C_{10}^3} = \frac{5!}{2!3!}\frac{7!3!}{10!} =\\
    = \frac{3\cdot 4\cdot 5}{8\cdot 9\cdot 10} = \frac{1}{12}\Rightarrow\\
    \Rightarrow P(A) = 1 - \frac{1}{12} = \frac{11}{12}$
    \subsection*{Задача 9}
    Колода 52 карты, какова вероятность достать 4 карты одной масти:\\
    $P(A) = \frac{4\cdot C^4_{13}}{C_{52}^4}$
    \subsection*{Задача 10}
    Достать тройку, семёрку и туза из колоды с 52 картами:\\
    $P(A) = \frac{4^3}{C_{52}^3}$
    \subsection*{Задача 11}
    90 хороших и 10 плохих деталей, какова вероятность, что среди пяти вытащенных деталей нет брака:\\
    $P(A) = \frac{C^5_{90}}{C^5_{100}} = \frac{90!\,5!\,95!}{5!\,85!\,100!} = \frac{86\cdot 87\cdot_{\dots}\cdot 89\cdot 90}{96\cdot 97\cdot_{\dots}\cdot 99\cdot 100}$\\
    Хотим 3 хороших и 2 плохих:\\
    $P(B) = \frac{C_{90}^3\cdot C_{10}^2}{C_{100}^5}$
    \subsection*{Задача 12}
    В спортлото угадать четыре из шести:\\
    $P(A) = \frac{C_6^4C_{43}^2}{C_{49}^6}$
    \subsection*{Задача 13}
    Из 52 карт достать 2 красные и 2 чёрные карты:\\
    $P(A) = \frac{C^2_{26}\cdot C^2_{26}}{C^4_{52}}$
    \subsection*{Задача 14}
    Вероятность при трёх бросках кубика получить три разных цифры:\\
    $P(A) = \frac{A^3_6}{6^3},\ A^3_6 = 6\cdot 5\cdot 4$\\
    При броске шести кубиков выпали все цифры:\\
    $P(A) = \frac{6!}{6^6}$
    \subsection*{Задача 15}
    В лифте девятиэтажного здания на первом этаже окалось 8 студентов. Никто не выходит на первом этаже, 
    какова вероятность того, что лифт не остановится хотя бы на одном этаже.\\
    $P(A) = 1 - \frac{8!}{8^8}$
    \subsection*{Задача 16}
    Какова вероятность, что 10 монет выпадут на одинаковую сторону:\\
    $P(A) = \frac{2}{2^{10}} = \frac{1}{2^9}$
    \subsection*{Задача 17}
    Коробка с 100 шариками, каждый имеет номер от 1 до 100. Какова вероятность вытащить все 
    шары и получить возрастающую последовательность, если:\\
    a) Не возвращать шары в коробку:\\
    $P(A) = \frac{1}{100!}$\\
    б) Возвращать шары в коробку:\\
    $P(A) = \frac{1}{100^{100}}$
    \begin{center}
        Семинар 13 сентября
    \end{center}
    \subsection*{Задача 1}
    Бросаем три шестигранных кубика, найти вероятность выпадения суммы 11 и 12.\\
    $|\Omega| = 6^3$
    \[\begin{tabular}{c|c|c|c}
        Комбинаций & $\sum 11$ & $\sum 12$ &Комбинаций\\
        \hline 
        6 & 5 - 4 - 2 & 5 - 5 - 2 & 3\\
        3 & 5 - 3 - 3 & 6 - 5 - 1 & 6\\
        3 & 4 - 4 - 3 & 6 - 4 - 2 & 6\\
        3 & 5 - 5 - 1 & 6 - 3 - 3 & 3\\
        6 & 6 - 3 - 2 & 5 - 4 - 3 & 6\\
        6 & 6 - 4 - 1 & 4 - 4 - 4 & 1
    \end{tabular}\]
    $P(A_{11}) = \frac{27}{6^3},\ P(A_{12}) = \frac{25}{6^3}$\\
    \subsection*{Задача со * из ДЗ}
    Какова вероятность, что если взять 4 башмака из 10 пар, получишь пару.
    $P(\overline{A}) = \frac{20\cdot 18\cdot 16\cdot 14}{20\cdot 19\cdot 18\cdot 17}$
    \subsection*{Задача 2}
    В девятиэтажном доме три человека садятся в лифт. Какова вероятность, что лифт остановится для высадки два раза?\\
    $P(A) = 1 - \frac{8 + 8\cdot 7\cdot 6}{8^3} = \frac{21}{64}$\\
    $\dfrac{A^2_8\cdot 3}{8^3} = \dfrac{21}{64}$
    \subsection*{Задача 3}
    Имеется 100 чисел. Из них вытаскивают 15 чисел и упорядочивают по возрастанию.\\
    Какова вероятность, что 13 число в полученной последовательности равно 87.\\
    $\dfrac{C_{86}^{12}\cdot C^2_{13}}{C_{100}^{15}}$
    \subsection*{Задача 4}
    Есть 10 вагонов. Какова вероятность, что два человека окажутся в одном вагоне/ в соседних?\\
    В одном: $\frac{1}{10}$ (оба в один и тот же вагон, выбрать вагон можно 10 способами)\\
    В соседних: $\frac{18}{100}$ (9 различных пар (1, 2), (2, 3), (3, 4) и т.д. и наоборот).
    \section*{Геометрические вероятности}
    \subsection*{Задача 5}
    В квадрат со стороной $R$ вписан круг, какова вероятность, что брошенная в квадрат точка попадёт в круг.\\
    $P(A) = \frac{\frac{\pi R^2}{4}}{R^2} = \frac{\pi}{4}$
    \subsection*{Задача 6}
    На интервале $[0,\ 1]$ выбираются точки $x,\ y$, найти вероятность события:\\
    $x^2\leq y\leq\sin \frac{\pi x}{2}$\\
    $P(A) = \displaystyle \frac{\int_0^1 \sin\frac{\pi x}{2} - x^2 dx}{1} = \frac{2}{\pi}\int_0^1 \sin\frac{\pi x}{2} d\frac{\pi x}{2} - \frac{1}{3} = \frac{2}{\pi}\left.(-\cos \frac{\pi x}{2})\right|_0^1 - \frac{1}{3} = \\
    = \frac{2}{\pi} - \frac{1}{3}$
    \section*{Условные вероятности}
    \subsection*{Задача 7}
    В коробке есть $n$ белых и $m$ чёрных шаров. A = первый шар белый, B = последний шар чёрный\\
    $P(A) = \frac{n}{m + n},\ P(B) = \frac{m}{m + n}\\
    P(A/B) = \frac{P(AB)}{P(B)} = \frac{\frac{n}{m + n}\cdot \frac{m}{m + n - 1}}{\frac{m}{n + m}} = \frac{n}{n + m - 1}$
    \subsection*{Задача 8}
    Два игрока подбрасывают кость по одному разу, побеждает тот, кто выбил больше. A = победил первый, B = победитель определён.\\
    1. $P(A) = \frac{15}{36}$\\
    2. $P(B) = \frac{30}{36}$\\
    3. $P(A/B) = \frac{15}{30} = \frac{1}{2}$\\
    4. $P(A/B) = \frac{P(AB)}{P(B)} = \frac{\frac{15}{36}}{\frac{5}{6}} = \frac{1}{2}$, так как $A\subseteq B$\\
    События $A,\ B$ зависимы, так как $P(A/B) \neq P(A)$.
    \subsection*{Задача 9}
    2 партии по 100 деталей, в каждой партии 10 бракованных деталей.\\
    A = \{деталь из первой партии\}.\\
    B = \{деталь бракованная\}.\\
    $P(A) = \frac{100}{200} = \frac{1}{2},\ P(B) = \frac{20}{200} = \frac{1}{10}\\
    P(AB) = \frac{10}{200} = \frac{1}{20}\Rightarrow P(AB) = P(A)\cdot P(B)$, события независимы
    \subsection*{Задача 10}
    Почти как задача 9, но во второй 20 бракованных:\\
    $P(A) = \frac{1}{2}\\
    P(B) = \frac{3}{20}\\
    P(AB) = \frac{1}{20}\Rightarrow P(A)P(B)\neq P(AB)$, события зависимы.
    \section*{Формула сложения вероятностей}
    $P(A_1 + A_2 + \dots + A_n) =\displaystyle \sum_{i = 1}^{n} P(A_i) - \sum_{i \leq j} P(A_i A_j) + \sum_{i\leq j\leq k} P(A_i A_j A_k) + \dots \\
    \dots + (-1)^{n + 1}(A_1\dots A_n)$\\
    $P(A_1 A_2) = P(A_1)P(A_2/ A_1)$
    \subsection*{Задача 11}
    В урне 10 белых, 8 синих, 2 красных шара. Одновременно извлекают 3 шара, какова вероятность, что вытащенные шары одного цвета.\\
    A = \{вытащили (вовремя) 3 белых шара\}
    B = \{вытащили 3 синих шара\}\\
    $A_i$ = \{i-й шар белый\}
    $B_i$ = \{i-й шар синий\}
    Так как A, B несовместны: $P(A + B) = P(A) + P(B) = P(A_1 A_2 A_3) + P(B_1 B_2 B_3) =\\
    =P(A_1)P(A_2/A_1)P(A_3/A_1 A_2) + P(B_1)P(B_2/ B_1)P(B_3/ B_1 B_2) =\\
    = (\frac{10}{20} \cdot \frac{9}{19} \cdot \frac{8}{18}) + (\frac{8}{20} \cdot \frac{7}{19} \cdot \frac{6}{18})$

    \begin{center}
        Семинар 20 сентября.
    \end{center}

    \subsection*{Задача 46 из дз}
    Дано: $P(A/B) = 0.05,\ P(A\overline{B}) = 0.079,\ P(\overline{A}B) = 0.0089,\ P(\overline{A}\overline{B}) = 0,782$\\
    Хотим: $P(B/A)$\\
    $P(B/A) = \frac{P(BA)}{P(A)}\\
    P(A) = P(AB + A\overline{B}) = P(AB) + P(A\overline{B}) - \underset{=0}{\underbrace{P(AB\cap A\overline{B})}}=\\
    = 0.05 + 0.079 = 0.129\Rightarrow P(B/A) = \frac{0.05}{0.129} \approx 0.3875$\\
    Ответ: $39\%$
    \subsection*{Задача}
    5 мальчиков и 10 девочек, какова вероятность, что при разбиении их на 5 равных групп получим только группы вида ЖМЖ.\\
    $A_i$ --- \{в i группе 2 девочки и 1 мальчик.\}\\
    $P(A_1 A_2 A_3 A_4 A_5) = P(A_1)\cdot P(A_2/ A_1)\cdot P(A_3/ A_1 A_2)\cdot P(A_4/ A_3 A_2 A_1)\cdot P(A_5 / A_4 A_3 A_2 A_1) = \\
    = \frac{C^2_{10} C^1_5}{C^3_{15}} \frac{C^2_{8} C^1_4}{C^3_{12}} \frac{C^2_{6} C^1_3}{C^3_{9}} \frac{C^2_{4} C^1_2}{C^3_{6}} \frac{C^2_{2} C^1_1}{C^3_{3}} 1$\\
    \subsection*{Задача}
    Система состоит из последовательно соединённых резисторов.\\
    $A_i = \{\text{i-й элемент системы работает}\},\ i = \overline{1,\ n}\\
    A = \{\text{Система работает}\}\\
    A = A_1\cdots A_n\\
    P(A) = P(\displaystyle\prod^n_{i = 1} A_i) = \prod_{i = 1}^{n} p_i$\\
    Если система состоит из параллельно соединённых резисторов, тогда:\\
    $P(A) = P(A_1 + A_2) = P(A_1) + P(A_2) - P(A_1 A_2) = p_1 + p_2 - p_1 p_2$
    \subsection*{Задача}
    Обращается русалочка к трём ведьмам, каждая даёт ей сосуд с зельем.\\
    Вероятность отсутсвия эффекта:
    \begin{enumerate}
        \item 0.5
        \item 0.4
        \item 0.3
    \end{enumerate}
    Она пьёт их по очереди и останавливается, если какое-то сработало. Какова вероятность успеха (хотя бы одно зелье сработало).\\
    $A_i$ = \{подействует i-е зелье\}\\
    Пусть ни одно не сработало: $P(A) = 1 - P(\overline{A}) = 1 - P(\overline{A}_1)P(\overline{A}_2)P(\overline{A}_3) =\\
    = 1 - 0.5\cdot 0.6 \cdot 0.7 =  0.79$\\
    Другой способ решения: $P(A) = P(A_1) + P(\overline{A}_1)P({A}_2) + P(\overline{A}_1)P(\overline{A}_2)P(A_3) = 0.5 + 0.5\cdot0.4 + 0.5\cdot0.6\cdot0.3 = 0.79$\\
    Ещё один: $P(A) = P(A_1 + A_2 + A_3) = P(A_1) + P(A_2) + P(A_3) - P(A_1 A_2) -\\
    - P(A_1 A_3) - P(A_2 A_3) + P(A_1 A_2 A_3) = 0.79$
    \subsection*{Задача}
    Теннист участвует в турнире, у него есть соперник A и соперник B, известно, что соперник A играет лучше соперника B. Теннист хочет выиграть два матча подряд, какой порядок ему лучше выбрать?\\
    $A - B - A$ или $B - A - B$?\\
    $P(A) < P(B)$ --- вероятность победы.\\
    У него есть варианты: выиграть первые две, проиграть первую и выиграть оставшиеся.\\
    В первом случае: $(1) = P(A) P(B) + \big(1 - P(A)\big) P(B) P(A)$\\
    Во втором: $(2) = P(B) P(A) + \big(1 - P(B)\big) P(A) P(B)$\\
    $(1) - (2) = P(A)P(B)(1 - P(A) - 1 + P(B)) = P(A)P(B)(P(B) - P(A)) > 0\Rightarrow (1) > (2)$\\
    В первом случае вероятность победы больше.\\
    \subsection*{Задача}
    Два равносильных шахматиста играют между собой матчи, ничьих быть не может. Какое событие более вероятно:\\
    $C^3_4 \left( \frac{1}{2} \right)^4 = \frac{4!}{3!}\frac{1}{16} = \frac{1}{4}\\
    C^5_8\cdot \left( \frac{1}{2} \right)^8 = \frac{7}{32}$\\
    Вероятность выиграть 3 из 4 больше.
    \subsection*{Задача}
    В круг радиуса $r$ вписан квадрат. В круг кидают 4 точки, какова вероятность попадания ровно 3 точек в квадрат.\\
    $p = \frac{2r^2}{\pi r^2} = \frac{2}{\pi}$ - вероятность попасть в квадрат.\\
    $q = 1 - p = \frac{1} - \frac{2}{\pi}\\
    C^3_4\cdot \left(\frac{2}{\pi}\right)^3\cdot \frac{\pi - 2}{\pi} = \frac{32(\pi - 2)}{\pi^4}$
    \subsection*{Задача}
    Стрелок попадает в 10 с вероятностью $0.7$, а в 9 с вероятностью $0.3$. Какова вероятность получения за 3 выстрела не менее 29 очков.\\
    Нас устраивают события:\\
    $A_{29} = \{\text{Набрал 29 очков}\},\ A_{30} = \{\text{Набрал 30 очков}\}\\
    P(A_{29} + A_{30}) = P(A_{29}) + P(A_{30}) = C^2_3 (0.7)^2\cdot 0.3 + (0.7)^3 = 0,784$
    \subsection*{Задача}
    7 писем. 0.6 - письмо отправлено Онегину, 0.4 - письмо отправлено Ленскому.\\
    $A_i = \{\text{Ровно i писем отправлено Онегину}\}$.
    $p(A) = P(A_7) + P(A_6) + P(A_5) = (0.6)^7 + 7(0.6)^6\cdot 0.4 + С^2_7 0.6^5\cdot 0.4^2$\\
    \subsection*{Задача}
    Стрелок попадает в мишень с вероятнотсью $0,7$. Ему позволяют стрелять до трёх промахов. Какова вероятность того, что он сделает ровно 8 выстрелов.\\
    $C^2_7 (0,7)^5\cdot 0,3^2\cdot 0,3$ (Два промаха можно как-то расположить в первых 7 выстрелах, последний всегда восьмой).
    \subsection*{Задача}
    Всего 5 испытаний, вероятность искажения результата - $0,1$.
    A = ни одного искажённого.
    Б = не менее двух искажённых.
    В = Искажённых больше, чем неискажённых.\\
    $P(A) = 0,9^5,\quad P(\text{Б}) = 1 - 0.9^5 - (C^1_5\cdot 0,1\cdot 0,9^4),\\
    P(\text{В}) = P(B) - C^2_5\cdot 0,1^2\cdot 0,9^3$\\
    \subsection*{Задача}
    Гипотезы при подбрасывании двух кубиков:\\
    $\begin{cases}
        H_1 = \{\text{На 1-м кубике выпала ``1''}\}\\
        \dots\\
        H_6 = \{\text{На 1-м кубике выпала ``6''}\}
    \end{cases}\\
    \begin{cases}
        H_1 = \{\text{1 --- 1}\}\\
        \dots\\
        H_{36} = \{\text{6 --- 6}\}
    \end{cases}\\
    \begin{cases}
        H_1 = \{\text{На 1-м --- чётное}\}\\
        H_2 = \{\text{На 1-м --- чётное}\}
    \end{cases}$
    \begin{center}
        Семинар 27 сентября
    \end{center}
    \subsection*{Формула полной вероятности}
    $H_1,\dots,\ H_n$ --- полная группа событий. $i \neq j \Rightarrow H_i\cdot H_j = \emptyset,\ H_1 + \dots + H_n = \Omega$\\
    $P(A) = \displaystyle \sum_{i = 1}^{n} P(H_i)P(A/H_i)\\
    P(H_k/A) = \frac{P(H_k)P(A/H_k)}{\sum\limits_{i = 1}^{n} P (H_i) P(A/H_i)}$\\
    \subsection*{Задача}
    Какова вероятность успешного переливания крови от одного человека к другому? Произошло успешное переливание, какова вероятность, что кровь переливали человеку с 1 группой? С 4 группой?\\
    Медицинская справка:\\
    Первой группе крови можно переливать только первую\\
    Второй --- вторую и первую\\
    Третей --- первую и третью\\
    Четвёртой --- любую.\\
    $A = \{\text{Успешное переливание}\}\\
    H_{i} = \{\text{У больного i группа}\},\ i = \overline{1,\ 4},\text{ по условию:}\ P(H_1) = 0.33,\\
    P(H_2) = 0.36,\ P(H_3) = 0.23,\ P(H_4) = 0.08$\\
    $P(A/H_1) = 0.33,\ P(A/H_2) = 0.69,\ P(A/H_3) = 0.56,\ P(A/H_4) = 1\\
    P(A) = \sum\limits_{i = 1}^{4} P(A/H_i)P(H_i) = 0.33^2 + 0.36\cdot 0.69 + 0.23\cdot 0.56 + 0.08 = 0.5661$\\
    $P(H_1/A) = \frac{P(A/H_1)P(H_1)}{P(A)} = \frac{0.33^2}{0.5661} \approx 0,19237\\
    P(H_4/A) = \frac{P(A/ H_4)P(H_4)}{P(A)} = \frac{0.08}{0.5661}\approx 0.14$
    \subsection*{Задача}
    Две корзинки, в каждой по 10 бутылок, в первой корзине 2 отравленные, во второй --- 3, по пути одна бутылка из первой корзины разбилась. Какова вероятность не отравиться при распитии одной бутылки?\\
    $A = \{\text{распитие благополучно}\}\\
    H_1 = \{\text{бутылка из 1 корзины}\},\ P(H_1) = \frac{1}{2}\\
    H_2 = \{\text{бутылка из 2 корзины}\},\ P(H_2) = \frac{1}{2}\\$
    Так как выбираем корзину, а не бутылку из неё.\\
    $P(A/H_1) = 0.8,\ P(A/ H_2) = 0.7$. Вспоминаем задау про выученные билеты (поэтмоу вероятность отравиться бутылкой из первой корзины не поменяется).\\
    $P(A) = P(H_1)P(A/H_1) + P(H_2)P(A/H_2) = \frac{8}{20} + \frac{7}{20} = \frac{3}{4}$
    \subsection*{Задача}
    По каналу передаётся два вида сигналов $x,\ y$, при этом $y$ передаётся в три раза чаще $x$. $x$ искажается в 10\% случаев, а $y$ --- в 20\%. По каналу передан какой-то сигнал, какова вероятность, что будет получен сигнал $x$? Какова вероятность, что при получении сигнала $x$ он и был передан?\\
    $H_1 = \{\text{Передавали x}\},\ P(H_1) = \frac{1}{4}\\
    H_2 = \{\text{Передавали y}\},\ P(H_2) = \frac{3}{4}\\
    A = \{\text{Зафиксирован x}\},\\
    P(A) = P(H_1)P(A/H_1) + P(H_2)P(A/H_2) = \frac{1}{4}0.9 + \frac{3}{4}0.2 = \frac{3}{8}\\$
    $B = H_1/A\Rightarrow P(B) = P(H_1/A) = \frac{P(A/H_1)P(H_1)}{P(A)} = \frac{\frac{9}{40}}{\frac{3}{8}} = \frac{3}{5} = 0.6$

    \subsection*{Задача}
    Три стрелка стреляют по мишени. Они соответсвенно попадают в мишень с вероянтностью:
    \begin{enumerate}
        \item[1 стрелок] $\frac{4}{5}$
        \item[2 стрелок] $\frac{3}{4}$
        \item[3 стрелок] $\frac{2}{3}$
    \end{enumerate}
    Все трое выстрелили одновременно, в мишень попали два раза. Какое событие более вероятно: промах третьего стрелка или его попадание?\\
    $A = \{\text{попали ровно 2 раза}\}\\
    H_1 = \{\text{третий попал}\},\ P(H_1) = \frac{2}{3}\\
    H_2 = \{\text{третий промахнулся}\},\ P(H_2) = \frac{1}{3}\\
    P(A/H_1) = \frac{4}{5}\frac{1}{4} + \frac{1}{5}\frac{3}{4} = \frac{7}{20}\\
    P(A/H_2) = \frac{4}{5}\frac{3}{4} = \frac{12}{20} = \frac{3}{5}\\
    P(A) = P(A/H_1)P(H_1) + P(A/H_2)P(H_2) = \frac{7}{20}\frac{2}{3} + \frac{3}{5}\frac{1}{3} = \frac{7}{30} + \frac{1}{5} = \frac{13}{30}\\
    P(H_1/A) = \frac{P(A/H_1)P(H_1)}{P(A)} = \frac{\frac{7}{20}\frac{2}{3}}{\frac{13}{30}} = \frac{7}{30}\frac{30}{13} = \frac{7}{13}$
    \subsection*{Задача}
    Было две урны. В первой 5 белых шаров и 1 чёрный, во второй --- 3 белых и 3 чёрных.\\
    Из первой урны взяли два шара, из второй один и поместили их в третью урну. Какова вероятность того, что наугад вытащенный из третьей урны шар окажется белым?\\
    $A = \{\text{Вытащили белый шар из третьей урны}\}\\
    H_1 = \{\text{Вытащенный шар был в первой урне}\}\\
    H_2 = \{\text{Вытащенный шар был во второй урне}\}\\
    P(A/H_1) = \frac{5}{6}\\
    P(A/H_2) = \frac{1}{2}\\
    P(H_1) = \frac{2}{3}\\
    P(H_2) = \frac{1}{3}\\
    P(A) = P(A/H_1) P(H_1) + P(A/H_2)P(H_2) = \frac{10}{18} + \frac{1}{6} = \frac{13}{18}$
    \begin{center}
        Семинар 4 октября
    \end{center}
    Решаем какие-то задачки из учебника.
    \subsection*{Задача}
    $\begin{tabular}{c|c|c|c}
        $\xi$ & -1 & 0 & 1\\
        \hline
        $p$ & $p_1$ & $p_2$ & $p_3$ 
    \end{tabular},\ E\xi = 0,\ \mathcal{D}\xi = 0.5$. Найти $p_1,\ p_2,\ p_3$.\\
    $\begin{tabular}{c|c|c}\
        $\xi$ & $0$ & $1$\\
        \hline
        $p$ & $p_2$ & $p_1 + p_3$ 
    \end{tabular},\ \mathcal{D}\xi = E\xi^2 - (E\xi)^2 = E\xi^2 = p_1 + p_3\\
    \begin{cases}
        p_1 + p_2 + p_3 = 1\\
        p_1 = p_3\\
        p_1 + p_3 = 0.5
    \end{cases}\Rightarrow \begin{cases}
        p_1 = 0.25\\
        p_2 = 0.5\\
        p_3 = 0.25
    \end{cases}$
    \subsection*{Задача}
    a.\\
    $\begin{tabular}{c|c|c|c|c|c}
        $\xi$ & $-0,5$ & $0$ & $0,5$ & $1$ & $1,5$\\
        \hline
        $p$ & $0,1$ & $0,4$ & $0,1$ & $0,3$ & $0,1$
    \end{tabular}\\
    Y = 10x - 1\Rightarrow \begin{tabular}{c|c|c|c|c|c}
        $Y$ & $-6$ & $-1$ & $4$ & $9$ & $14$\\
        \hline
        $p$ & $0,1$ & $0,4$ & $0,1$ & $0,3$ & $0,1$
    \end{tabular}\\
    EY = \sum\limits_{i = 1}^{n} y_i p_i = 3,5\\
    \mathcal{D}Y = 3,5^2$\\
    b.\\
    $\begin{tabular}{c|c|c|c|c}
        $\xi$ & $-0,25$ & $0$ & $-1$ & $-2,25$\\
        \hline
        $p$ & $0,2$ & $0,4$ & $0,3$ & $0,1$
    \end{tabular}\\
    Y = -x^2\Rightarrow EY = -0,53\\
    \mathcal{D}Y = 0,25^2\cdot 0,2 + 0,3 + 2,25^2\cdot 0,1 - (0,53)^2$
    \subsection*{Задача}
    $\begin{tabular}{c|c|c|c}
        $\xi$ & 2 & 1 & 0\\
        \hline
        $p$ & $\frac{28}{45}$ & $\frac{16}{45}$ & $\frac{1}{45}$
    \end{tabular}\\
    P(\xi = 2) = \frac{C^2_8}{C^2_{10}} = \frac{7\cdot 8}{9\cdot 10} = \frac{28}{45}\\
    P(\xi = 1) = \frac{2}{10}\frac{8}{9}\cdot 2 = \frac{16}{45}\\
    P(\xi = 0) = \frac{1}{C^2_{10}} = \frac{1}{45}\\
    E\xi = \frac{2\cdot 28 + 16}{45} = \frac{72}{45} = \frac{8}{5}\\
    \mathcal{D}\xi = E\xi^2 - \big( E\xi \big)^2 = \frac{128}{45} - \frac{64}{25}$
    \subsection*{Задача}
    В связке есть ключи, человек пробует все ключи, пока не подберёт нужный. Сколько в среднем ключей он переберёт?\\
    $\xi$ --- количество попыток. Показана вероятность успеха.
    \[\begin{tabular}{c|c|c|c|c}
        $\xi$ & 1 & 2 & $\dots$ & 10\\
        \hline
        $p$ & $\frac{1}{10}$ & $\frac{1}{10}$ & $\dots$ & $\frac{1}{10}$
    \end{tabular}\]
    $E\xi = \frac{55}{10} = 5,5$
    \subsection*{Задача}
    Есть 4 стула, в одном из них драгоценности. Человек ломает стулья, пока не найдёт драгоценности. Сколько стульев в среднем будет сломано?
    $\xi$ --- количество сломанных стульев.
    \[\begin{tabular}{c|c|c|c|c}
        $\xi$ & 1 & 2 & 3 & 4\\
        \hline
        $p$ & $\frac{1}{4}$ & $\frac{3}{4}\frac{1}{3}$ & $\frac{3}{4}\frac{2}{3}\frac{1}{2}$ & $\frac{3}{4}\frac{2}{3}\frac{1}{2}1$
    \end{tabular}\Rightarrow \begin{tabular}{c|c|c|c|c}
        $\xi$ & 1 & 2 & 3 & 4\\
        \hline
        $p$ & $\frac{1}{4}$ & $\frac{1}{4}$ & $\frac{1}{4}$ & $\frac{1}{4}$
    \end{tabular}\]
    $E\xi = \frac{1}{4}(1 + 2 + 3 + 4) = \frac{10}{4}\\
    \mathcal{D}\xi = E(xi^2) - 2,5^2 = \frac{30}{4} - 6,25 = \frac{20}{16} = \frac{5}{4}$
    \subsection*{Задача}
    $P(\xi = k) = \frac{c}{k(k + 1)}\\
    c - ?,\ P(\xi \leq 10) - ?,\ P(10 \leq \xi \leq 20) - ?\\
    \displaystyle \sum_{k = 1}^{\infty} \frac{c}{k(k + 1)} = 1\Rightarrow c\sum_{k = 1}^{\infty} \left( \frac{1}{k} - \frac{1}{k - 1} \right) = 1\\
    c\lim_{k\to \infty} \left(1 - \frac{1}{k}\right) = 1\Rightarrow c\cdot 1 = 1\Rightarrow c = 1\\
    P(\xi = k) = \frac{1}{k(k + 1)}\\
    P(\xi \leq 10) = \sum_{k = 1}^{\infty} \frac{1}{k(k + 1)} = 1 - \frac{1}{11} = \frac{10}{11}\\
    P(10 \leq \xi \leq 20) = P(\xi \leq 20) - p(\xi \leq 9) = (1 - \frac{1}{21}) - (1 - \frac{1}{10}) = \frac{11}{210}\\
    E\xi = \sum_{k = 1}{\infty} \frac{1}{k(k + 1)}k = \sum_{k = 1}^{\infty} \frac{1}{k + 1}$ - расходится.\\
    $E\xi = +\infty$
    \begin{center}
        11 октября
    \end{center}
    Разбор задач из домашнего задания.
    $F_{\xi}(x) = \begin{cases}
        1 - \frac{1}{x},\ x\geq 1\\
        0,\ x < 1
    \end{cases}$\\
    Найти a: $P(\xi > a) = \frac{1}{3}\\
    P(\xi > a) = 1 - F_{\xi}(a) = \frac{1}{3}\\
    F_{\xi}(x) = \frac{2}{3}\\
    \begin{cases}
        1 - \frac{1}{x} = \frac{2}{3}\\
        x \geq 1
    \end{cases}\Rightarrow \begin{cases}
        \frac{1}{3} = \frac{1}{x}\\
        x\geq 1
    \end{cases}\\
    x = 3\Rightarrow a = 3$
    \begin{center}
        Распределения
    \end{center}
    Распределение Бернулли: $\xi \sim \operatorname{Ber}(p)\\
    \begin{tabular}{c|c|c}
        $\xi$ & 0 & 1\\
        \hline
        $p$ & 1 - p = q & p
    \end{tabular},\ E\xi = p,\ \mathcal{D}\xi = pq$\\
    Биномиальное распределение.\\
    Распределение Пуассона.\\
    Геометрическое распределение.

    \subsection*{Задача}
    С вероятностью $\frac{5}{6}$ яблоко падает недалеко от яблони, всего 10 яблок.
    Случайная величина $\xi$ --- количество яблок, которые упали недалеко от яблони.\\
    $\xi \sim \operatorname{Bi}(10,\ \frac{5}{6})$\\
    $E\xi = \frac{25}{3},\ \mathcal{D}\xi = \frac{25}{18}$.
    \subsection*{Задача}
    $\xi \sim \operatorname{Bi}(4,\ 0.7)\\
    E\xi = 2,8\\
    \mathcal{D}\xi = 0,24\\
    Z_{\frac{8}{1000}} = \min(x,\ F(x) \geq \frac{8}{1000})\\
    F(0) = \frac{81}{10000} > \frac{8}{1000}\Rightarrow Z_{\frac{8}{1000}} = 0$
    \subsection*{Задача}
    Три стрелка, вероятность попадания:
    \begin{enumerate}
        \item 0,7
        \item 0,6
        \item 0,5
    \end{enumerate}
    Каждый выстрелил по одному разу, какое среднее количество попаданий?\\
    $\xi_1 \sim \operatorname{Bi}(1,\ 0.7),\ \xi_2 \sim \operatorname{Bi}(1,\ 0.6),\ \xi_3\sim \operatorname{Bi}(1,\ 0.5)\\
    E\xi = E(\xi_1 + \xi_2 + \xi_3) = E\xi_1 + E\xi_2 + E\xi_3 = 1.8$\\
    Так как события независимы, их ковариация равна 0, поэтому:\\
    $\mathcal{D}\xi = \mathcal{D}\xi_1 + \mathcal{D}\xi_2 + \mathcal{D}\xi_3 = 0,7\cdot 0,3 + 0,6\cdot 0,4 + 0,5\cdot 0,5 = 0,7$
    \subsection*{Задача}
    Известно, что бутерброд падает маслом вниз с вероятностью $0,999$. Некий экспериментатор уронил бутерброд 1000 раз. Какова вероятность, что бутерброд упадёт маслом вверх более 2 раз?
    $\xi$ --- маслом вверх.\\
    $\xi \sim \operatorname{Bi}(1000, 0.001)$\\
    Хотелось бы отпуассонить (апроксимировать распределением Пуассона) это распределение.\\
    После апроксимации: $\Delta = \left|C_n^k p^k q^{n - k} - \frac{e^{-np}(np)^k}{k!}\right|\leq np^2$\\
    $\Delta = 0.001$. Можем апроксимировать: $\xi \sim \Pi(1),\ \lambda = np = 1\\
    P(\xi > 2) = 1 - P(\xi\leq 2) = 1 - P(\xi = 0) - P(\xi = 1) - P(\xi = 2) =\\
    = 1 - \frac{e^{-1}1^0}{0!} - \frac{e^{-1} 1}{1!} - \frac{e^{-1}1^2}{2!} = 1 - \frac{5}{2e}$
    \subsection*{Задача}
    $\xi \sim \operatorname{G}(0,2)\Rightarrow E\xi = 5$. Наиболее вероятное значение: $P(\xi = k) = q^{k - 1}p \leq p = 0,2\Rightarrow k = 1$
    \begin{center}
        Семинар 1 ноября
    \end{center}
    Задача 34 страница 93.\\
    \begin{tikzpicture}[scale=2]
        \draw[->] (-2, 0) -- (3, 0) node[anchor=west]{$x$};
        \draw[->] (0, -1) -- (0, 1) node[anchor=west]{$y$};
        \draw (-1, 0.6) -- node[anchor=south]{$3h$} node[anchor=north]{\small$S_1$} (0, 0.6);
        \draw (1, 0.2) -- node[anchor=south]{$h$} node[anchor=north]{\tiny$S_2$} (2, 0.2);
        \draw[dashed] (-1, 0.6) -- (-1, 0) node[anchor = north]{-1};
        \draw[dashed] (1, 0.2) -- (1, 0) node[anchor = north]{1};
        \draw[dashed] (2, 0.2) -- (2, 0) node[anchor = north]{2};
    \end{tikzpicture}\\
    $S = S_1 + S_2 = 3h + h = 4h = 1\Rightarrow h = \frac{1}{4}$\\
    $F(x) \displaystyle = \begin{cases}
        0,\ x\in (-\infty,\ -1)\\
        \displaystyle \int_{-\infty}^{-1} 0\, dt + \int_{-1}^{x} \frac{3}{4}\, dt = \frac{3}{4}(x + 1),\ x\in [-1,\ 0]\\
        \displaystyle \int_{-\infty}^{-1} 0\, dt + \int_{-1}^{1} \frac{3}{4}\, dt + \int_0^{x} 0\, dt = \frac{3}{4},\ x \in (0,\ 1)\\
        \displaystyle \int_{-\infty}^{-1} 0\, dt + \int_{-1}^{1} \frac{3}{4}\, dt + \int_0^{1} 0\, dt + \int_{1}^{2} \frac{1}{4}\, dt = \frac{3}{4} + \frac{1}{4}(x - 1),\ x\in [1,\ 2]\\
        1,\ x \in (2,\ +\infty)
    \end{cases}$\\
    $M[]$ --- матожидание.\\
    $M[(2 - x)(3 - x)] = M[6 - 5x + x^2] = M[6] - M[5x] + M[x^2] =\\
    = 6 - 5\left(\displaystyle \int_{-1}^{0} x \frac{3}{4}\, dx + \int_{1}^2x\cdot \frac{1}{4}\, dx\right) + \left(\int_{-1}^{0}x^2\cdot \frac{3}{4}\, dx + \int_{1}^{2}x^2 \cdot \frac{1}{4}\, dx\right)\\
    \mathcal{D}[2 - 3x] = 9\mathcal{D}[x] = 3(M[x^2] - M^2[x])$\\
    \begin{center}
        Распределение Гаусса
    \end{center}
    $\xi \sim N(m, \sigma^2)\\
    f(x) = \frac{1}{\sqrt{2\pi}\sigma} e^{-\frac{(x - m)^2}{2\sigma^2}}\\
    \Phi_0(x) =\displaystyle \int_0^x \frac{1}{\sqrt{2\pi}} e^{-\frac{t^2}{2}}\, dt\\
    P(\alpha < \xi < \beta) = \Phi_0\left( \frac{\beta - m}{\sigma} \right) - \Phi_0\left( \frac{\alpha - m}{\sigma} \right)$\\
    Находим значения $\Phi_0$ через таблицу функции Лапласа.\\
    Если $x > 5$, то $\Phi_0(x) = 0.5$\\
    \subsection*{Задача}
    $f(x) = \frac{1}{\sqrt{72\pi}} e^{-\frac{(x + 3)^2}{72}}\Rightarrow \xi \sim N(-3,\ 36)\\
    P(0 < \xi < 12) = \Phi_0\left(\frac{12 - 12}{3}\right) - \Phi_0\left( \frac{0 - 12}{3} \right) = \Phi_0(0) - \Phi_0(-4) = 0 + \Phi_0(4) =\\
    = 0.4999683$\\
    Второй начальный момент $\mu_2 = E\xi^2 = \mathcal{D}\xi + (E\xi)^2 = 9 + 12^2 = 153\\
    \mathcal{D}(5 - 3\xi) = 9\mathcal{D}(\xi) = 9\cdot 9$\\
    \subsection*{Задача 51 стр 94}
    По условию $\xi \sim N\big(0,\ 0.4^2\big)$. Нужно посчитать $P(-0.7 < \xi < 0.7) = \Phi_0\left( \frac{0.7}{0.4} \right) - \Phi_0\left( \frac{-0.7}{0.4} \right) = 2\Phi_0\left( \frac{7}{4} \right)\approx 2\cdot 0.46 = 0.92\\
    \text{В общем случае:} P(|\xi - E\xi| < \delta) = 2\Phi_0\left( \frac{\delta}{\sigma} \right)$\\
    Пусть $\eta$ --- количество годных шариков из 50 изготовленных. Заметим, что $\eta\sim Bi(50,\ 0.92)\Rightarrow E\eta = n\cdot p = 50\cdot 0.92 = 46$\\
    \subsection*{Задача}
    $\xi \sim N(0,\ 1)\\
    Z_{0.9} - ?\Rightarrow \Phi_0(Z_{0.9}) = 0.9 - 0.5 = 0.4$ (так как интегрировать надо от $-\infty$, а не от 0, а $\displaystyle\int_{-\infty}^{0}\dots = \frac{1}{2}$), тогда $Z_{0.9} \approx 1.29$.\\
    Отсюда можно посчитать $Z_{0.1} = -Z_{0.9} = -1.29$\\
    Интересный факт $P(-1.65 < \xi < 1.65) = 0.9$. И ещё один:
    \[P(|\xi| < 1.96) = 0.95\]
    1.96 является знаменитой точкой, использующейся в подсчёте доверительных интервалов (случайная величина попадает в него с вероятностью $95$ процентов, немало).
    \subsection*{Теория}
    $\xi \sim f_{\xi} (x),\ \eta = \phi(\xi),\ f_{\eta}(y) - ?$\\
    Пусть $\varphi(x)$ --- монотонная функция. $y = \varphi(x),\ x = \varphi^{-1}(y)\\
    f_{\eta}(y) = f_{\xi} \big( \varphi^{-1}(y) \big) \left| \big(\varphi^{-1}(y)\big)' \right|$\\
    Разобьём $\varphi$ на отрезки монотонности (пусть их $k$ штук):
    \[f_{\eta}(y) = \sum_{i = 1}^{k} f_{\xi}\left( \varphi^{-1}_{i}(y) \right)\left| \big(\varphi^{-1}_i(y)\big)' \right|\]
    \subsection*{Задача}
    $\xi\sim N(0,\ \sigma^2)$
    \begin{enumerate}
        \item[a.] $\eta = \xi^3,\ f_{\eta}(y) -?$, $f_{\eta}(y) = \frac{1}{\sqrt{2\pi} \sigma}e^{-\frac{\left( y^{\frac{1}{3}}\right)^2}{2\sigma^2}}\cdot \frac{1}{3}\cdot y^{-\frac{2}{3}} $
        \item[b.] $\eta = \xi^2$. Разбиваем на интервалы монотонности:
        \[\varphi_1^{-1}(y) = -\sqrt{y},\ \varphi_2^{-1}(y) = \sqrt{y}\Rightarrow \left(\varphi_1^{-1}\right)' = -\frac{1}{2}\sqrt{y} = -\left(\varphi_2^{-1}\right)'\]
        \[f_{\eta}(y) = \frac{1}{\sqrt{2\pi}\sigma} e^{-\frac{y}{2\sigma^2}} \cdot \left| \frac{1}{2\sqrt{y}} \right|\cdot 2,\ x > 0\]
        При $x \leq 0: f_{\eta}(y) = 0$
    \end{enumerate}
    \begin{center}
        Семинар 8 ноября
    \end{center}
    Разбор домашнего задания.\\
    $f(x) = \begin{cases}
        \frac{1}{\pi},\ x\in [\frac{-\pi}{2},\ \frac{\pi}{2}]\\
        0,\ \text{иначе}
    \end{cases}$\\
    \[f(x)\]\[\begin{tikzpicture}[every circle node/.style={fill, minimum width=0.25mm}]
        \draw[->] (-3, 0) -- (3, 0) node[anchor=west]{$x$};
        \draw[->] (0, -1) -- (0, 1) node[anchor=west]{$y$};
        \draw (-1.6, 0.33) node[circle]{} -- (1.6, 0.33) node[circle]{};
    \end{tikzpicture}\]
    $\eta = \sin \xi = \varphi(\xi)\\
    \varphi^{-1}(y) = \arcsin y\\
    \left(\varphi^{-1}(y)\right)' = \frac{1}{\sqrt{1 - y^2}}\\
    f_{\eta}(y) = f_{\xi} (\varphi^{-1}(y))\left( \varphi^{-1}(y) \right)' = \begin{cases}
        \frac{1}{\pi}\cdot \frac{1}{\sqrt{1 - y^2}},\ y\in(-1,\ 1)\\
        0,\ \text{иначе}
    \end{cases}$
    \[f(x)\]\[\begin{tikzpicture}
        \draw[->] (-4, 0) -- (4, 0) node[anchor=west]{$x$};
        \draw[->] (0, -1) -- (0, 1) node[anchor=west]{$y$};
        \draw (0, 0.33) -- (3.14, 0.33);
        \draw[dashed] (3.14, 0.33) -- (3.14, 0) node[anchor=north]{$\pi$};
    \end{tikzpicture}\]
    $f_{\xi}(x) = \begin{cases}
        \frac{1}{\pi},\ x\in[0,\ \pi]\\
        0,\ \text{иначе}
    \end{cases}\\
    \eta = \sin \xi\\
    \varphi^{-1}_1(y) = \arcsin(y)\\
    \varphi^{-1}_2(y) = \pi - \arcsin(y)\\
    \left( \varphi^{-1}_1(y) \right)' = \frac{1}{\sqrt{1 - y^2}}\\
    \left( \varphi^{-1}_2(y) \right)' = -\frac{1}{\sqrt{1 - y^2}}\\
    f_{\eta}(y) = \sum\limits_{i = 1}^{2} f_{\xi} \varphi^{-1}_i(y) \left|\left(\varphi^{-1}_i(y) \right)'\right| = \begin{cases}
        \frac{2}{\pi}\cdot \frac{1}{\sqrt{1 - y^2}},\ y\in(0,\ 1)\\
        0,\ \text{иначе}
    \end{cases}$
    \subsection*{Многомерные случайные величины}
    $\begin{tabular}{c|ccc}
        $\xi_1,\ \xi_2$ & $y_1,\dots$ & $y_j$ & $\dots,\ y_k$\\
        \hline
        $x_1$ & \vdots & \vdots\\
        $\vdots$ & \vdots & \vdots\\
        $x_i$ & \dots & $p_{i\, j}$ \\
        \vdots & \\
        $x_m$
    \end{tabular}$\\
    $p_{i\, j} = p(\xi_1 = x_i,\ \xi_2 = y_j)\\
    \displaystyle\sum_i \sum_{j} p_{i\, j} = 1\\
    p_{i\, \bullet} = \sum_{j = 1}^{m} p_{i\, j},\ p_{\bullet\, j} = \sum_{i = 1}^{k} p_{i\, j}\\
    \operatorname{cov}(\xi_1,\ \xi_2) = E(\overset{\circ}{\xi_1}\overset{\circ}{\xi_2}) = E\xi_1 \xi_2 - E\xi_1 E\xi_2\\
    E\xi_i \xi_2 = \sum_{i = 1}^{m}\sum_{j = 1}^{k} x_i \cdot y_j \cdot p_{i\, j}$
    \subsection*{Задача}
    Построить одномерное распределение, определить матожидание и дисперсию двумерной случайной величины.
    \[\begin{tabular}{c|c|c|c}
        $\downarrow\xi,\ \eta\rightarrow$ & -1 & 0 & 1\\
        \hline
        -2 & 0.1 & 0.2 & 0.3\\
        2 & 0.1 & 0.1 & ?
    \end{tabular}\]
    За знаком ? стоит вероятность 0.2, так как сумма всех вероятностей в таблице должна равняться 1.\\
    Построим табличку, где $p_{i\, j} = p_{i\, \bullet}\cdot p_{\bullet\, j}$\\
    Строить не будем, так как уже $p_{1\, 1} \neq p_{1\, \bullet}\cdot p_{\bullet\, 1}$, так как $0.1 \neq 0.2\cdot 0.6 = 0.12\\
    E\xi = -1.2  + 0.8 = -0.4\\
    E\eta = -0.2  + 0.5 = 0.3\\
    \begin{tabular}{c|c|c|c|c|c|c}
        $\xi + \eta$ & -3 & -2 & -1 & 1 & 2 & 3\\
        \hline
         & 0.1 & 0.2 & 0.3 & 0.1 & 0.1 & 0.2
    \end{tabular}\\
    E(\xi + \eta) = -0.3 - 0.4 - 0.3 + 0.1 + 0.2 + 0.6 = -0.1 = E\xi + E\eta\\
    \mathcal{D}\xi = 4 - 0.16 - 3.84\\
    \mathcal{D}\eta = 0.2 + 0.5 - 0.09 = 0.61\\
    \mathcal{D}(\eta + \xi) = 9\cdot 0.3 + 4\cdot 0.3 + 0.4\cdot 1 - 0.01 = 2.7 + 1.2 + 0.4 - 0.01 = 4.29 = E(\eta + \xi)^2 - \left(E(\eta + \xi)\right)^2\\
    \begin{tabular}{c|c|c|c}
        $\xi\cdot \eta$ & -2 & 0 & 2\\
        \hline
         & 0.4 & 0.3 & 0.3
    \end{tabular}\\
    \operatorname{cov}(\xi,\ \eta) = E(\xi \eta) - E\xi E\eta = -0.2 + 0.12 = -0.08$\\
    Проверим корректность
    \[\mathcal{D}(\eta + \xi) = \mathcal{\xi} + \mathcal{\eta} + \operatorname{cov}(\xi,\ \eta) = 0.61 + 3.84 - 2\cdot 0.08 = 4.29\]
    Верно.
    \subsection*{Задача}
    Изучение таблички (доделать дома).\\
    $\xi$ --- материальное положение.\\
    $\eta$ --- уровень образования.\\
    Определить зависимость и коэффициент корреляции.\\
    $\begin{tabular}{c|c|c|c}
        $\eta,\ \xi$ & 1 & 2 & 3\\
        \hline
        1 & 0.083 & 0.035 & 0.001\\
        2 & 0.31 & 0.375 & 0.026\\
        3 & 0.04 & 0.116 & 0.014
    \end{tabular}$
    \begin{center}
        Семинар 15 ноября
    \end{center}
    \subsection*{Задача 7 стр 132}
    Запишем условие в виде таблицы:
    \[\begin{tabular}{c|c|c|c}
        B$\setminus$ A & 1 & 0 & sum\\
        \hline
        1 & $\frac{25}{1000}$ & $\frac{2}{100}$ & $\frac{45}{1000}$\\
        \hline
        0 & $\frac{5}{1000}$ & $\frac{95}{100}$ & $\frac{955}{1000}$\\
        \hline
        sum & $\frac{3}{100}$ & $\frac{97}{100}$ & 1
    \end{tabular}\]
    Посчитаем вероятность произведения:
    \[\begin{tabular}{c|c|c}
        AB & 1  &0\\
        \hline
        p & $\frac{25}{1000}$ & $\frac{975}{1000}$
    \end{tabular}\]
    $\cov(A, B) = E(AB) - EA\cdot EB = \frac{25}{1000} - \frac{3}{100}\cdot\frac{45}{1000} =  \frac{2365}{10000}$
    \subsection*{Задача 1} я решал, перепишу с фотографии.
    \subsection*{Задача 2}
    Дано $f(x,\ y) = \begin{cases}
        \frac{1}{4}\sin x \sin y,\ \text{если } 0\leq x,\ y \leq \pi\\
        0, \text{ иначе}
    \end{cases}$\\
    $\xi = (\xi_1,\ \xi_2),\ P(\xi\in \mathcal{D})$
    \[f_{\xi_1}(x) = \int\limits_0^{\pi} \frac{1}{4}\sin x \sin y\, dy = \left.\frac{1}{4}\sin x (-\cos y)\right|_0^{\pi} = \frac{1}{2}\sin x,\ x\in [0,\ \pi]\]
    \[f_{\xi_2}(y) = \int\limits_0^{\pi} \frac{1}{4}\sin x \sin y\, dx = \left.\frac{1}{4}\sin y (-\cos x)\right|_0^{\pi} = \frac{1}{2}\sin y,\ y\in [0,\ \pi]\]
    Заметим, что $f_{\xi}(x,\ y) = f_{\xi_1}(x) f_{\xi_2}(y)$, значит $\xi_1,\ \xi_2$ независимы $\Rightarrow\\
    \Rightarrow\cov(\xi_1,\ \xi_2) = 0$\\
    \[\displaystyle P(\xi \in \mathcal{D}) = \int\limits_{0}^{\frac{\pi}{4}}\int\limits_{\frac{\pi}{6}}^{\frac{\pi}{3}} \frac{1}{4}\sin x \sin y\, dy \, dx = \int_0^{\frac{\pi}{4}} \frac{1}{4}\sin x(-\frac{1}{2} + \frac{\sqrt{3}}{2})\, dx =\\
    = \int\limits_0^{\frac{\pi}{4}} \frac{\sqrt{3} - 1}{8} \sin x \, dx = \frac{\sqrt{3} - 1}{8}\left(-\frac{\sqrt{2}}{2} + 1\right) = \frac{\sqrt{3} - 1}{8} - \frac{\sqrt{2}(\sqrt{3} - 1)}{16}\]
    Посчитаем вектор матожидания:\\
    \[E\xi_1 = E\xi_2 = \int\limits_0^{\pi} \frac{x}{2}\sin x\, dx = \int\limits_0^{\pi}\frac{x}{2}\, d\cos x = \left.-\frac{x\cos x}{2}\right|_0^{\pi} - \int\limits_0^{\pi} -\frac{\cos x}{2}\, dx =\]
    \[= \frac{\pi}{2} + \left.\frac{\sin x}{2}\right|_0^{\pi} = \frac{\pi}{2}\]
    Получается $E\xi = (\frac{\pi}{2},\ \frac{\pi}{2})$.\\
    $K_{\xi} = \begin{pmatrix}
        ? & 0\\
        0 & ?
    \end{pmatrix}$. Найдём дисперсии:
    \[E\xi_1^2 = E\xi_2^2 = \int\limits_0^{\pi} \frac{x^2}{2} \sin x\, dx = \left.-\frac{x^2\cos x}{2}\right|_0^{\pi} + \int\limits_0^{\pi}x\cos x\, dx = \frac{\pi^2}{2} - \int\limits_0^{\pi} \sin x \, dx = \frac{\pi^2}{2} - 2\]
    $\mathcal{D}\xi_1 = \mathcal{D}\xi_2 = \frac{\pi^2}{2} - 2 - \frac{\pi^2}{4} = \frac{\pi^2}{4} - 2$\\
    Итак, $K_{\xi} = \begin{pmatrix}
        \frac{\pi^2}{4} - 2 & 0\\
        0 & \frac{\pi^2}{4} - 2
    \end{pmatrix}$
    \begin{center}
        Семинар 22 ноября
    \end{center}
    \subsection*{Задача 1}
    $\xi_1 \sim R(0,\ 1),\quad \xi_2 \sim R(0,\ 1),\ \xi_2$ и $\xi_1$ независимы.\\
    Найти $\xi = \xi_1 + \xi_2$. Для решения воспользуемся формулой свёртки:
    \[f_{\xi}(z) = \int\limits_{-\infty}^{+\infty} f_{\xi_1}(x) f_{\xi_2}(z - x)\, dx\]
    \[f_{\xi_1}(x) = \begin{cases}
        1,\ x\in (0,\ 1)\\
        0,\ \text{иначе}
    \end{cases}\]
    \[f_{\xi_2}(z - x) = \begin{cases}
        1,\ x\in (z - 1,\ z)\\
        0,\ \text{иначе}
    \end{cases}\]
    Рассмотрим случаи:
    \begin{enumerate}
        \item[]$z\in(0,\ 1)$ Тогда нас интересует интервал $(0,\ z)$
        \item[]$z\in(1,\ 2)$ Тогда --- $(z - 1, 1)$
        \item[] Во всех остальных случаях произведение функций будет 0.
    \end{enumerate}
    Получаем плотность распределения $\xi$:
    \[f_{\xi}(z) = \begin{cases}
        z,\ z\in (0,\ 1)\\
        2 - z,\ z\in (1,\ 2)\\
        0,\ \text{иначе}
    \end{cases}\]
    Дана таблица двумерного дискретного распределения.
    \subsection*{Задача 5 стр. 131-132}
    $E(y\ \big|\ x = 1)$
    \[\begin{tabular}{|c|c|c|}
        \hline
        $y\setminus x$ & -1 & 1\\
        \hline
        1 & $\frac{1}{6}$ & $\frac{1}{3}$\\
        \hline
        2 & 0 & $\frac{1}{6}$\\
        \hline
        3 & $\frac{1}{3}$ & 0\\
        \hline
    \end{tabular}\]
    $P(x = 1) = \frac{1}{2},\\
    P(y = 1\ |\ x = 1) = \frac{2}{3},\\
    P(y = 2\ |\ x = 1) = \frac{1}{3},\\
    P(y = 3\ |\ x = 1) = 0$\\
    $E(y\ \big|\ x = 1) = 1\cdot\frac{2}{3} + 2\cdot\frac{1}{3} = \frac{4}{3}\\
    E(y) = 1\cdot\frac{1}{2} + 2\frac{1}{6} = \frac{11}{6}$
    \subsection*{Задача про табличку с данными}
    $P(\eta = 1\ |\ \xi = 1) = \frac{0.085}{0.119} \approx 0.697\\
    P(\eta = 2\ |\ \xi = 1) = \frac{0.035}{0.119} \approx 0.294
    P(\eta = 3\ |\ \xi = 1) = \frac{0.001}{0.119} \approx 0.009\\
    E(\eta\ |\ \xi = 1) = 1\cdot 0.697 + 2\cdot 0.294 + 3\cdot 0.009 = 1.312$
    \begin{center}
        Семинар 29 ноября.
    \end{center}

    Вспоминаем неравенство Чебышёва:
    $E|\xi|^r < \infty\Rightarrow \forall \varepsilon > 0\quad P(|\xi| > \varepsilon) \leq \frac{E|\xi|^r}{\varepsilon^r}$\\
    \subsection*{Задача}
    Найди вероятность того, что в некоторой области скорость ветра превысит 80 км/ч, если:\\
    a) $E\xi = 16 \frac{\text{км}}{\text{ч}}$\\
    б) $E\xi = 16\frac{\text{км}}{\text{ч}},\ \sigma = 4 \frac{\text{км}}{\text{ч}}$\\
    В пункте а: $P(\xi > 80) \leq \frac{E\xi}{80} = \frac{16}{80} = 0.2$\\
    В пункте б: можем воспользоваться $\sigma$ для нахождения матожидания квадрата:
    \[E\xi^2 = \mathcal{D}\xi + (E\xi)^2 = 16 + 256 = 272 < \infty\]
    $P(\xi > 80) \leq \frac{E\xi^2}{80^2} = \frac{272}{80^2} = 0.0425$
    \subsection*{Задача}
    Пусть $\xi$ = \{Количество доживших до 50 лет среди 1000 новорождённых\}\\
    $\xi \sim Bi(1000,\ 0.87)$. Можем оценить $P(|\underset{\eta}{\underbrace{\frac{\xi}{1000} - 0.87}}| > 0.04) \leq \frac{\mathcal{D}\frac{\xi}{n}}{0,04^2}$\\
    Из свойств биномиального распределения: $E\xi = np,\ E\frac{\xi}{n} = p,\ \mathcal{D}\xi = npq,\ \mathcal{D}\frac{\xi}{n} = \frac{pq}{n}$\\
    Тогда $P(|\eta| > 0,04) \leq 0,07\Rightarrow P(|\eta| < 0,04) \geq 1 - 0,07 = 0,93$.
    \subsection*{Задача}
    \[f(x,\ y) = \begin{cases}
        axy,\ 0 < y < 1, y < x < y + 1\\
        0,\ \text{иначе}
    \end{cases}\]
    \[\begin{tikzpicture}
        \draw[->] (-2, 0) -- (2, 0) node[anchor=west]{x};
        \draw[->] (0, -2) -- (0, 2) node[anchor=south]{y};
        \draw (0, 0) -- (1, 1) node[anchor=south]{x = y};
        \draw (1, 0) -- (2, 1);
        \draw[dashed] (1, 1) -- (2, 1);
    \end{tikzpicture}\]
    $\displaystyle \int_0^1\int_y^{y + 1} f(x,\ y)\, dx\, dy = \int_0^1 \int_y^{y + 1} axy\, dx\, dy = \int_0^1\int_y^{y + 1} a\left.\frac{x^2}{2} y\, \right|_y^{y + 1} dy = \int_0^1 ay^2 + \frac{ay}{2}\, dy = \frac{a}{3} + \frac{a}{4} = 1\Rightarrow a = \frac{12}{7}$\\
    $f_{\xi_1}(x) = \begin{cases}
        \displaystyle\int_0^x \frac{12}{7} xy\, dy = \frac{6x^3}{7}, 0 < x < 1\\
        \displaystyle\int_{x - 1}^{1} \frac{12}{7} xy\, dy = \frac{6y^2}{7} x |_{x - 1}^{1} = \frac{6x^2}{7}(2 - x),\ 1 < x < 2\\
        0,\ \text{иначе}
    \end{cases}$\\
    $f_{\xi_2}(y) = \begin{cases}
        \displaystyle\int_y^{y + 1} \frac{12}{7}xy\, dx = \frac{6y}{7}(2y + 1),\ 0 < y < 1\\
        0,\ \text{иначе}
    \end{cases}$\\
    $f_{\xi_1}(x\ |\ y) = \begin{cases}
        \frac{\frac{12}{7}xy}{\frac{6}{7} y(2y + 1)} = \frac{2x}{2y + 1},\ y < x < y + 1\\
        \text{0},\ \text{иначе}
    \end{cases}$\\
    $f_{\xi_2}(y\ |\ x) = \begin{cases}
        \frac{\frac{12}{7}xy}{\frac{6}{7}x^3}=\frac{2y}{x^2} 0 < x < 1,\ 0 < y < 1\\
        \frac{\frac{12}{7}xy}{\frac{6}{7}x^2(2 - x)} = \frac{2y}{x(2 - x)} 1 < x < 2,\ 0 < y < 1\\
    \end{cases}$\\
    $\displaystyle E(\xi_2\ |\ \xi_1 = 1) = \int_{-\infty}^{+\infty} x f_{\xi_1}(x\ |\ y)\, dx = \int_1^2 \frac{x}\cdot \frac{2x}{3}\, dx = \left.\frac{2x^3}{9}\right|_1^2 = \frac{14}{9}$
    \begin{center}
        Семинар 6 декабря
    \end{center}
    Пусть $\zeta = (\xi,\ \eta)$, где $\xi$ --- рост человека, а $\eta$ --- вес.
    $\zeta \sim N(m_{\zeta},\ k_{\zeta}),\ m_{\zeta} = (175,\ 74)^T,\ K_{\zeta} = \begin{pmatrix}
        49 & 26\\
        26 & 36
    \end{pmatrix}$\\
    $E(\xi - \eta) = E(\xi) - E(\eta) = 175 - 74 = 101\\
    \mathcal{D}(\xi - \eta) = \mathcal{D}(\xi) + \mathcal{D}(\eta) + 2\cov(\xi,\ -\eta) = 49 + 36 - 2\cdot 26 = 33\\
    \xi - \eta \sim N(101,\ 33)$\\
    Считается, что человек страдает избыточным весом, если $\xi - \eta \leq 90$. Посчитаем вероятность этого события:
    $P(\xi - \eta \leq 90) = \Phi_0\left( \frac{90 - 101}{\sqrt{33}} \right) - \Phi_0(-\infty) = -\Phi_0\left(\sqrt\frac{11}{3}\right) + \frac{1}{2} = 0,5 - 0,4719 = 0,0281$\\
    По теореме о нормальной корреляции:\\
    $E(\xi\ |\ \eta = y) = E(\xi) + \frac{\cov(\xi,\ \eta)}{\mathcal{D}\eta}(y - E(\eta))\\
    \mathcal{D}(\xi\ |\ \eta = y) = \mathcal{D}(\xi) - \frac{(\cov(\xi,\ \eta))^2}{\mathcal{D}\eta}$\\
    Нас интересуют следующие величины:\\
    $E(\eta\ |\ \xi = 180) = 74 + \frac{26}{49}\cdot (180 - 175) = 76,65$\\
    $E(\eta\ |\ \xi = 190) = 74 + \frac{26}{49}\cdot (190 - 175) = 81,96$\\
    $\forall a\ \mathcal{D}(\eta\ |\ \xi = a) = 36 - \frac{26^2}{49} = 22,2$
    \subsection*{Центральная предельная теорема}
    Считается, что шаг пешехода распределён равномерно от 0,7 до 0,8 метра. Какова вероятность, что за 10\,000 шагов человек пройдёт от 7,49 до 7,51 километра?\\
    $\xi_i$ --- длина i-го шага, $\xi_i\sim R(0,7;\ 0,8)$\\
    $n = 10\,000,\ S = \displaystyle\sum_{k = 1}^{n} \xi_k.$\\
    $E\left( \displaystyle\sum_{k = 1}^{n} \xi_k \right) = \sum_{k = 1}^{n} E(\xi_k) = n\cdot \frac{0,7 + 0,8}{2} = 0,75 n = 7500\\
    \mathcal{D}(\xi_i) = \frac{(0,8 - 0,7)^2}{12} = \frac{1}{1200},\ \mathcal{D}(S) = \frac{10\,000}{1200} = \frac{100}{12} = \frac{25}{3}$\\
    $\frac{S - 7\,500}{\sqrt{\mathcal{D}(S)}} = \frac{S - 7500}{\frac{5}{\sqrt{3}}}\sim U(0,\ 1)\Rightarrow P\left( 7490 < S < 7510 \right) = 2\Phi_0\left(\frac{10}{\frac{5}{\sqrt{3}}}\right) = 2\Phi_0(2\sqrt{3}) \approx 2\cdot 0,4997 = 0,9994$
    \subsection*{Задача}
    Мальчик рождается с вероятностью 0,52. Найти вероятность того, что среди 1\,000 новорождённых девочек окажется не меньше, чем мальчиков.\\
    $\xi \sim Bi(1000;\ 0,52),\ E\xi = 520,\ \mathcal{D} = 520\cdot 0,48 = 249,6$\\
    Найдём $P(\xi \leq 500) = \Phi_0\left( \frac{500 - 520}{\sqrt{249,6}} \right) - \Phi_0(-\infty) = \frac{1}{2} - \Phi_0\left( \frac{20}{15,8} \right) \approx \frac{1}{2} - 0,3962 = 0,1038$\\
    Проводим $n$ испытаний, частота успеха $\frac{\xi_n}{n}\Rightarrow E\frac{\xi_n}{n} = \frac{np}{n} = p,\ \mathcal{D}\frac{\xi_n}{n} = \frac{pq}{n}$\\
    Тогда при $n\to\infty$:
    \[\frac{\frac{\xi_n}{n} - p}{\sqrt{\frac{pq}{n}}}\xrightarrow{d} U \sim N(0,\ 1)\]
    Пусть хотим найти $P\left( \left| \frac{\xi_n}{n} - p \right| < \delta\right) = 2\Phi_0\left( \frac{\delta\sqrt{n}}{\sqrt{pq}} \right)$
    \subsection*{Задача}
    $\xi_n \sim Bi(1\,000;\ 0,87)$. Нас интересует $P \left( \left| \frac{\xi_n}{n} - 0,87 \right| < 0,04 \right)\\
    E\left( \frac{\xi_n}{n} \right) = p = 0,87,\ \mathcal{D}\left( \frac{\xi_n}{n} \right) = \frac{npq}{n^2} = \frac{pq}{n} = \frac{8,7\cdot 1,3}{1000} = 11,31$\\
    Тогда можно посчитать $P \left( \left| \frac{\xi_n}{n} - 0,87 \right| < 0,04 \right) = 2\Phi_0\left( \frac{0,04}{\sqrt{11,31}} \right) = 2\Phi_0\left( \frac{0,04}{0,1063} \right) = 2\Phi_0\left( \frac{400}{1063} \right) \approx 2\Phi_0(0,38)$
    \begin{center}
        Семинар 13 декабря
    \end{center}
    Посчитаем $\displaystyle\int\limits_0^1 x^2\, dx$. Пусть $\xi_1,\dots,\ \xi_n \sim R(0,\ 1)$
    \[\frac{1}{n} \sum_{i = 1}^{n} \xi_i^2 \xrightarrow[n\to \infty]{\text{п. н.}} E\xi^2 = \int\limits_0^1 x^2\, dx = I\]
    $P\left( \left| \frac{1}{n} \sum_{i = 1}^{n} \xi_i^2 - I \right| < \delta \right) = 0,95$, где $\delta = 0,01$.\\
    По ЦПТ:
    \[\frac{\frac{1}{n}\sum_{i = 1}^{n} \xi_i^2 - I}{\sqrt{\mathcal{D}\big( \frac{1}{n} \sum_{i = 1}^{n} \xi_i^2 \big)}} \xrightarrow[]{d} U\sim N(0,\ 1)\]
    Тогда $P\left( \left| \frac{1}{n} \sum_{i = 1}^{n} \xi_i^2 - I \right| < \delta \right) = 2\Phi_0\left( \frac{\delta \sqrt{n}}{\sqrt{\mathcal{D}\xi^2_1}} \right)$\\
    Величина простая, поэтому можем в явном виде посчитать дисперсию:
    \[\mathcal{D}\xi^2_1 = E\xi_1^4 - (E\xi_1^2)^2 = \int\limits_0^1 x^4\, dx - \left( \int\limits_0^1 x^2 \, dx \right)^2 = \frac{1}{5} - \frac{1}{9} = \frac{4}{45}\]
    Получаем следующее:
    \[2\Phi_0 \left( \frac{\delta\sqrt{n} 3\sqrt{5}}{2} \right) = 2\Phi_0\left(\frac{0.03\sqrt{5}}{2}\sqrt{n}\right) = 0.95\Rightarrow n\approx 3400\]

\end{document}