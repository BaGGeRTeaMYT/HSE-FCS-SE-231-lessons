\documentclass[12pt, a4paper]{article}
\usepackage[utf8]{inputenc}
\usepackage[T2A]{fontenc}
\usepackage[russian]{babel}
\usepackage{amsfonts}
\usepackage{amsmath}
\usepackage{mathabx}
\usepackage{graphicx}
\usepackage{hyperref}
\usepackage{listings}
\usepackage{color}
\usepackage[margin=0.25in]{geometry}
\usepackage{pgfplots}
\pgfplotsset{width=10cm,compat=1.9}
\usepackage{nicefrac}
\graphicspath{../Images}
\usepackage{tikz}
\usepackage{cancel}

\usepackage{fontspec}
\setmainfont{Times New Roman}


\title{Математическая статистика.}
\author{Андрей Тищенко \href{https://t.me/AndrewTGk}{@AndrewTGk}}
\date{2024/2025}

\begin{document}
\maketitle
\begin{center}
    Лекция 10 января
\end{center}
\section*{Преамбула}
\textit{Статистика}. Мнения о появлении этого слова:
\begin{enumerate}
    \item Статистиками в Германии назывались люди, собирающие данные о населении и передающие их государству.
    \item В определённый день в Венеции народ выстраивался для выплаты налогов (строго фиксированных, в зависимости от рода действий). Государство собирало данные обо всём населении. Это происходило до появления статистиков в Германии, поэтому мы будем считать, что статистика пошла из Венеции.
\end{enumerate}
\textit{Задача статистики} --- по результатам наблюдений построить вероятностную модель наблюдаемой случайной величины.
\section*{Основные определения}
\subsection*{Определение}
\underline{Однородной выборкой объёма $n$} называется случайный вектор $X = (X_1,\dots,\ X_n)$, компоненты которого являются независимыми и одинаково распределёнными. Элементы вектора $X$ называются \underline{элементами}\\\underline{выборки}. 
\subsection*{Определение}
Если элементы выборки имеют распределение $F_{\xi}(x)$, то говорят, что выборка соответствует распределению $F_{\xi}(x)$ или порождена случайной величиной $\xi$ с распределением $F_{\xi}(x)$. 
\subsection*{Определение}
Детерминированный вектор $x = (x_1,\dots,\ x_n)$, компоненты которого $x_i$ являются реализациями соответствующих случайных величин $X_i\ (i = \overline{1,\ n})$, называется \underline{реализацией выборки}.
\subsection*{Уточнение}
Если $X$ --- однородная выборка объёма $n$, то его реализацией будет вектор $x$, каждый элемент $x_i$ которого является значением соответствующей ему случайной величины (элемента выборки) $X_i$. 
\subsection*{Определение}
\underline{Выборочным пространством} называется множество всех возможных реализаций выборки\\
$X = (X_1,\dots,\ X_n)$.\\
\subsection*{Пример}
У вектора $X = (X_1,\dots,\ X_{10})$ каждый элемент $X_i$ которой порождён случайной величиной $\xi\sim U(0,\ 1)$, выборочным пространством является $\mathbb{R}^{10}$ (так как $X_i$ может принять любое значение на $\mathbb{R}$)
\subsection*{Определение}
Пусть реализация выборки упорядочена по возрастанию:
\[x_{(1)} \leq x_{(2)} \leq \dots \leq x_{(n)}\]
Где $x_{(i)}$ --- $i$-ый по возрастанию элемент.\\
Обозначим $X_{(k)}$ случайную величину, реализация которой при каждой реализации $x$ выборки $X$ принимает значение $x_{(k)}$. Тогда последовательность $X_{(1)},\dots,\ X_{(n)}$ называется \underline{вариационным рядом выборки}.
\subsection*{Определение}
Случайная величина $X_{(k)}$ называется \underline{$k$-ой порядковой статистикой выборки}.
\subsection*{Определение}
Случайные величины $X_{(1)},\ X_{(n)}$ называются \underline{эстремальными порядковыми статистиками}.
\subsection*{Определение}
Порядковая статистика $X_{(\lceil n\cdot p \rceil)}$ называется \underline{выборочной квантилью уровня $p$}, где $p\in [0,\ 1]$
\subsection*{Определение}
Пусть каждый элемент выборки $X$ объёма n имеет распределение $F_{\xi}(x)$. \underline{Эмпирической функцией}\\ \underline{распределения} такой выборки называется
\[\hat{F}_n(x) = \frac{1}{n} \sum_{k = 1}^{n} I(X_k\leq x)\]
$I$ --- индикаторная функция. $I = \begin{cases}
    1,\ \text{если аргумент верен}\\
    0,\ \text{иначе}
\end{cases}$\\
Пусть $x_1,\dots,\ x_n$ --- реализация выборки $X_1,\dots,\ X_n$\\
\[\begin{tikzpicture}[scale=2, >=latex]
    \draw[->, >=latex] (0, -1) -- (0, 1) node(0)[anchor = east]{$\hat{F}_n(x)$};
    \draw (-2, -0.05) node(1)[anchor=north]{$x_1$} -- (-2, 0.05);
    \draw (-1, -0.05) node(2)[anchor=north]{$x_2$}-- (-1, 0.05);
    \draw (0.07, -0.05) node(3)[anchor=north east]{$x_3$}-- (0.07, -0.05);
    \draw (1, -0.05) node(4)[anchor=north]{$x_4$}-- (1, 0.05);
    \draw (2, -0.05) node(5)[anchor=north]{$x_5$}-- (2, 0.05);
    \draw[->] (-3, 0) -- (-2, 0);
    \draw[->] (-2, 0.2) -- (-1, 0.2);
    \draw[dashed] (-2, 0) -- (-2, 0.2);
    \draw[->] (-1, 0.45) -- (0, 0.45);
    \draw[dashed] (-1, 0) -- (-1, 0.45);
    \draw[->] (0, 0.6) -- (1, 0.6);
    \draw[dashed] (1, 0) -- (1, 0.85);
    \draw[->] (1, 0.85) -- (2, 0.85);
    \draw[dashed] (2, 0) -- (2, 1);
    \draw[->] (2, 1) -- (3, 1);
    \draw[->] (-3, 0) -- (3, 0) node(2)[anchor = south]{$x$};
\end{tikzpicture}\]
\begin{center}
    Свойства $\hat{F}_n(x)$
\end{center}
\begin{enumerate}
    \item $\displaystyle\forall x\in\mathbb{R}\quad E\hat{F}_n(x) = E\left(\frac{1}{n} \sum_{k = 1}^{n} I(X_k \leq x)\right) = \frac{1}{n} \sum_{k = 1}^{n} EI(X_k \leq x) = P(X_1 \leq x) = F_{\xi}(x)$
    \item По усиленному закону больших чисел (УЗБЧ)
    \[\forall x\in \mathbb{R}\quad \hat{F}_n(x) = \frac{1}{n} \sum_{k = 1}^{n} I(X_k \leq x) \xrightarrow[n\to\infty]{\text{п. н.}} EI(X_k \leq x) = F_{\xi}(x)\]
\end{enumerate}
\subsection*{Гистограмма}
Разбить $\mathbb{R}$ на $(m + 2)$ непересекающихся интервала. Рассматриваются $x_{(1)},\dots,\ x_{(n)}$
\[\begin{tikzpicture}[scale=2, >=latex]
    \draw[->] (-2, 0) -- (2, 0) node(2)[anchor = south]{$x$};
    \draw (-1.5, -0.1) node[anchor=north](4){$x_{(1)}$} -- (-1.5, 0.1) node(3){};
    \draw (1.5, -0.1) node[anchor=north](6){$x_{(n)}$} -- (1.5, 0.1) node(5){};
    \draw[<->] (-1.5, -0.08) -- node[anchor=north](10){$r$} (1.5, -0.08);
    \draw (0.75, -0.05) -- (0.75, 0.05) node(6){};
    \draw[<->] (-0.75, 0.08) -- node[anchor=south](11){$\Delta$} (0, 0.08);
    \draw (0, -0.05) -- (0, 0.05) node(8){};
    \draw (-0.75, -0.05) -- (-0.75, 0.05) node(9){};
    \;
    \draw (-1.5, 0) rectangle (-0.75, 0.5);
    \draw[->] (0.375, 0) -- node(0)[anchor = west]{$h_i$} (0.375, 0.75);
    \draw (-0.75, 0) rectangle (-0.0, 0.8);
    \draw (-0.0, 0) rectangle (0.75, 0.75);
    \draw (0.75, 0) rectangle (1.5, 0.2);
\end{tikzpicture}\]
Размах выборки $r = x_{(n)} - x_{(1)}$\\
$\Delta = \frac{r}{m}$ --- ширина интервала.\\
$h_k = \frac{\nu_k}{\Delta},\ k = \overline{1,\ k}$, где $\nu_k$ --- количество попаданий на интервал.
\end{document}
