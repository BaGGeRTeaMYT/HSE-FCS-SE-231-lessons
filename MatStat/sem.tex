\documentclass[12pt, a4paper]{article}
\usepackage[utf8]{inputenc}
\usepackage[T2A]{fontenc}
\usepackage[russian]{babel}
\usepackage{amsfonts}
\usepackage{amsmath}
\usepackage{mathabx}
\usepackage{graphicx}
\usepackage{hyperref}
\usepackage{listings}
\usepackage{color}
\usepackage[margin=0.25in]{geometry}
\usepackage{pgfplots}
\pgfplotsset{width=10cm,compat=1.9}
\usepackage{nicefrac}
\graphicspath{../Images}
\usepackage{tikz}
\usepackage{cancel}

\usepackage{fontspec}
\setmainfont{Calibri}

\title{Математическая статистика.}
\author{Андрей Тищенко \href{https://t.me/AndrewTGk}{@AndrewTGk}}
\date{2024/2025}

\newcommand{\dev}{\mathcal{D}}
\newcommand{\cov}{\operatorname{cov}}
\newcommand{\oo}{\infty}

\begin{document}
\maketitle
\begin{center}
    Семинар 10 января
\end{center}
\subsection*{Задача 1}
$x_1,\dots,\ x_n\sim F_{\xi}(x)$, найти функцию распределения для $X_{(n)},\ X_{(1)}$\\
$F_{X_{(n)}}(x) = P(X_{(n)} \leq x) = P(X_{(1)} \leq x,\dots,\ X_{(n)} \leq x) = P(X_1 \leq x,\dots,\ X_n \leq x) =\\
    =P(X_1\leq x) \dots P(X_n \leq x) ={(F_{\xi}(x))}^{n}$\\
$F_{X_{(1)}}(x) = P(X_{(1)} \leq x) = 1 - P(X_{(1)} > x) = 1 - P(X_{(1)} > x,\dots,\ X_{(n)} > x) =\\
    = 1 - P(X_1 > x,\dots,\ X_n > x) = 1 - P(X_1 > x)\dots P(X_n > x) = 1 - {\big(1 - F_{\xi}(x)\big)}^n$
\subsection*{Задача 2}
$x_1,\dots,\ x_n \sim R(0,\ 1)$. Найти $EX_{(n)},\ EX_{(1)}$.\\
$F_{X_{(n)}}(x) = \big( F_{\xi}(x) \big)^n$\\
$f_{X_{(n)}}(x) = \big( F_{X_{(n)}}(x) \big)' = n\big(F_{\xi}(x)\big)^{n - 1}\cdot f_{\xi}(x)$\\
$F_{\xi}(x) = \begin{cases}
        0,\ x < 0        \\
        x,\ x\in [0,\ 1] \\
        1,\ x > 1
    \end{cases}$\\
Подставим в предыдущее уравнение:
\[f_{X_{(n)}} = \begin{cases}
        0,\ x < 0                 \\
        nx^{n - 1},\ x\in [0,\ 1] \\
        0,\ x > 1
    \end{cases}\]
$EX_{(n)} = \displaystyle\int\limits_{-\infty}^{+\infty} xf_{X_{(n)}}(x)\, dx = \int\limits_0^1 xnx^{n - 1}\, dx = n\int\limits_0^1 x^n\, dx = \frac{n}{n + 1}$\\
Посчитаем для $X_{(1)}$:\\
$F_{X_{(1)}}(x) = 1 - (1 - F_{\xi}(x))^n\\
    f_{X_{(1)}}(x) = \big(F_{X_{(1)}}(x)\big)' = n(1 - F_{\xi}(x))^{n - 1} \big(F_{\xi}(x)\big)' = n(1 - F_{\xi}(x))^{n - 1} f_{\xi}(x) = \begin{cases}
        0,\ x < 0                          \\
        n(1 - x)^{n - 1},\ 0 \leq x \leq 1 \\
        0,\ x > 1
    \end{cases}$\\
$EX_{(1)} = \displaystyle \int\limits_0^1 xn(1 - x)^{n - 1}\, dx = n\int\limits_0^1 x(1 - x)^{n - 1}\, dx = \left< \begin{matrix}
        t = 1 - x \\
        x = 1 - t
    \end{matrix}\right> = -n\int\limits_1^0 (1 - t)t^{n - 1}\, dt = n\int\limits_0^1(1 - t)t^{n - 1}\, dt =\\
    = n\int\limits_0^1 t^{n - 1}\, dt - n\int\limits_0^1 t^{n}\, dt = 1 - \frac{n}{n + 1}$
\subsection*{Задача 3}
$\overline{x} = \frac{1}{n}\sum_{i = 1}^{n} x_i$\\
$E\overline{x} = E\left( \frac{1}{n} \sum_{i = 1}^{n} x_i \right) = \frac{1}{n} \sum_{i = 1}^{n} E(x_i) = Ex_i\\
    \mathcal{D}(\overline{x}) = \mathcal{D}\left( \frac{1}{n} \sum_{i = 1}^{n} x_i \right) = \frac{1}{n^2} \sum_{i = 1}^{n} \mathcal{D}x_i = \frac{\mathcal{D}x_i}{n}$\\
Посчитаем выборочную дисперсию:
\[S^2 = \frac{1}{n} \sum_{i = 1}^{n}(x_i - \overline{x})^2\]
$ES^2 = E\left( \frac{1}{n} \sum_{i = 1}^{n} (x_i - \overline{x})^2 \right) = \frac{1}{n}\sum_{i = 1}^{n} E(x_i - \overline{x})^2 = \mathcal{D}(x_1 - \overline{x}) = \mathcal{D}(x_1) + \mathcal{D}(\overline{x}) - 2\cov(x_1,\ \overline{x}) = \frac{(n + 1)\dev(x_1)}{n} -\\- 2\cov(x_1,\ \overline{x})\\
    \cov (x_1,\ \overline{x}) = \cov(x_1,\ \frac{1}{n} \sum_{i = 1}^{n} x_i) = \frac{1}{n}\cov (x_1,\ \sum_{i = 1}^{n} x_i) = \frac{1}{n} \cov (x_1,\ x_1) = \frac{\dev(x_1)}{n}$\\
Тогда
\[ES^2 = \frac{(n + 1)\dev(x_1)}{n} - \frac{2\dev(x_1)}{n} = \dev(x_1)\left(1 - \frac{1}{n}\right)\]
Несмещённая выборачная дисперсия (её математическое ожидание равняется дисперсии $x_1$):
\[\tilde{S}^2 = \frac{1}{n - 1} \sum_{i = 1}^{n} (x_i - \overline{x})^2\]

Посчитаем дисперсию $S^2$:
\[\dev\left( x_1 - \frac{1}{n} \sum_{i = 1}^{n} x_i \right) = \dev\left( \frac{(n - 1)x_1}{n} \right) + \dev\left( \frac{1}{n} \sum_{i = 2}^n x_i \right) = \frac{(n - 1)^2}{n^2}\dev(x_1) + \frac{n - 1}{n^2}\dev(x_1) =\]
\[= \dev(x_1)\left( \frac{(n - 1)(n - 1 + 1)}{n^2}\right) = \dev (x_1) \frac{n - 1}{n}\]
\begin{center}
    Семинар 17 января.
\end{center}
$T(x_1, x_2, \dots, x_n) = \sqrt{\frac{\pi}{2}}\frac{1}{n}\sum \limits_{i=1}^n |x_i - m|$, $x_i \sim N(m, \theta^2)$

$\displaystyle E T(x_1, x_2, \dots, x_n) = \sqrt{\frac{\pi}{2}}\frac{1}{n}\sum \limits_{i=1}^n E |x_i - m| = \sqrt{\frac{\pi}{2}}\ E |x_1 - m| = \sqrt{\frac{\pi}{2}}\ \int \limits_{-\infty}^{+\infty} {|x - m| \frac{1}{\sqrt{2\pi}\theta} e^{-\frac{(x-m)^2}{2\theta^2}} dx}$

Заменим $\frac{x-m}{\theta}$ на $y$

$\displaystyle\frac{\theta}{2}\int \limits_{-\infty}^{+\infty}|y|\cdot e^{\frac{-y^2}{2}} dy  = \theta \int \limits_{0}^{+\infty} y\cdot e^{\frac{-y^2}{2}} dy = \theta (1-0) = \theta$

\[\frac{1}{n}\sum \limits_{i=1}^{n} \sqrt{\frac{\pi}{2}} \ |x_i-m| \xrightarrow[n\to+\infty]{\text{п.\ н.}}  E\sqrt{\frac{\pi}{2}} \ |x_i-m|\]

\subsection*{Задача}
$X = (X_1,\dots,\ X_n),\ X_i\sim R(0,\ \theta)$\\
$\hat{\theta} = X_{(n)},$ доказать $ \lim\limits_{n\to\infty} EX_{(n)} = \theta\\
    F_{X_{(n)}} (x) = (F_{X_i}(x))^n = \left(\frac{x}{\theta}\right)^n\\
    f_{X_{(n)}} (x) = \frac{d F_{X_{(n)}}}{dx} = \frac{nx^{n - 1}}{\theta}\\
    \displaystyle E X_{(n)} = \int \limits _0^{\theta} \frac{n x^n}{\theta^n}\, dx = \left.\frac{n x^{n + 1}}{(n + 1)\theta^n} \right|_0^{\theta} = \frac{n}{n + 1}\theta \xrightarrow[n\to\infty]{} \theta$. То есть смещённая, но асимптотически несмещённая.\\
Докажем состоятельность, хотим:
\[\forall \varepsilon > 0\quad P(|\hat{\theta} - \theta| < \varepsilon) \xrightarrow[n\to\infty]{} 1\]
\[P(-\varepsilon < X_{(n)} - \theta < \varepsilon) = F_{X_(n)} (\varepsilon + \theta) - F_{X_{(n)}}(\theta - \varepsilon) = 1 - \left( \frac{\theta - \varepsilon}{\theta} \right)^n \xrightarrow[n\to\infty]{} 1\]
\subsection*{Задача}
$I_n(\theta) = E\left( \frac{\delta \ln f(x,\ \theta)}{\delta \theta} \right)^2,\ I_n(\theta) = nI_1(\theta),\ x_1,\dots,\ x_n\sim N(\theta,\ \sigma^2)$.\\
$f(x,\ \theta) = \frac{1}{\sqrt{2\pi}\sigma} e^{-\frac{(x - \theta)^2}{2\sigma^2}}$\\
$\ln f(x,\ \theta) = \ln \left( \frac{1}{\sqrt{2\pi}\sigma }e^{-\frac{(x - \theta)^2}{2\sigma^2}} \right) = -\frac{(x - \theta)^2}{2\sigma^2} + \ln \frac{1}{\sqrt{2\pi}\sigma}\\
    \frac{\delta\ln f(x,\ \theta)}{\delta\theta} = -\frac{2(x - \theta)}{2\sigma^2}\cdot (-1) = \frac{x - \theta}{\sigma^2}\\
    E\left( \frac{x - \theta}{\sigma^2} \right)^2 = \frac{1}{\sigma^4} E(x - \theta)^2 = \frac{1}{\sigma^4} \sigma^2 = \frac{1}{\sigma^2} = I_1(\theta)\\
    \dev \hat{\theta} \geq \frac{1}{nI_1(\theta)} = \frac{\sigma^2}{n} =\mathcal{D}\overline{x}$
\begin{center}
    Семинар 24 января
\end{center}
\subsection*{Задача 4 ДЗ}
$\hat{K}_{xy} = \frac{1}{n} \sum_{i = 1}^{n} (x_i - \overline{X})(y_i - \overline{Y}) = \frac{1}{n} \sum_{i = 1}^{n} (x_i - \overline{X} + Ex_1 - Ex_1)(y_i - \overline{Y} + Ey_1 - Ey_1)$\\
$E\hat{K}_{xy} = E\frac{1}{n} \sum_{i = 1}^{n} \left( (x_i - Ex_1) - (\overline{X} - Ex_1) \right) \left( (y_1 - Ey_1) - (\overline{Y} - Ey_1) \right) =\\
    = E\left( (x_i - Ex_1) - (\overline{X} - Ex_1) \right) \cdot \left( (y_1 - Ey_1) - (\overline{Y} - Ey_1) \right) = E\left( (x_1 - Ex_1)(y_1 - Ey_1) + (x_1 - Ex_1)(\overline{Y} - Ey_1) +\\
    +(\overline{X} - Ex_1)(y_1 - Ey_1) + (\overline{X} - Ex_1)(\overline{Y} - Ey_1) \right) = \\
    =\cov(x,\ y) - \frac{1}{n}\cov(x,\ y) - \frac{1}{n}\cov(x,\ y) + \frac{1}{n}\cov(x,\ y) $
\subsection*{Задача 5 ДЗ}
Решал у доски, всем gl.
\subsection*{Задача 1}
$X_1,\dots,\ X_n \sim \Pi(\theta)$. Проверить, что оценка $\hat{\theta} = \overline{X}$ является R-эффективной.\\
$E\hat{\theta} = E\frac{1}{n} \sum_{i = 1}^{n} x_i = Ex_1 = \theta\\
    \dev \frac{1}{n} \sum_{i = 1}^{n} x_i = \frac{1}{n}\theta\\
    P(\xi = x_1) = \frac{e^{-\theta}\theta^{x_1}}{x_1!}$. Логарифмируем:
\[\ln \frac{e^{-\theta}\theta^{x_1}}{x_1!} = {-\theta + x_1 \ln \theta - \ln x_1!}\]
Возьмём частную производную:
\[\frac{\delta(-\theta + x_1 \ln \theta - \ln x_1!)}{\delta\theta} = -1 + \frac{x_1}{\theta}\]
Возьмём матожидание квадрата этой величины:
\[E(-1 + \frac{x_1}{\theta})^2 = \frac{1}{\theta^2} E(x_1 - \theta)^2 = \frac{\dev x_1}{\theta^2} = \frac{1}{\theta}\Rightarrow I_n(\theta) = \frac{n}{\theta}\]
Попробуем самостоятельно подогнать оценку:\\
$U(x,\ \theta) = \sum_{i = 1}^{n} -1 + \frac{x_1}{\theta} = \frac{1}{\theta} \sum_{i = 1}^{n} (x_i - \theta) = \frac{1}{\theta}\left(-n\theta + \sum_{i = 1}^{n} \frac{x_i}{n}\right) = \frac{n}{\theta}\left( \sum_{i = 1}^{n} \left( \frac{x_i}{n}\right) - \theta \right)\\
    \hat{\theta} - \theta = a(\theta)U(x,\ \theta)\Rightarrow a(\theta) = \frac{\theta}{n},\ \hat{\theta} = \sum_{i = 1}^{n} \frac{x_i}{n}$
\begin{center}
    ДЗ
\end{center}
\subsubsection*{Задача 1}
$X_1,\dots,\ X_n\sim N(\theta, \sigma^2)\Rightarrow \forall i = \overline{1,\ n}\quad f(x_i,\ \theta) = \frac{1}{\sqrt{2\pi}\sigma} e^{-\frac{(x - \theta)^2}{2\sigma^2}}$\\
\[\ln f(x_i,\ \theta) = \ln \frac{1}{\sqrt{2\pi}\sigma} - \frac{(x - \theta)^2}{2\sigma^2} = \ln \frac{1}{\sqrt{2\pi}\theta} - \frac{x^2}{2\sigma^2} + \frac{\theta x}{\sigma^2} - \frac{\theta^2}{2\sigma^2}\Rightarrow \frac{\delta}{\delta \theta} f(x_i,\ \theta) = \frac{x}{\sigma} - \frac{\theta}{\sigma^2}\]
\[U(x,\ \theta) = \sum\limits_{i = 1}^{n}\left( \frac{x_i}{\sigma} - \frac{\theta}{\sigma^2} \right)\]
По критерию эффективности хотим:
\[\hat{\theta} - \theta = \alpha(x)U(x,\ \theta)\]
Преобразуем: $U(x,\ \theta) = \left(\sum\limits_{i = 1}^{n} \frac{x_i}{\sigma}\right) - \frac{n\theta}{\sigma^2}\Rightarrow \underset{\alpha(\sigma)}{\underbrace{\frac{\sigma^2}{n}}} U(x,\ \theta) = \underset{\hat{\theta}}{\underbrace{\left(\frac{1}{n}\sum\limits_{i = 1}^{n} \sigma x_i \right)}} - \theta$
\subsubsection*{Задача 2}
$X_1,\dots,\ X_n\sim N(m,\ \theta)\Rightarrow f(x_i,\ \theta) = \frac{1}{\sqrt{2\pi\theta}} e^{-\frac{(x - m)^2}{2\theta}}$\\
\[\ln f(x,\ \theta) = \ln \frac{1}{\sqrt{2\pi}} - \frac{1}{2}\ln \theta - \frac{(x - m)^2}{2\theta}\Rightarrow \frac{\delta}{\delta \theta} f(x,\ \theta) = -\frac{1}{2\theta} + \frac{(x - m)^2}{2\theta^2}\]
Применим критерий эффективности:
\[U(x,\ \theta) = \sum\limits_{i = 1}^{n} \left( \frac{(x - m)^2}{2\theta^2} - \frac{1}{2\theta} \right) = \sum\limits_{i = 1}^{n} \left( \frac{(x - m)^2 - \theta}{2\theta^2}\right) = \frac{1}{2\theta^2} \sum\limits_{i = 1}^{n} \big((x - m)^2 - \theta\big) = \]
\[= \frac{1}{2\theta^2} \left( \sum\limits_{i = 1}^{n} \big((x - m)^2\big) - n\theta\right) = \frac{n}{2\theta^2} \left( \frac{1}{n} \sum\limits_{i = 1}^{n} \big((x - m)^2\big) - \theta\right)\Rightarrow \underset{\alpha(\theta)}{\underbrace{\frac{2\theta^2}{n}}} U(x,\ \theta) = \underset{\hat{\theta}}{\underbrace{\left(\frac{1}{n} \sum\limits_{i = 1}^{n} (x - m)^2\right)}} - \theta\]
\subsubsection*{Задача 3}
$X_1,\dots,\ X_n \sim G(\theta)\Rightarrow Ex = \frac{1}{\theta}$. Проверить оценку $\hat{\theta} = \frac{1}{\overline{X}}$ на несмещённость.\\
Хотим $E\hat{\theta} = \theta$. Попробуем по определению:
\[E\hat{\theta} = E \frac{n}{\sum\limits_{i = 1}^{n} x_i} = n E\frac{1}{\sum\limits_{i = 1}^{n} x_i}?\]
Для $k = 1$
Попробуем решить через функцию правдоподобия:
\[L(x_1,\dots,\ x_n,\ \theta) = \prod_{i = 1}^{n} P(\xi = x_i,\ \theta) = \prod_{i = 1}^{n} (1 - \theta)^{x_i - 1}\theta \approx f(x,\ \theta)\]
\[\ln f(x_i,\ \theta) = \ln \big( (1 - \theta)^{x_i - 1} \theta \big) = (x_i - 1)\ln (1 - \theta) + \ln \theta\]
\[\frac{\delta}{\delta \theta} \ln f(x,\ \theta) = \frac{1}{\theta} - \frac{x_i - 1}{1 - \theta} = \frac{1 - \theta - \theta x_i + \theta}{\theta - \theta^2} = \frac{1 - \theta x_i}{\theta - \theta^2}\]
Применим критерий эффективности:
\[U(x,\ \theta) = \sum_{i = 1}^{n} \frac{1 - \theta x_i}{\theta - \theta^2} = \frac{1}{\theta - \theta^2} \left( n - \theta \sum_{i = 1}^{n} x_i \right) = \frac{n}{\theta - \theta^2} \left( 1 - \frac{\theta}{n} \sum_{i = 1}^{n} \right) = \frac{n\overline{X}}{\theta - \theta^2} \left( \frac{1}{\overline{X}} - \theta \right)\]
Значит $\frac{1}{\overline{X}}$ является R-эффективной, то есть несмещённой.
\subsubsection*{Задача 4}
$X_1,\dots,\ X_n \sim Bi(k,\ \theta)$. Показать, что $\hat{\theta}  = \frac{\overline{X}}{k}$ R-эффективная.\\
Посчитаем функцию правдоподобия:
\[L(x_1,\dots,\ x_n,\ \theta) = \prod_{i = 1}^{n} P(\xi = x_i,\ \theta) = \prod_{i = 1}^{n} C_n^k\theta^{x_i}\cdot (1 - \theta)^{k - x_i} \approx f(x,\ \theta)\]
\[\ln f(x_i,\ \theta) = \ln\frac{n!}{k!(n - k)!} + x_i \ln \theta + (k - x_i)\ln (1 - \theta)\]
\[\frac{\delta}{\delta \theta} \ln f(x_i,\ \theta) = \frac{x_i}{\theta} + \frac{x_i - k}{1 - \theta} = \frac{x_i - \theta x_i + \theta x_i - \theta k}{\theta - \theta^2} = \frac{x_i - \theta k}{\theta - \theta^2}\]
\[I_1(\theta) = E \left( \frac{x_i - \theta k}{\theta - \theta^2} \right)^2 = \int\limits_{-\infty}^{+\infty} \frac{(x - \theta k)^2}{(\theta - \theta^2)^2} C_n^k \theta^{x} (1 - \theta)^{k - x}\, dx\]
\subsubsection*{Задача 5}
\[U(x,\ \theta) = \sum_{i = 1}^{n} \frac{x_i - \theta k}{\theta - \theta^2} = \frac{1}{\theta - \theta^2}\left(-n\theta k + \sum_{i = 1}^{n} x_i\right) = \frac{nk}{\theta - \theta^2}\left( \frac{1}{nk} \sum_{i = 1}^{n} (x_i) - \theta \right) = \frac{nk}{\theta - \theta^2} \left( \frac{\overline{X}}{k} - \theta \right)\]
Получается, что $\frac{\overline{X}}{k}$ является R-эффективной
\begin{center}
    \bf Семинар 31 января
\end{center}
\subsection*{Задача 1}
$X_1,\dots,\ X_n \sim f(x,\ \theta)$\\
$f(x,\ \theta) = \begin{cases}
        \frac{2}{\theta} x e^{-\frac{x^2}{\theta}},\ x \geq 0 \\
        0,\ x < 0
    \end{cases}$\\
Решал у доски.
\subsection*{Задача 2}
$X_1,\dots,\ X_n \sim R(\theta_1,\ \theta_2)$, найти оценку максимального правдоподобия.\\
$f(x,\ \theta_1,\ \theta_2) = \begin{cases}
        \frac{1}{\theta_2 - \theta_1},\ x\in (\theta_1,\ \theta_2) \\
        0,\ \text{иначе}
    \end{cases}\\
    L(x,\ \theta_1,\ \theta_2) = \prod\limits_{i = 1}^{n} f(x_i,\ \theta) = \begin{cases}
        \left( \frac{1}{\theta_2 - \theta_1} \right)^n,\ x_i \in (\theta_1,\ \theta_2) \\
        0,\ \text{иначе}
    \end{cases}$\\
Тогда $\hat{\theta}_1 = X_{(1)},\ \hat{\theta}_2 = X_{(n)}$.\\
Попробуем по методу моментов:
\[\begin{cases}
        \hat{\mu}_1 = \mu_1 \\
        \hat{\mu}_2 = \mu_2 = \frac{(\theta_2 - \theta_1)^2}{12} + (\mu_1)^2
    \end{cases}\]
Распишем эту систему:
\[\begin{cases}
        \frac{1}{n} \sum_{i = 1}^{n} x_i = \frac{\theta_1 + \theta_2}{2} \\
        \frac{1}{n} \sum_{i = 1}^{n} x_i^2 = \frac{(\theta_2 - \theta_1)^2}{12} + \left( \frac{\theta_1 + \theta_2}{2} \right)^2
    \end{cases}\Rightarrow \begin{cases}
        \frac{1}{n} \sum_{i = 1}^{N} X_i^2 = \frac{\theta_2^2 + \theta_1^2 - 2\theta_1 \theta_2}{12} + \frac{\theta_1^2 + \theta_2^2 + 2\theta_1 \theta_2}{4} = \frac{1}{3}(\theta_1^2 + \theta_2^2 + \theta_1 \theta_2)
    \end{cases}\]
\[\begin{cases}
        2 \hat{\mu}_1 = \theta_1 + \theta_2 \\
        3 \hat{\mu}_2 = \theta_1^2 + \theta^2_2 + \theta_1 \theta_2
    \end{cases}\]
Если решать эту систему до конца, можно получить
\[\begin{cases}
        \hat{\theta_1} = \overline{X} - \sqrt{3}S \\
        \hat{\theta_2} = \overline{X} + \sqrt{3}S
    \end{cases}\]
\subsection*{Задача 3}
$X_1,\dots,\ X_n \sim G(\theta)$. Найдём оценку по методу моментов и по методу максимального правдоподобия:\\
Сначала по методу моментов:
\[\hat{\mu}_1 = \mu_1 = \frac{1}{\theta}\Rightarrow \hat{\theta} = \frac{1}{\overline{X}}\]
Теперь по методу максимального правдоподобия:
\[L(x,\ \theta) = \prod_{i = 1}^n P(\xi = x_i,\ \theta) = \prod_{i = 1}^n \theta(1 - \theta)^{x_i - 1} = \theta^n (1 - \theta)^{\sum_{i = 1}^{n} (x_i) - n}\]
\[\ln L(x,\ \theta) = n \ln \theta + \left(\sum_{i = 1}^{n} (x_i) - n\right) \ln(1 - \theta)\]
\[\frac{\delta }{ \delta \theta} L(x,\ \hat\theta) = \frac{n}{\hat\theta} - \frac{\sum_{i}^{n} (x_i) - n}{1 - \hat\theta} = 0\Rightarrow \frac{n - n\hat\theta - \hat\theta\sum_{i = 1}^{n} + n\hat\theta}{\hat\theta - \hat\theta^2} = 0 \Rightarrow\]
\[\Rightarrow n - \hat\theta \sum_{i = 1}^{n} = 0\Rightarrow \hat\theta = \frac{n}{\sum_{i  =1}^{n}x_i} = \frac{1}{\overline{X}}\]
\subsection*{Задача 4}
$X_1 \sim Bi(12,\ p),\ X_2 \sim Bi(12,\ p),\ X_3 \sim Bi(15,\ p)$. По методу максимального правдоподобия построим оценку $p$:
\[L(x_1,\ x_2,\ x_3,\ p) = \prod_{i = 1}^{n} P(X_i = x_i) = P(X_1 = 5) P(X_2 = 4) P(X_3 = 4) =\]
\[ = C^5_{12} p^5 (1 - p)^7 \cdot C^4_{12} p^4 (1 - p)^8 C^4_{15} p^4 (1 - p)^{11} = C^5_{12}\cdot C^4_{12} \cdot C^4_{15} \cdot p^{13}\cdot (1 - p)^{26}\]
\[\ln L(x_1,\ x_2,\ x_3,\ p) = \ln (C^5_{12}\cdot C^4_{12} \cdot C^4_{15}) + 13 \ln p + 26 \ln (1 - p)\]
\[\frac{\delta}{\delta p} L(x_1,\ x_2,\ x_3,\ p) = \frac{13}{p} - \frac{26}{1 - p}\Rightarrow \frac{13}{\hat p} - \frac{26}{1 - \hat p} = 0\Rightarrow \hat p = \frac{1}{3}\]

\begin{center}
    \bf ДЗ к семинару 7 января
\end{center}

\subsection*{Задача из учебника №14 стр. 203}
Пусть $Z_n = (X_1,\dots,\ X_n)$ --- выборка, соответствующая биномиальному распределению $Bi(10,\ \theta)$. Оценить неизвестный параметр $\theta$ методом максимального правдоподобия.\\

Построим функцию правдоподобия для вектора $(X_1,\dots,\ X_n)$:
\[L(x_1,\dots,\ x_n,\ \theta) = \prod_{i = 1}^{n} P(X_i = x_i) = \prod_{i = 1}^{n} C_n^{x_i} \cdot \theta^{x_i}(1 - \theta)^{n - x_i}\]
Логарифмируем и дифференцируем по $\theta$ полученное произведение:
\begin{equation*}
    \begin{aligned}
        \frac{\delta}{\delta \theta}\ln L(x_1,\dots,\ x_n,\ \theta) & = \frac{\delta}{\delta \theta}\ln\left(\prod_{i = 1}^{n} C_n^{x_i} \cdot \theta^{x_i}(1 - \theta)^{n - x_i} \right)  = \frac{\delta}{\delta \theta}\sum_{i = 1}^{n} \Big(\ln \big( C_n^{x_i} \cdot \theta^{x_i}(1 - \theta)^{n - x_i} \big)\Big) = \\
                                                                    & =\frac{\delta}{\delta \theta}\sum_{i = 1}^{n} \left(\ln C_{n}^{x_i}\right) + \frac{\delta}{\delta \theta}\sum_{i = 1}^{n} (x_i\ln \theta) + \frac{\delta}{\delta \theta}\sum_{i = 1}^{n} \big( (n - x_i) \ln(1 - \theta)\big) =                    \\
                                                                    & =0 + \frac{\delta}{\delta \theta}\ln\theta \sum_{i = 1}^{n} x_i + \frac{\delta}{\delta \theta}\ln(1 - \theta) \sum_{i = 1}^{n} (n - x_i) = \frac{1}{\theta} \sum_{i =1}^{n} x_i - \frac{1}{1 - \theta} \sum_{i = 1}^{n} (n - x_i) =                \\
                                                                    & =\frac{1}{\theta}\sum_{i = 1}^{n} x_i - \frac{n^2}{1 - \theta} + \frac{1}{1 - \theta}\sum_{i = 1}^{n} x_i = \frac{(1 - \theta)n\overline{x} - \theta n^2 + \theta n\overline{x}}{\theta - \theta^2} =                                              \\
                                                                    & =\frac{n\overline{x} - \theta n\overline{x} - \theta n^2 + \theta n\overline{x}}{\theta - \theta^2} = \frac{n\overline{x} - \theta n^2}{\theta - \theta^2} = n\frac{\overline{x} - \theta n}{\theta - \theta^2}
    \end{aligned}
\end{equation*}
Полученную производную стоит приравнять к 0 для поиска точки экстремума. Стоит заметить, что случаи $\theta = 0$ или $\theta = 1$ интереса не представляют и количество испытаний ненулевое, иначе оценивание параметра бессмысленно, поэтому достаточно приравнять к нулю только числитель:
\begin{equation*}
    n\frac{\overline{x} - \hat\theta n}{\hat\theta - \hat\theta^2} =  0 \Rightarrow \overline{x} - \hat\theta n = 0\Rightarrow \hat\theta n = \overline{x}\Rightarrow \hat\theta = \frac{\overline{x}}{n}
\end{equation*}

\textbf{Ответ:} ОМП для $\theta$ является $\frac{\overline{x}}{n}$.
\subsection*{Задача 2}
Выборка $X_1,\dots,\ X_n$ порождена случайной величиной $\xi$ с плотностью распределения
\[f_{\xi}(x,\ \theta) = \frac{1}{2} \exp(-|x - \theta|)\]
Построим оценки параметра $\theta$ по методу максимального правдоподобия и по методу моментов.
\subsubsection*{Метод максимального правдоподобия}
Построим функцию правдоподобия:
\begin{equation*}
    \begin{aligned}
        L(x_1,\dots,\ x_n,\ \theta) & = \prod_{i = 1}^{n} f_{\xi}(x_i,\ \theta) = \prod_{i = 1}^{n} \frac{1}{2} \exp\left(-|x_i - \theta|\right) = \frac{1}{2^n} \exp\left(-\sum_{i = 1}^{n}|x_i - \theta|\right)
    \end{aligned}
\end{equation*}
Логарифмируем и продифференцируем по $\theta$:
\begin{equation*}
    \begin{aligned}
        \frac{\delta}{\delta \theta} \ln L(x_1,\dots,\ x_n,\ \theta) & = \frac{\delta}{\delta \theta} \ln \frac{1}{2^n} \exp\left(-\sum_{i = 1}^{n}|x_i - \theta|\right) = \frac{\delta}{\delta \theta} \ln \frac{1}{2^n} - \frac{\delta}{\delta \theta} \sum_{i = 1}^{n}|x_i - \theta| = \\
                                                                     & = -\frac{\delta}{\delta \theta} \sum_{i = 1}^{n}|x_i -\theta| = -\sum_{i = 1}^{n} \frac{\delta}{\delta \theta} |x_i - \theta| = -\sum_{i = 1}^{n} g(x_i,\ \theta)
    \end{aligned}
\end{equation*}
Где $g(x,\ \theta) = \begin{cases}
        -1, & x > \theta \\
        0,  & x = \theta \\
        1,  & x < \theta
    \end{cases}$, (производная модуля).\\
Приравняем производную к нулю:
\begin{equation*}
    -\sum_{i = 1}^{n} g(x_i,\ \theta) = 0\Rightarrow \sum_{i = 1}^{n} g(x_i,\ \theta) = 0
\end{equation*}
Пусть $\begin{cases}

        G_{\theta} = \{x\ |\ x\in (x_1,\dots,\ x_n) \wedge x > \theta\} \\ E_{\theta} =
        \{x\ |\ x\in (x_1,\dots,\ x_n) \wedge x = \theta\}              \\ L_{\theta} = \{x\ |\ x\in
        (x_1,\dots,\ x_n) \wedge x < \theta\}
    \end{cases}$, тогда
\[\begin{cases}
        \forall x\in G_{\theta}\quad g(x,\ \theta) = -1 \\
        \forall x\in E_{\theta}\quad g(x,\ \theta) = 0  \\
        \forall x\in L_{\theta}\quad g(x,\ \theta) = 1  \\
    \end{cases}\Rightarrow \sum_{i = 1}^{n} g(x_i,\ \theta) = (-1)\cdot|G_{\theta}| + 0\cdot|E_{\theta}| + 1\cdot |L_{\theta}|\]
Преобразуем:
\begin{equation*}
    -|G_{\theta}| + 0|E_{\theta}| + |L_{\theta}| = 0\Rightarrow |G_{\theta}| = |L_{\theta}|
\end{equation*}
То есть количество элементов больше параметра $\theta$ в выборке должно совпадать с количеством элементов меньше параметра $\theta$.\\

Получается $\hat \theta = \begin{cases}
        x_{(\lfloor n/2 \rfloor)},           & n\underset{2}{\equiv} 1 \\
        \frac{x_{(n/2)} + x_{(n/2 + 1)}}{2}, & n\underset{2}{\equiv} 0
    \end{cases}$
\subsubsection*{Метод моментов}
Напишем систему уравнений для моментов (поскольку неизвестный параметр $\theta$ единственный, должно хватить одного уравнения):
\begin{equation*}
    \hat\mu_1 = \mu_1(\theta)\Rightarrow \frac{1}{n} \sum_{i = 1}^{n} x_i = E\xi
\end{equation*}
Посчитаем математическое ожидание случайной величины $\xi$:
\begin{equation*}
    \begin{aligned}
        E\xi & = \int\limits_{-\infty}^{+\infty} x f_{\xi}(x,\ \theta)\, dx = \int\limits_{-\infty}^{+\infty} x \frac{1}{2}\exp\left( -|x - \theta|\right)\, dx = \left< \begin{aligned}
                                                                                                                                                                             a  & = x - \theta \\
                                                                                                                                                                             da & = dx
                                                                                                                                                                         \end{aligned}  \right> =                                                   \\
             & = \frac{1}{2}\int\limits_{-\infty}^{+\infty} (a + \theta)\exp(-|a|)\, da = \frac{1}{2} \underset{=0}{\underbrace{\int\limits_{-\infty}^{+\infty}a\exp(-|a|)\, da}} + \frac{\theta}{2}\int\limits_{-\infty}^{+\infty} e^{-|a|}\, da = \\
             & = \theta \int\limits_{0}^{+\infty} e^{-a}\, da = -\theta \int\limits_0^{+\infty} e^{-a}\, d(-a) = -\theta e^{-a}\Big|_0^{+\infty} = -\theta (0 - 1) = \theta
    \end{aligned}
\end{equation*}
Итак, получаем уравнение:
\begin{equation*}
    \overline{X} = \theta
\end{equation*}
Его даже решать не надо, получаем $\hat\theta = \overline{X}$.\\
\subsubsection*{Ответ}
По методу максимального правдоподобия: $\hat \theta = \begin{cases}
        x_{(\lfloor n/2 \rfloor)},           & n\underset{2}{\equiv} 1 \\
        \frac{x_{(n/2)} + x_{(n/2 + 1)}}{2}, & n\underset{2}{\equiv} 0
    \end{cases}$\\
По методу моментов: $\hat \theta = \overline{X}$
\subsection*{Задача 3}
Выборка $X_1,\dots,\ X_n \sim \Pi(\theta)\Rightarrow \forall i\quad \begin{cases}
        P(X_i = k) = \frac{e^{-\theta} \theta^k}{k!} \\
        EX_i = \theta
    \end{cases}$. Построим оценки ММ и МП для $\theta$
\subsubsection*{Метод моментов}
Снова неизвестный параметр только один, поэтому достаточно одного уравнения:
\begin{equation*}
    \frac{1}{n} \sum_{i = 1}^{n} x_i = \theta\Rightarrow \hat \theta = \overline{X}
\end{equation*}
\subsubsection*{Метод максимального правдоподобия}
Функция правдоподобия:
\begin{equation*}
    L(x_1,\dots,\ x_n,\ \theta) = \prod_{i = 1}^{n} \frac{e^{-\theta} \theta^{x_i}}{x_i!} = e^{-n\theta} \prod_{i = 1}^{n} \frac{\theta^{x_i}}{x_i!}
\end{equation*}
Логарифм:
\begin{equation*}
    \ln L(x_1,\dots,\ x_n,\ \theta) = -n\theta + \sum_{i = 1}^{n} (x_i\ln \theta - \ln x_i!) = -n\theta + n\overline{X}\ln\theta  - \sum_{i = 1}^{n} \ln x_i!
\end{equation*}
Производная по $\theta$
\begin{equation*}
    \frac{\delta}{\delta \theta} L(x_1,\dots,\ x_n,\ \theta) = -n + \frac{n\overline{X}}{\theta}
\end{equation*}
Приравняем к нулю:
\begin{equation*}
    -n + \frac{n\overline{X}}{\hat\theta} = 0 \Rightarrow \hat\theta = \overline{X}
\end{equation*}

\textbf{Ответ:} оценки МП и ММ равны $\overline{X}$
\subsection*{Задача 4}
Ученик и тренер стреляют в цель до первого попадания (геометрическое распределение). Известно, что тренер попадает в цель с вероятностью в два раза большей, чем ученик. В ходе соревнования тренер попал в цель при втором выстреле, а ученик --- при пятом. Построить ОМП для вероятности попадания учеником в цель при единичном выстреле.\\

Пусть $\xi$ --- количество выстрелов, необходимых тренеру для попадания. Знаем $\xi \sim G(\theta_1)$.\\
Пусть $\eta$ --- количество выстрелов, необходимых ученику для попадания. Знаем $\eta \sim G(\theta_2)$.\\
Также знаем, что $\theta_1 = 2\theta_2$.\\
(Неоднородная выборка???) $(\xi,\ \eta)$ получила реализацию $(x_1,\ x_2) = (2,\ 5)$. Нужно построить оценку максимального правдоподобия для параметра $\theta_2$.\\

Функция правдоподобия:
\begin{equation*}
    L(x_1,\ x_2,\ \theta_1,\ \theta_2) = P(\xi = 2)\cdot P(\eta = 5) = (1 - \theta_1)\cdot \theta_1 \cdot (1 - \theta_2)^4 \cdot \theta_2 = 2(1 - 2\theta_2)\cdot(1 - \theta_2)^4\cdot\theta_2^2
\end{equation*}
Логарифмируем:
\begin{equation*}
    \ln L(x_1,\ x_2,\ \theta_1,\ \theta_2) = \ln 2 + \ln(1 - 2\theta_2) + 4\ln(1 - \theta_2) + 2\ln \theta_2
\end{equation*}
Продифференцируем:
\begin{equation*}
    \begin{aligned}
        \frac{\delta}{\delta \theta_2} \ln L(x_1,\ x_2,\ \theta_1,\ \theta_2) & = -\frac{2}{1 - 2\theta_2} - \frac{4}{1 - \theta_2} + \frac{2}{\theta_2} = \frac{2(1 - 2\theta_2)(1 - \theta_2) - 4\theta_2 \cdot(1 - 2\theta_2)  -2\theta_2\cdot(1 - \theta_2)}{(\theta_2 - 2\theta^2_2)(1 - \theta_2)} =                                            \\
                                                                              & = \frac{2(1 - 3\theta_2 + 2\theta_2^2) - 4(\theta_2 - 2\theta_2^2) - 2(\theta_2 - \theta_2^2)}{\theta_2 - 3\theta_2^2 + 2\theta_2^3} = \frac{2 - 6\theta_2 + 4\theta_2^2 - 4\theta_2 + 8\theta_2^2 - 2\theta_2 + 2\theta_2^2}{\theta_2 - 3\theta_2^2 + 2\theta_2^3} = \\
                                                                              & = \frac{14\theta_2^2 - 12\theta_2 + 2}{2\theta_2^3 - 3\theta_2^2 + \theta_2}
    \end{aligned}
\end{equation*}
Приравняем к нулю:
\begin{equation*}
    \begin{aligned}
        \frac{14\hat\theta_2^2 - 12\hat\theta_2 + 2}{2\hat\theta_2^3 - 3\hat\theta_2^2 + \hat\theta_2} = 0 \Rightarrow {14\hat\theta_2^2 - 12\hat\theta_2 + 2} = 0 \Rightarrow 7\hat\theta_2^2 - 6\hat\theta_2 + 1 = 0 \Rightarrow \mathcal{D}' = 9 - 7 = 2\Rightarrow \left[\begin{gathered}
                                                                                                                                                                                                                                                                                 \begin{aligned}
                \hat\theta_2 & = \frac{3 + \sqrt{2}}{7} \Rightarrow \theta_1 > 1 \\
                \hat\theta_2 & = \frac{3 - \sqrt{2}}{7}
            \end{aligned}
                                                                                                                                                                                                                                                                             \end{gathered}\right.
    \end{aligned}
\end{equation*}
\textbf{Ответ:} $\hat \theta_2 = \frac{3 - \sqrt{2}}{7} \approx 0.22654$
\subsection*{Задача 5}
Выборка $X_1,\dots,\ X_n$ порождена случаной величиной $X$ с плотностью распределения:
\[f(x,\ \theta) = \begin{cases}
        \frac{1}{\theta}x^{\frac{1 - \theta}{\theta}}, & x\in (0,\ 1)    \\
        0,\                                            & x\notin (0,\ 1)
    \end{cases}\]

Построим оценку максимального правдоподобия для параметра $\theta$ и исследуем его на несмещённость.\\
Построим функцию правдоподобия:
\begin{equation*}
    L(x_1,\dots,\ x_n,\ \theta) = \prod_{i = 1}^{n} f(x_i,\ \theta) = \theta^{-n} \prod_{i = 1}^{n} x_i^{\frac{1 - \theta}{\theta}}
\end{equation*}
Логарифмируем функцию правдоподобия:
\begin{equation*}
    \ln L(x_1,\dots,\ x_n,\ \theta) = \sum_{i = 1}^{n} \left(\frac{1 - \theta}{\theta} \ln x_i\right) -n\ln \theta = \frac{1}{\theta} \sum_{i = 1}^{n} (\ln x_i) - \sum_{i = 1}^{n} (\ln x_i) - n \ln \theta
\end{equation*}
Продифференцируем логарифм по $\theta$:
\begin{equation*}
    \frac{\delta}{\delta \theta} L(x_1,\dots,\ x_n,\ \theta) = - \frac{n}{\theta} - \frac{1}{\theta_2} \sum_{i = 1}^{n} \ln x_i = \dfrac{-n\theta - \sum\limits_{i = 1}^{n} \ln x_i}{\theta^2}
\end{equation*}
Приравняем к нулю:
\begin{equation*}
    -n\hat\theta - \sum_{i = 1}^{n} \ln x_i = 0\Rightarrow \hat\theta = - \frac{1}{n} \sum_{i = 1}^{n} \ln x_i
\end{equation*}
Проверим на несмещённость:
\begin{equation*}
    \begin{aligned}
        E\hat\theta & = -E\ln x_1 = -\int\limits_0^1\ln (x) \cdot \frac{1}{\theta}x^{\frac{1 - \theta}{\theta}}\, dx = \left<\begin{aligned}
                                                                                                                                 a  & = x^{\frac{1}{\theta}},\ \frac{d}{dx} x^{\frac{1}{\theta}} = \frac{1}{\theta} x^{\frac{1}{\theta} - 1} \\
                                                                                                                                 da & = \frac{1}{\theta}x^{\frac{1 - \theta}{\theta}}dx,\ x = a^{\theta}                                     \\
                                                                                                                             \end{aligned}  \right> = -\int\limits_{0^{\frac{1}{\theta}}}^{1^{\frac{1}{\theta}}} \ln(a^{\theta})\, da = \\
                    & = -\theta \int\limits_0^1 \ln (a)\, da = -\theta \left( a\ln a - a \right)\Big|_0^1 = \theta
    \end{aligned}
\end{equation*}
Несмещённая.
\begin{center}
    \bf Семинар 7 февраля
\end{center}
$X_1,\dots,\ X_n \sim F(x,\ \theta)$. Считается, что $\big( T_1(x_1,\dots,\ x_n),\ T_2(x_1,\dots,\ x_n) \big)$ является доверительным интервалом уровня $1 - \alpha$, если:
\[P\big(T_1(x_1,\dots,\ x_n) < \theta < T_2(x_1,\dots,\ x_n)\big) = 1 - \alpha\]
Например, для $X_1,\dots,\ X_n \sim N(m,\ \sigma^2)$, $\sigma$ известна.\\
$\hat m = \overline{X},\ \dev \overline{X} = \frac{\sigma^2}{n}\Rightarrow \frac{\sqrt{n}(\overline{X} - m)}{\sigma} \sim N(0,\ 1)$. Для построения доверительного интервала нужно оценить вероятность попадания опорной статистики на интервал:
\[P\left( Z_{\alpha/2} < \frac{\sqrt{n} (\overline{X} - m)}{\sigma} < Z_{1 - \alpha/2} \right) = 1 - \alpha\]
\[P\left( \overline{X} - \frac{\sigma Z_{1 - \alpha/2}}{\sqrt{n}} < m < \overline{X} + \frac{\sigma Z_{1 - \alpha/2}}{\sqrt{n}} \right) = 1 - \alpha\]
Если $\sigma$ тоже неизвестна, то подставляем её оценку $\tilde S = \frac{1}{n - 1} \sum\limits_{i = 1}^{n} (x_i - \overline{X})^2$ и получаем распределение Стьюдента, значит стоит брать его квантили.
\[\frac{\sqrt{n} (\overline{X} - m)}{\tilde S} = \frac{\sqrt{n} (\frac{\overline{X} - m}{\sigma})}{\frac{1}{n - 1} \sum_{i = 1}^{n} \left( \frac{x_i - \overline{X}}{\sigma} \right)^2}\]
То есть стандартное гауссовское делим на корень из $\chi^2$.\\
Итого:
\[P\left( \overline{X} - \frac{\tilde S t_{1 - \alpha/2,\ n - 1}}{\sqrt{n}} < m < \overline{X} + \frac{\tilde S t_{1 - \alpha/2,\ n - 1}}{\sqrt{n}} \right) = 1 - \alpha\]
Если математическое ожидание известно, но мы хотим интервал для дисперсии:
\[\sum_{i = 1}^{n} \frac{(x_i - m)^2}{\sigma^2} \sim \chi^2(n)\]
\[P\left( \chi^2_{n,\ 1 - \alpha/2} < \frac{\sum(x_i - m)^2}{\sigma^2} < \chi^2_{n,\ 1 - \alpha/2} \right) = 1 - \alpha\]
\[P\left(\frac{\sum_{i = 1}^{n} (x_i - m)^2}{\chi^2_{n,\ 1 - \alpha/2}} < \sigma^2 < \frac{\sum_{i = 1}^{n} (x_i - m)^2}{\chi^2_{n,\ 1 - \alpha/2}} \right) = 1 - \alpha\]
Если неизвестны оба:
\[\sum_{i = 1}^{n} \frac{(x_i - \overline{X})^2}{\sigma^2} \sim \chi^2(n - 1)\]
\[P\left(\frac{\sum_{i = 1}^{n} (x_i - \overline{X})^2}{\chi^2_{n - 1,\ 1 - \alpha/2}} < \sigma^2 < \frac{\sum_{i = 1}^{n} (x_i - \overline{X})^2}{\chi^2_{n - 1,\ 1 - \alpha/2}} \right) = 1 - \alpha\]
\subsection*{Задача}
Импортёр упаковывает чай в пакеты с номинальным весом 125 грамм. Известно, что упаковочная машина работает с известным среднеквадратическим отклонением 10 грамм. Выбрали 50 пакетов чая, выборочное среднее их веса оказалось равно $125,8$.\\
То есть $n = 50,\ \overline{X} = 125,8,\ X_1,\dots,\ X_n \sim N(m,\ 100)$.\\
$\overline{X} \sim N\left(m,\ \frac{\sigma^2}{n}\right)\Rightarrow \frac{\sqrt{n} (\overline{X} - m)}{\sigma} \sim N(0,\ 1)\Rightarrow$
\[P\left( Z_{0,025} < \frac{\sqrt{n}(\overline{X} - m)}{\sigma} < Z_{0,95} \right) = 0,95\]
\[P\left( \overline{X} - \frac{\sigma Z_{0,95}}{\sqrt{n}} < m < \overline{X} + \frac{\sigma Z_{0,95}}{\sqrt{n}} \right) = 0,95\]
\[P(123,028 < m < 128,571) = 0,95\]
125 лежит в этом интервале, поэтому всё хорошо.\\
Длина интервала получается $\frac{2\sigma Z_{0,95}}{\sqrt{n}}$, хотим, чтбы это равнялось $2$
\[\sqrt{n} = \sigma Z_{0,95} \Rightarrow n \approx 384\]
\begin{center}
    ДЗ 14 февраля
\end{center}
\subsection*{Задача 1}
10 изделий сделано за 79, 74, 112, 95, 83, 96, 77, 84, 70, 90 минут. Построить ДИ уровня 0.95 для среднего времени сборки.\\
Получаем $X_i \sim N(m,\ \sigma)$, просят доверительный интервал для $m$. С прошлого семинара:
\[P\left( \overline{X} - \frac{\tilde S t_{1 - \alpha/2,\ n - 1}}{\sqrt{n}} < m < \overline{X} + \frac{\tilde S t_{1 - \alpha/2,\ n - 1}}{\sqrt{n}} \right) = 1 - \alpha\]
Здесь:
\begin{equation*}
    \begin{aligned}
        n =                        & \, 10                                                                      \\
        \alpha =                   & \, 0.05                                                                    \\
        \overline{X} =             & \, \frac{1}{n} \sum_{i = 1}^{n} x_i                                        \\
        \tilde S^2 =               & \, \frac{1}{n - 1}\sum_{i = 1}^{n} (x_i - \overline{X})^2                  \\
        t_{1 - \alpha/2,\ n - 1} = & \, \text{так и не понял где посмотреть (квантиль распределения Стьюдента)} \\
    \end{aligned}
\end{equation*}

\subsection*{Задача 2}
Теперь ДИ для дисперсии уровня 0.9, опять воспользуемся записями семинара:
\[P\left(\frac{\sum_{i = 1}^{n} (x_i - \overline{X})^2}{\chi^2_{n - 1,\ 1 - \alpha/2}} < \sigma^2 < \frac{\sum_{i = 1}^{n} (x_i - \overline{X})^2}{\chi^2_{n - 1,\ 1 - \alpha/2}} \right) = 1 - \alpha\]

\subsection*{Задача 3}
Тоже построить ДИ для матожидания и дисперсии гауссовской величины, только с другими значениями. Из сложностей только $\tilde S^2 = \frac{n}{n - 1} S^2$
\subsection*{Задача 4}
Показать, что $\displaystyle S^2 = \hat \mu_2 - (\hat \mu_1)^2 = \frac{1}{n} \sum_{i = 1}^{n} x_i^2 - \left( \frac{1}{n}\sum_{i = 1}^{n} x_i \right)^2 = \frac{1}{n} \sum_{i = 1}^{n} x_i^2 - \frac{1}{n^2} \sum_{i = 1}^{n} x_i^2 - \frac{2}{n^2}\sum_{i = 1}^{n} x_i \sum_{j = 1}^{n} x_j$
\[S^2 = \frac{1}{n} \sum_{i = 1}^{n}(x_i - \overline{X})^2 = \frac{1}{n} \sum_{i = 1}^{n} x_i^2 - \frac{2\overline{X}}{n}\sum_{i = 1}^{n} x_i + \frac{1}{n} \sum_{i = 1}^{n} \overline{X}^2\]

\begin{center}
    \bf Семинар 14 февраля
\end{center}
Даны две выборки:
\[\begin{cases}
        X_1,\dots,\ X_n\sim N(m_1,\ \sigma_1^2) \\
        Y_1,\dots,\ Y_n\sim N(m_2,\ \sigma_2^2)
    \end{cases}\]
$\sigma_1,\ \sigma_2$ известны, тогда для построения ДИ $\theta = m_1 - m_2$:
\[\frac{\overline{X} - \overline{Y} - \theta}{\sqrt{\frac{\sigma_1^2}{n_1} + \frac{\sigma_2^2}{n_2}}} \sim N(0,\ 1)\]
Если дисперсии неизвестны, но одинаковы:
\[\hat \dev (\overline{X} - \overline{Y}) = \sigma^2 \left(\frac{1}{n_1} + \frac{1}{n_2}\right)\]
Дисперсию не знаем, поэтому подставим оценку:
\[S^2 = \frac{\sum_{i = 1}^{n} (x_i - \overline{X})^2 + \sum_{i = 1}^{n} (y_i - \overline{Y})^2}{n_1 + n_2 - 2}\]
Тогда можно сказать
\[\frac{\overline{X} - \overline{Y} - \theta}{S\sqrt{\frac{1}{n_1} + \frac{1}{n_2}}} \sim t(n_1 + n_2 - 2)\]
\subsection*{Задача}
$\overline{X} = -11.87,\ \overline{Y} = -13.75,\ \sigma_1^2 = 20,\ \sigma_2^2 = 22,\ n_1 = n_2 = 13,\ \alpha = 0.05\\
    X \sim N(m_1,\ 20),\ Y \sim N(m_2,\ 22)$
Знаем матожидания и дисперсию, тогда ДИ для $\theta = m_2 - m_1$
\[P \left( (\overline{X} - \overline{Y}) -1.96 \sqrt{\frac{\sigma_1^2}{n_1} + \frac{\sigma_2^2}{n_2}} \leq \theta \leq (\overline{X} - \overline{Y}) + 1.96 \sqrt{\frac{\sigma_1^2}{n_1} + \frac{\sigma_2^2}{n_2}} \right) = 0.95\]
\[P \left( -1.64 \leq \theta \leq 5.4 \right) = 0.95\]
Модифицирем задачу. $\sigma$ теперь неизвестны, но мы считаем их одинаковыми, тогда
\[P \left( (\overline{X} - \overline{Y}) -2.06 S \sqrt{\frac{1}{n_1} + \frac{1}{n_2}} \leq \theta \leq (\overline{X} - \overline{Y}) + 2.06 S \sqrt{\frac{1}{n_1} + \frac{1}{n_2}} \right) = 0.95\]
Если у нас посчитано ${S_{X}}^2$ и $S_Y^2$, то можем посчитать $S$:
\[S^2 = \frac{n_1 S_X^2 + n_2 S_Y^2}{n_1 + n_2 - 2}\]
Если посчитать, то получаем
\[P \left( -1.98 \leq \theta \leq 5.74 \right) = 0.95\]
\subsection*{Задача}
$X_1,\dots,\ X_n \sim \Pi(\theta)$. Построим асимптотический доверительный интервал.\\
Для распределения Пуассона верно: $\hat \theta = \overline{X},\ \dev \overline{X} = \frac{\sigma^2}{n} = \frac{\theta}{n}$\\
Тогда при больших $n$:
\[\frac{(\hat\theta - \theta)}{\sqrt{\frac{\hat\theta}{n}}} \sim N(0,\ 1)\]
\begin{equation*}
    \begin{aligned}
        P \Bigg( Z_{1 - \alpha/2}                                            & \,\leq  \frac{(\overline X - \theta)}{\sqrt{\frac{\overline X}{n}}} \leq Z_{1 - \alpha/2} \Bigg) = 1- \alpha \\
        P \Bigg( \sqrt{\frac{n}{\overline X}}Z_{1 - \alpha/2}                & \,\leq  {(\overline X - \theta)} \leq  \sqrt{\frac{n}{\overline X}}Z_{1 - \alpha/2} \Bigg) = 1- \alpha       \\
        P \Bigg( \overline{X} - \sqrt{\frac{n}{\overline X}}Z_{1 - \alpha/2} & \,\leq  \theta \leq  \overline{X} + \sqrt{\frac{n}{\overline X}}Z_{1 - \alpha/2} \Bigg) = 1- \alpha
    \end{aligned}
\end{equation*}
\begin{center}
    \bf ДЗ на 21 февраля
\end{center}
\subsection*{Задача 1}
Имеются данные о доходах Центрального федерального округа:
\[10043;\ 9596;\ 10305;\ 8354;\ 9413;\ 19776;\ 9815;\ 11311;\ 11253;\ 10856;\ 11389\Rightarrow n_x = 11,\ \overline{X} = 11\,101\]
И Приволжского федерального округа:
\[14253;\ 7843;\ 9581;\ 8594;\ 16119;\ 10112;\ 10173;\ 9756 \Rightarrow n_y = 8,\ \overline{Y} = 10\,803.875\]
Построить ДИ уровня 0.95 для разности значений среднедушевных доходов населения Центрального и Приволжского федеральных округов. Предполагается, что все наблюдения имеют гауссовское распределение и одинаковые дисперсии.\\
$X \sim N(m_1,\ \sigma^2)$ (доход в ЦФО), $Y \sim N(m_2,\ \sigma^2)$ (доход в ПФО). Оценим величину $m = m_1 - m_2$\\
Для гауссовских величин хорошей оценкой $m$ будет величина $\overline{X} - \overline{Y}$.
\begin{equation*}
    \begin{aligned}
        E\left( \overline{X} - \overline{Y} \right)         & = m                                                                                                                         \\
        \hat\dev \left( \overline{X} - \overline{Y} \right) & = \sigma^2\left( \frac{1}{n_x} + \frac{1}{n_y} \right)                                                                      \\
        \text{$\sigma$ не знаем, подставим оценку}\ S^2     & = \frac{\sum\limits_{i = 1}^{n_x} (x_i - \overline{X})^2 + \sum\limits_{i = 1}^{n_y} (y_i - \overline{Y})^2}{n_x + n_y - 2}
    \end{aligned}
\end{equation*}
Теперь мы можем составить хорошую случайную величину:
\[\frac{\left( \overline{X} - \overline{Y} \right) - m}{ \sqrt{\hat \dev \left( \overline{X} - \overline{Y} \right)}} = \frac{\overline{X} - \overline{Y} - m}{S \sqrt{\frac{1}{n_x} + \frac{1}{n_y}}} \sim t(n_x + n_y - 2) = t(17)\]
Сейчас сделаю фокус, чтобы было понятнее, почему это Стьюдент:
\[\frac{\overline{X} - \overline{Y} - m}{S \sqrt{\frac{1}{n_x} + \frac{1}{n_y}}} = \frac{\frac{\overline{X} - \overline{Y} - m}{\sigma}}{\frac{S}{\sigma} \sqrt{\frac{1}{n_x} + \frac{1}{n_y}}}\]
Далее во избежание страшных дробей я распишу числитель и знаменатель отдельно. Начнём с числителя:
\[\frac{\overline{X} - \overline{Y} - m}{\sigma}: \begin{cases}
        E\left( \frac{\overline{X} - \overline{Y} - m}{\sigma}  \right) = 0 \\
        \dev \left( \frac{\overline{X} - \overline{Y} - m}{\sigma}  \right) = 1
    \end{cases}\Rightarrow \frac{\overline{X} - \overline{Y} - m}{\sigma}  \sim N(0,\ 1)\]
Теперь знаменатель:
\[\frac{S}{\sigma} \sqrt{\frac{1}{n_x} + \frac{1}{n_y}} = \frac{\sqrt{\sum\limits_{i = 1}^{n_x} (x_i - \overline{X})^2 + \sum\limits_{i = 1}^{n_y} (y_i - \overline{Y})^2}}{\sigma\sqrt{n_x + n_y - 2}} \sqrt{\frac{1}{n_x} + \frac{1}{n_y}} = \frac{\sqrt{\sum\limits_{i = 1}^{n_x} \frac{(x_i - \overline{X})^2}{\sigma^2} + \sum\limits_{i = 1}^{n_y} \frac{(y_i - \overline{Y})^2}{\sigma^2}}}{\sqrt{n_x + n_y - 2}} \sqrt{\frac{1}{n_x} + \frac{1}{n_y}} =\]
\[= \frac{\sqrt{\sum\limits_{i = 1}^{n_x} \left(\frac{x_i - \overline{X}}{\sigma}\right)^2 + \sum\limits_{i = 1}^{n_y} \left( \frac{y_i - \overline{Y}}{\sigma}\right)^2}}{\sqrt{n_x + n_y - 2}}  \sqrt{\frac{1}{n_x} + \frac{1}{n_y}}\]
Это сумма квадратов центрированных и нормированных гауссовских величин, то есть знаменатель распределён по $\chi^2$.\\
Получается, что наша случайная величина получается в результате деления $N(0,\ 1)$ на $\chi^2$, то есть это по определению распределение Стьюдента.\\
Перед построением доверительного интервала введём обозначение $\tau = t_{17,\ 0.975} = -t_{17,\ 0.025} \approx 2.11$.
\[P\left(-\tau < \frac{\overline{X} - \overline{Y} - m}{S \sqrt{\frac{1}{n_x} + \frac{1}{n_y}}} < \tau\right) = 0.95\]
\[P\left(\left( \overline{X} - \overline{Y} \right) - \tau S \sqrt{\frac{1}{n_x} + \frac{1}{n_y}} < m < \left( \overline{{X}} - \overline{Y} \right) + \tau S \sqrt{\frac{1}{n_x} + \frac{1}{n_y}}  \right) = 0.95\]
\[P\left( -2\,446.617 < m < 3\,040.867 \right) = 0.95\]
\subsection*{Задача 2}
Для проверки качества деталей из большой партии выбрали 200 деталей. Среди них оказалось 12 бракованных. Построить асимптотический доверительный интервал уровня надёжности 0.95 для доли бракованных деталей.\\
Полагаем, что количество бракованных деталей имеет распределение $Bi(200,\ p)$, где $p$ и будет искомой долей бракованных деталей. Оценкой максимального правдободоия для $p$ является $\hat p = \frac{\overline{X}}{n}$ (было в домашке за 7 января). 200 тяжело сосчитать на пальцах, поэтому считаем его достаточно большим, чтобы применить теорему Муавра-Лапласа:
\[\frac{\hat p - p}{\sqrt{\frac{\hat p (1 - \hat p)}{n}}} \sim N(0,\ 1)\]
Теперь можно очень просто построить доверительный интервал ($z = Z_{0.975} = -Z_{0.025} = 1.96$):
\[P\left( -z < \frac{\hat p - p}{\sqrt{\frac{\hat p (1 - \hat p)}{n}}} < z \right) = 0.95\]
\[P\left( \hat p - z\sqrt{\frac{\hat p (1 - \hat p)}{n}} < p < \hat p + z\sqrt{\frac{\hat p (1 - \hat p)}{n}} \right) = 0.95\]
\[P\left( 0.027 < p < 0.093 \right) = 0.95\]
\subsection*{Задача 3}
$X\sim Bi(n_1,\ p_1),\ Y\sim Bi(n_2,\ p_2)$. По условию $n_1,\ n_2$ большие. Построить асимптотический доверительный интервал для $p = p_1 - p_2$. \Big(Показать, что статистика $\frac{\hat p_1 - \hat p_2 - (p_1 - p_2)}{\sqrt{\hat \dev \left( \hat p_1 - \hat p_2 \right)}}$, где $\hat\dev \left( \hat p_1 - \hat p_2 \right) = \frac{\hat p_1(1 - \hat p_1)}{n_1} + \frac{\hat p_2 (1 - \hat p_2)}{n_2}$, имеет асимптотически стандартное нормальное распределение\Big).\\
Найдём оценку максимального правдоподобия для $p$:
\[\xi = X - Y \Rightarrow \begin{cases}
        P(\xi = 1) = P(X = 1)\cdot P(Y = 0) = p_1 q_2                                    \\
        P(\xi = 0) = P(X = 1)\cdot P(Y = 1) + P(X = 0)\cdot P(Y = 0) = p_1 p_2 + q_1 q_2 \\
        P(\xi = -1) = P(X = 0)\cdot P(Y = 1) = q_1 p_2
    \end{cases}\]
Построим функцию правдоподобия для реализации вектора $(Z_1,\dots,\ Z_n)$, порождённого случайной величиной $\xi$:
\[L(z_1,\dots,\ z_n,\ p) = \prod_{i = 1}^{n} P(\xi = z_i)\]
Дальше непонятно, значит всё-таки надо воспользоваться подсказкой. Обозначим $T(\hat p_1 - \hat p_2) = \frac{(\hat p_1 - \hat p_2) - (p_1 - p_2)}{\sqrt{\hat \dev \left( \hat p_1 - \hat p_2 \right)}}$\\
Из предыдущей задачи:
\[\hat p_1 = \frac{\overline{X}}{n_1},\ \hat p_2 = \frac{\overline{Y}}{n_2}\]
$E\left( \hat p_1 - \hat p_2 \right) = E\hat p_1 - E\hat p_2 = p_1 - p_2\\
    \dev \left( p_1 - p_2 \right) = \frac{p_1(1 -  p_1)}{n_1} + \frac{p_2(1 - p_2)}{n_2}\\
    \hat\dev \left( \hat p_1 - \hat p_2 \right) = \hat\dev \left( \hat p_1 \right) + \hat\dev \left( \hat p_2 \right)$\\
Тогда при больших $n_1,\ n_2$ (в нашем случае это так) должно выполняться:
\[\frac{\hat p_1 - \hat p_2 - (p_1 - p_2)}{\sqrt{\hat \dev \left( \hat p_1 - \hat p_2 \right)}} \sim N(0,\ 1)\]
Что-то очень странное, надо будет уточнить на семинаре.\\
Если это верно, тогда доверительный интервал уровня $1 - \alpha$ выглядит так:
\[P\left( \hat p_1 - \hat p_2 - Z_{1 - \alpha/2} \sqrt{\hat \dev \left( \hat p_1 - \hat p_2 \right)} < p_1 - p_2 < \hat p_1 - \hat p_2 + Z_{1 - \alpha/2} \sqrt{\hat \dev \left( \hat p_1 - \hat p_2 \right)} \right) = 1 - \alpha\]
\subsection*{Задача 4}
Два года назад у 252 студентов было 29 неудов. В прошлом году у 286 оказалось 42 неуда. Построить доверительный интервал уровня надёжности 0.95 для разности вероятностей неудов в этих двух выборках.\\
Если пользоваться результатом задачи 3:
\[\frac{\hat p_1 - \hat p_2 - (p_1 - p_2)}{\sqrt{\hat \dev \left( \hat p_1 - \hat p_2 \right)}} \sim N(0,\ 1)\]
Где $\hat p_1 = \frac{29}{252},\ \hat p_2 = \frac{42}{286},\ n_1 = 252,\ n_2 = 286$, тогда:
\[ P\left( -z < \frac{\hat p_1 - \hat p_2 - (p_1 - p_2)}{\sqrt{\hat \dev \left( \hat p_1 - \hat p_2 \right)}} < z \right)  = 0.95\]
\[ P \left( \hat p_1 - \hat p_2 - z\sqrt{\hat\dev \left( \hat p_1 - \hat p_2 \right)} < p_1 - p_2 < \hat p_1 - \hat p_2 + z\sqrt{\hat\dev \left( \hat p_1 - \hat p_2 \right)} \right) = 0.95\]
\[P \left( -0.087 < p_1 - p_2 < 0.025 \right) = 0.95\]
\subsection*{Задача 5}
Из 500 опрошенных клиентов магазина 100 человек довольны обслуживанием. Построить асимптотический доверительный интервал уровня надёжности 0.95 для доли покупателей, довольных обслуживанием.\\
Полагаем, что количество довольных клиентов распределено как $Bi(500,\ p)$.\\
Считаем 500 ОГРОМНЫМ числом, поэтому:
\[\frac{\hat p - p}{\sqrt{\frac{\hat p (1 - \hat p)}{n}}} \sim N(0,\ 1)\]
Здесь $\hat p = \frac{100}{500}$, получаем:
\[P\left( -z < \frac{\hat p - p}{\sqrt{\frac{\hat p (1 - \hat p)}{n}}} < z \right) = 0.95\]
\[P\left( \hat p - z\sqrt{\frac{\hat p (1 - \hat p )}{n}} < p < \hat p + z\sqrt{\frac{\hat p (1 - \hat p )}{n}} \right) = 0.95\]
\[P\left( 0.165 < p < 0.235 \right) = 0.95\]
\subsection*{Задача 6}
Из 400 опрошенных клиентов другого магазина 70 человек довольны обслуживанием. Построить асимптотический доверительный интервал уровня надёжности 0.98 для разности долей довольных клиентов (в этой задаче и предыдущей).\\
Пользуясь результатом задачи 3 получаем:
\[P\left(  \hat p_1 - \hat p_2 - Z_{0.99} \sqrt{\hat \dev \left( \hat p_1 - \hat p_2 \right)} < p_1 - p_2 < \hat p_1 - \hat p_2 + Z_{0.99} \sqrt{\hat \dev \left( \hat p_1 - \hat p_2 \right)} \right) = 0.98\]
Здесь $\hat p_1 = \frac{100}{500},\ \hat p_2 = \frac{70}{400},\ Z_{0.99} \approx 2.326,\ \sqrt{\hat \dev\left( \hat p_1 - \hat p_2 \right)} \approx 0.026$, подставим и получим:
\[P\left( -0.036 < p_1 - p_2 < 0.086 \right) = 0.98\]
\begin{center}
    \bf Семинар 21 февраля
\end{center}
\subsection*{Задача}
Из 200 деталей 12 бракованных. Проверить гипотезу о том, что $5\%$ деталей бракованные.\\
$X_1,\dots,\ X_{200} \sim Bi(1,\ p)$\\
$H_0: p = 0.05 = p_0$ против $H_1: p > 0.05$ при $\alpha = 0.05$.\\
$X = \sum_{i = 1}^{200} x_i$ --- количество успехов\\
$T(x) = \frac{x - 200\cdot p_0}{\sqrt{200p_0(1 - p_0)}}\\
    T(x)\big|_{H_0} \sim N(0,\ 1)$\\
Тогда доверительная область $(-\infty,\ Z_{0.95}) = (-\infty,\ 1.64)$\\
$T(x) = \frac{12 - 10}{\sqrt{200\cdot 0.05\cdot 0.95}} = 0.649\Rightarrow$ попали в доверительную область, значит верим $H_0$.
\subsection*{Задача}
Проводится тестирование по английскому языку. Предлагается 100 вопросов, на каждый из которых 4 ответа, 1 из них правильный. Один студент ответил правильно на 30 вопросов. Можно ли считать при $\alpha = 0.05$, что этот студент не знает английский язык?\\
$X_1,\dots,\ X_{100} \sim Bi(1,\ p)$\\
$H_0: p = 0.25 = p_0$ (то есть студент угадывает ответы $\Rightarrow$ не знает).\\
$H_1: p > 0.25$\\
Статистику возьмём такую же, как в прошлой задаче $T(x) = \frac{x - np_0}{\sqrt{np_0(1 - p_0)}}$. Тогда:
\[T(x)\big|_{H_0} \sim N(0,\ 1)\]
Критической областью является $(Z_{0.95},\ +\infty)$. Теперь посчитаем статистику:
\[T(30) = \frac{30 - 25}{\sqrt{100\cdot 0.25\cdot 0.75}} = 1.15 < Z_{0.95}\]
Статистика попала в доверительную область, значит студент не знает английский язык.
\subsection*{Задача}
Проведено исследование по выведению факторов риска заболеваемости туберкулёзом. Одним из факторов считается низкий доход в семье. Среди 300 семей с низким доходом 12 больных, среди 100 семей с высоким доходом 2 больных. Можно ли сказать, что низкий доход влияет на заболеваемость.
\newpage
\begin{center}
    \bf ДЗ к 28 февраля
\end{center}
\subsection*{Задача 1}
Для прохода в парламент необходимо 7\% голосов избирателей. Опросили 1\,000 человек, 68 из которых собираются голосовать за партию $A$. Можно ли на уровне значимости 0.05 считать, что партия $A$ пройдёт в парламент?
\subsubsection*{Решение}
Считаем, что для случайной величины $\xi = \{\text{``Количество проголосовавших за $A$''}\}$ справедливо
\[\xi \sim Bi(1\,000,\ p)\]
Составим две гипотезы про полученное распределение:
\[H_0: p = p_0 = 0.07,\ \text{против } H_1: p < 0.07\]
Если верна $H_0$, то мы можем считать, что партия прошла в парламент.\\
Уровень значимости указан в задаче и равен $0.05$.\\
Выбираем следующую статистику:
\[T(x) = \frac{\overline{X} - p_0}{\sqrt{ \frac{p_0(1 - p_0)}{1\,000} }}\]
В общем случае про эту статистику мы ничего сказать не можем, но при условии верности $H_0$ она становится центрированной и нормированной, потому что:
\begin{equation*}
    \begin{aligned}
         & \overline{X} = \hat{p}                                                                                                                                                                                                          \\
         & E\left( \hat{p} \right)\big|_{H_0} = E\left( \frac{1}{n}\sum_{i = 1}^{n}x_i \right) = Ex_i = p_0                                                                                                                                \\
         & \dev\left( \hat{p} \right)\big|_{H_0} = \dev \left( \frac{1}{n}\sum_{i = 1}^{n}x_i \right) = \frac{1}{n^2}\sum_{i = 1}^{n} \dev\left( x_i \right) = \frac{\dev(x_i)}{n} = \frac{p_0(1 - p_0)}{n} = \frac{p_0( 1 - p_0)}{1\,000}
    \end{aligned}
\end{equation*}
В этих уравнениях я преобразовал $\xi$ в случайный вектор $X$ состоящий из $1\,000$ случайных величин $Be(p_0)$.\\
Итак, $T(x)$ действительно центрированная и нормированная случайная величина, значит
\[T(x) \xrightarrow[n\to\infty]{} U \sim N(0,\ 1)\]
В нашем случае $\overline{X} = 68$, посчитаем эту статистику:
\[T(x) = \frac{\frac{68}{1\,000} - 0.07}{\sqrt{\frac{p_0(1 - p_0)}{1\,000}}} = -0.248\]
Доверительным интервалом в нашем случае является $(Z_{0.05},\ +\infty)$, где $Z_{0.05} = -1.64$.\\
Получаем $T(68) \in (-1.64,\ +\infty)\Rightarrow\,$принимаем $H_0$
\subsubsection*{Ответ}
Да, можно так считать.
\subsection*{Задача 2}
Известно, что женщины-водители составляют $30\%$ от общего числа водителей. Зафиксировали $n = 635$ ДТП, 132 из которых произошли по вине женщин-водителей. Можно ли на уровне значимости $0.01$ считать, что женщины водят машину аккуратнее (реже попадают в ДТП)?
\subsubsection*{Решение}
Пусть случаный вектор $X$, каждый элемент которого $x\sim Be(p)$ равен $1$, если причиной ДТП была женщина и 0 иначе.
Составим две гипотезы
\[H_0: p = p_0 = 0.3,\ \text{против } H_1: p < 0.3\]
Принятие $H_0$ означает, что женщины водят не аккуратнее мужчин (отрицательный ответ на вопрос задачи).\\
Требуемый уровень значимости: 0.01\\
Составим статистику:
\[T(x) = \frac{\overline{X} - np_0}{\sqrt{np_0 q_0}}\]
Если считать $H_0$ верной, то $np_0 = EX$, а $\sqrt{np_0q_0} = \sqrt{\dev X}$, значит:
\[T(x)\big|_{H_0} \xrightarrow[n\to \infty]{}U\sim N(0,\ 1)\]
В нашем случае $\overline{X} = \{\text{``Количество ДТП из-за женщин''}\} = 132$.\\
Доверительный интервал $(Z_{0.01},\ +\infty) \approx (-2.326,\ +\infty)$.
\[T(x) = \frac{132 - 635\cdot 0.3}{\sqrt{635\cdot 0.3\cdot 0.7}} \approx -5.066\]
Попали в критическую область, значит принимает альтернативную гипотезу.
\subsubsection*{Ответ}
Да, можно так считать.
\subsection*{Задача 3}
Есть два пресса, штампующих одинаковые детали. Из $n_1 = 1\,000$ деталей первого пресса оказалось 25 бракованных. Из $n_2 = 800$ деталей второго пресса оказалось 18 бракованных. Можно ли на уровне значимости 0.01 считать, что доля брака у этих прессов одинакова?
\subsubsection*{Решение}
Объявим две случайные величины $\xi \sim Be(p_1),\ \eta \sim Be(p_2)$.\\
Случаный вектор $X$, порождённый 1\,000 случаных величин $\xi$, и случаный вектор $Y$, порождённый 800 случайными величинами $\eta$.\\
Гипотезы:
\[H_0: p_1 = p_2,\ \text{против } H_1: p_1 \neq p_2\]
Принятие $H_0$ означает положительный ответ на задачу.\\
Требуемый уровень значимости $0.01$\\
Перефразируем гипотезы:
\[H_0: p_1 - p_2 = 0,\ \text{против } H_1: p_1 - p_2 \neq 0\]
Теперь можно составить статистику:
\[T(x,\ y) = \frac{\overline{X} - \overline{Y} - (p_1 - p_2)}{\sqrt{\dev(\overline{X} - \overline{Y})}}\]
Если считать $H_0$ верной, можно сделать следующие преобразования:
\begin{equation*}
    \begin{aligned}
        p_1 - p_2                                      & = 0;                                                                                                                                                                                                                                                                                   \\
        \dev(\overline{X} - \overline{Y})              & = \dev(\overline{X}) + \dev(\overline{Y}) = \frac{\left( \dev (\xi) \right)^2}{n_1} + \frac{\left( \dev (\eta) \right)^2}{n_2} = \left< \begin{aligned}
                                                                                                                                                                                                      & H_0: p_1 = p_2\Rightarrow \xi \sim \eta \\
                                                                                                                                                                                                      & \dev(\xi) = \dev(\eta) = \sigma^2
                                                                                                                                                                                                 \end{aligned} \right> =                                                                                                     \\
                                                       & =\sigma^2\left( \frac{1}{n_1} + \frac{1}{n_2} \right);                                                                                                                                                                                                                                 \\
        \sigma^2                                       & = (n_1 + n_2) \hat p \left( 1 - \hat p \right) = (n_1 + n_2)\frac{\overline{X} + \overline{Y}}{n_1 + n_2} \left( 1 - \frac{\overline{X} + \overline{Y}}{n_1 + n_2} \right) = \left(\overline{X} + \overline{Y}\right)\left( 1 - \frac{\overline{X} + \overline{Y}}{n_1 + n_2} \right); \\
        \dev\left( \overline{X} - \overline{Y} \right) & = \left( \overline{X} + \overline{Y} \right)\left( 1 - \frac{\overline{X} + \overline{Y}}{n_1 + n_2} \right)\left( \frac{1}{n_1} + \frac{1}{n_2} \right) = (25 + 18)\left( 1 - \frac{25 + 18}{1\,000 + 800} \right)\left( \frac{1}{1\,000} + \frac{1}{800} \right) \approx             \\
                                                       & \approx 0.0944 \Rightarrow \sqrt{\dev\left( \overline{X} - \overline{Y} \right)} \approx 0.307
    \end{aligned}
\end{equation*}
Снова получаем что-то сходящееся к центрированной и нормированной гауссовской величине:
\[T(x,\ y) = \frac{\overline{X} - \overline{Y} - 0}{\sqrt{\dev\left( \overline{X} - \overline{Y} \right)}} \xrightarrow[n\to\infty]{}U\sim N(0,\ 1)\]
При подстановке наших чисел получаем:
\[T(x,\ y) = \frac{\frac{25}{1000} - \frac{18}{800}}{0.307} \approx 0.008\]
Доверительный интервал в данной задаче $(Z_{0.005},\ Z_{0.995}) = (-2.576,\ 2.576)$. Статистика попала в доверительный интервал, значит мы верим $H_0$
\subsubsection*{Ответ}
Да, можно
\subsection*{Задача 4}
В Москве 66 человек из $n_1 = 600$ недовольны своей работой. В Московской области 60 человек из $n_2 = 500$ недовольны своей работой. Можно ли на уровне значимости 0.05 считать, что в области доля недовольных выше?
\subsubsection*{Решение}
Вспомним товарища Дашкова:\\
$X = (x_1,\dots,\ x_{600}),\ \forall x\quad x\in X\Rightarrow x\sim Be(p_1)$\\
$Y = (y_1,\dots,\ y_{500}),\ \forall y\quad y\in Y\Rightarrow y\sim Be(p_2)$\\
Гипотезы:
\[H_0: p_1 = p_2,\ \text{против } H_1: p_1 < p_2\]
Принятие $H_0$ влечёт отрицательный ответ на задачу.\\
Уровень значимости 0.05\\
Снова пошаманим с гипотезами:
\[H_0: p_1 - p_2 = 0,\ \text{против } H_1: p_1 - p_2 < 0\]
Составляем статистику:
\[T(x,\ y) = \frac{\overline{X} - \overline{Y} - (p_1 - p_2)}{\sqrt{\dev \left( \overline{X} - \overline{Y} \right)}}\]
Это идентично предыдущей задаче, поэтому сразу запишу всё получаемое при условии верности $H_0$:
\begin{equation*}
    \begin{aligned}
        p_1 - p_2                                             & = 0;                                                                                                                                                      \\
        \dev\left( \overline{X} - \overline{Y} \right)        & = \left( \overline{X} + \overline{Y} \right)\left( 1 - \frac{\overline{X} + \overline{Y}}{n_1 + n_2} \right)\left( \frac{1}{n_1} + \frac{1}{n_2} \right)= \\
                                                              & =(66 + 60)\left( 1 - \frac{66+60}{600+500} \right)\left( \frac{1}{600} + \frac{1}{500} \right) \approx 0.409;                                             \\
        \sqrt{\dev\left( \overline{X} - \overline{Y} \right)} & \approx 0.634
    \end{aligned}
\end{equation*}
Думаю уже понятно, что полученная нами центрированная и нормированная статистика сходится к $N(0,\ 1)$.
Подставим числа в статистику:
\[T(x,\ y) = \frac{\frac{66}{600} - \frac{60}{500}}{0.634} \approx -0.016\]
Доверительный интервал в нашем случае $\left( Z_{0.05},\ +\infty \right) \approx (-1.645,\ +\infty)$. Статистика лежит в доверительном интервале, значит верна $H_0\Rightarrow\,$ответ на задачу отрицательный.
\subsubsection*{Ответ}
Нет, нельзя
\subsection*{Задача 5}
Вероятность рождения мальчика $p_0 = 0.52$, в случайной выборке из $n = 5\,000$ людей от 30 до 60 лет оказалось 2\,500 мужчин и 2\,500 женщин. Можно ли на уровне значимости 0.05 считать, что смертность мужчин и женщин одинакова.
\subsubsection*{Решение}
$X = (x_1,\dots,\ x_{5\,000}),\ \forall x\in X\quad x\sim Be(p)$, в таком случае $\overline{X}$ --- частота события \{``Встретить мужчину''\}\\
Гипотезы:
\[H_0: p = p_0 = 0.52,\ \text{против } H_1: p < 0.52\]
Принятие $H_0$ означает, что смертность одинакова, то есть положительный ответ на задачу.
Уровень значимости 0.05\\
Составим статистику:
\[T(x) = \frac{\overline{X} - p_0}{\sqrt{\dev\left( \overline{X} \right)}}\]
При верности $H_0$ можем сказать, что $T(x)$ сходится к центрированной и нормированной гауссовской величине, помимо этого:
\[\sqrt{\dev(\overline{X})} = \sqrt{\dev\left( \frac{1}{n}\sum_{i = 1}^{n}x_i \right)} = \frac{1}{n} \sqrt{n\dev(x_1)} = \left.\sqrt{\frac{\dev(x_1)}{n}}\right|_{H_0} = \sqrt{\frac{p_0 (1 - p_0)}{n}} \approx 0.007\]
При подстановке в статистику получаем
\[T(x) = \frac{\frac{2\,500}{5\,000} - 0.52}{0.007} \approx -2.83\]
Доверительный интервал в нашем случае $(Z_{0.05},\ +\infty) \approx (-1.645,\ +\infty)$. Статистика попадает в критическую область, значит мы принимаем $H_1$.
\subsubsection*{Ответ}
Нет, нельзя.
\newpage
\begin{center}
    \bf Семинар 28 февраля.
\end{center}
\subsection*{Полезная информация}
$x_1,\dots,\ x_n \sim N(m,\ \sigma^2)$, дисперсия известна.\\
Проверяем гипотезу $H_0: m = m_0$. Для этого подходит статистика:
\[T(x) = \frac{(\overline{X} - m_0)\sqrt{n}}{\sigma} \sim N(0,\ 1)\]
Если дисперсия неизвестна, то подойдёт:
\[ T(x) = \frac{(\overline{X} - m_0) \sqrt{n - 1}}{S} = \frac{(\overline{X} - m_0)}{\tilde S} \sim t(n - 1)\]
\subsection*{Задача}
$n = 50,\ X = (x_1,\dots,\ x_n)\sim N(m,\ 10^2),\ \overline{X} = 125.8$. Хотим матожидание $125$.\\
Проверяем гипотезу $H_0: m = m_0 = 125$ против $H_1: m \neq 125$. Составим статистику:
\[T(x) = \frac{(\overline{X} - m_0)\sqrt{n}}{\sigma} = \frac{(125.8 - 125)\sqrt{50}}{10} \approx \frac{1}{2}\]
Доверительный интервал $(Z_{0.025},\ Z_{0.975}) \approx (-1.96,\ 1.96)$. Статистика попала в доверительный интервал.
\subsection*{Задача}
Есть выборка $X = (x_1,\dots,\ x_n),\ n = 10,\ \forall x \in X\quad x\sim N(m,\ \sigma^2)$. Никакие параметры неизвестны. Хотим проверить гипотезу $H_0: m = m_0 = 90$ против $H_1: m < 90$\\
$\overline{X} = 86,\ n = 10,\ \tilde S = 12.55,\ \alpha = 0.05$. Рассмотрим статистику:
\[T(x) = \frac{(86 - 90)\sqrt{10}}{12.55} = \frac{-4\sqrt{10}}{12.55} \approx -1.008\]
Доверительный интервал $(t_{9,\ 0.05},\ +\infty) = (-1.833,\ +\infty)$, то есть попали в доверительный, значит принимаем $H_0$.
\subsection*{Ещё полезной информации}
Если проверяем гипотезу $H_0: \sigma^2 = \sigma^2_0$:\\
С известным математическим ожиданием:
\[\frac{\sum_{i = 1}^{n} \left( x_i - m \right)^2}{\sigma_0^2}\big|_{H_0} \sim \chi^2(n)\]
С неизвестным математическим ожиданием:
\[\frac{\sum_{i = 1}^{n} \left( x_i - \overline{X} \right)^2}{\sigma_0^2}\big|_{H_0} \sim \chi^2(n - 1)\]
\subsection*{Задача}
$n = 25,\ \frac{1}{n}\sum_{i = 1}^{n}(x_i - \mu)^2 = 0.02,\ \alpha = 0.05$. Проверяем гипотезу $H_0: \sigma^2 = \sigma_0^2 = 0.01$, альтернатива $H_1: \sigma^2 > 0.01$.\\
Люди, которые для нас посчитали сумму знали математическое ожидание случайной величины, поэтому используем формулу, в которой оно известно:
\[T(x) = \frac{\sum_{i = 1}^{n}(x_i - \mu)^2}{\sigma_0^2} = \frac{0.5}{0.01} = 50\]
Доверительная область $(-\infty,\ \chi^2_{25,\ 0.95}) \approx (-\infty,\ 37.65)$. В доверительную область статистика не попала, значит принимаем $H_1$.
\subsection*{Задача}
Станок штампует валики, в выборке объёма $n = 17$, выборочное среднее получилось $20.5$, выборочная дисперсия $S^2 = 16$. Проверить на уровне значимости 0.05 гипотезу $H_0: \sigma^2 = 18$ (альтернативой будет $H_1: \sigma^2 \neq 18$).\\
Математическое ожидание и дисперсию мы не знаем, поэтому берём статистику
\[T(x) = \frac{\overset{=nS^2}{\overbrace{\sum_{i = 1}^{n}\left( x_i - \overline{X} \right)^2}}}{18} = \frac{17\cdot 16}{18}\approx 15.11 \sim \chi^2(16)\]
Доверительный интервал $(\chi^2_{0.25,\ 16},\ \chi^2_{0.975,\ 16}) \approx (6.9,\ 28.84)\Rightarrow$ статистика не попала, принимаем $H_1$.
\begin{center}
    \bf Семинар 7 марта
\end{center}
\subsection*{Задача}
В первых 10\,000 знаках после запятой числа $\pi$ справедлива такая таблица 
\[\begin{tabular}{|c|c|}
    \hline
    Цифра & Количество вхождений $\nu_i$\\
    \hline
    0 & 968\\
    \hline
    1 & 1026\\
    \hline
    2 & 1021\\
    \hline
    3 & 974\\
    \hline
    4 & 1014\\
    \hline
    5 & 1046\\
    \hline
    6 & 1021\\
    \hline
    7 & 970\\
    \hline
    8 & 948\\
    \hline
    9 & 1012\\
    \hline
\end{tabular}\]
Проверяем $H_0: p_1 = \dots = p_{10} = \frac{1}{10}\\
\hat \chi^2 = \sum\limits_{k = 1}^{10} \frac{n}{p_k^{(0)}} \left( \hat p_k - p_k^{(0)}\right)\Big|_{H_0} \sim \chi^2 (9)$. $\hat p_k = \frac{\nu_k}{n},\ p_k\big|_{H_0} = \frac{1}{10}$.\\
В нашем случае $\hat \chi^2 = \sum_{i =1}^{10}\frac{\left(\frac{\nu_i}{n} - p_k\right)^2}{p_k} = 9.318$, квантиль $\chi^2_{9,\ 0.95} = 16.92$. Попали в доверительную облатсь, значит $H_0$ верна.
\subsection*{Задачи на критерий Стьюдента}
$X_1,\dots,\ X_m \sim N(m_1,\ \sigma^2)\\
Y_1,\dots,\ Y_n \sim N(m_2,\ \sigma^2)$\\
$X,\ Y$ независимы, $\sigma$ неизвестны, но одинаковы (если сомневаемся, то проверяем критерием Фишера). Тогда для проверки гипотезы $H_0: \theta =  m_1 - m_2 = 0$ можно использовать статистику
\[T(x,\ y) = \frac{\overline{X} - \overline{Y}}{S\sqrt{\frac{1}{n} + \frac{1}{m}}}\Big|_{H_{0}} \sim t(m + n - 2)\]
$S^2 = \frac{\sum_{i = 1}^{m} (x_i - \overline{X})^2 = \sum_{i = 1}^{n} (y_i - \overline{Y})^2}{m + n - 2}$
\subsection*{Задача}
$X_1,\dots,\ X_{29} \sim N(m_1,\ \sigma^2)\\
Y_1,\dots,\ Y_{16} \sim N(m_2,\ \sigma^2)\\
H_0: m_1 = m_2$ против $H_1: m_1 > m_2$.\\
Дано $\overline{X} = 2.5,\ \overline{Y} = 2.06,\ S_{x}^2 = 0.67,\ S_{y}^2 = 0.42\\
T(x,\ y) = \frac{\overline{X} - \overline{Y}}{S\sqrt{\frac{1}{n} + \frac{1}{m} }}\\
S^2 = \frac{\sum_{i = 1}^{29} (x_i - \overline{X})^2 + \sum_{i = 1}^{16} (y_i - \overline{Y})}{m + n - 2} = \frac{nS_x^2 + mS_y^2}{m + n - 2} = 0.608\Rightarrow S = 0.78$\\
Подставим числа в статистику, получаем $T(x,\ y) = 1.81$, а квантиль $t_{43,\ 0.95} \approx 1.68$. Критическая область находится справа, значит статистика попала в критическую область и мы принимаем $H_1$.
\subsection*{Задача про кобальт и кроликов}
$X_1,\dots,\ X_m \sim F(t),\ \text{контрольная выборка}\\
Y_1,\dots,\ Y_n \sim F(t - \theta),\ \text{опытная группа}$.\\
Проверяем гипотезу $H_0: \theta = 0$ против $H_1: \theta > 0$. На лекции доказывалось, что $\theta = EY - EX$.\\
Распределение мы не знаем, значит не знаем ничего про равенство дисперсий, значит критерий Стьюдента применять мы здесь не можем, воспользуемся критерием Вилкоксона.\\
\[W = \sum_{i = 1}^{n} R_i,\ \text{сумма рангов $y$ в вариационном ряду}\]
Контрольная группа: 560, 580, 600, 420, 530, 490, 580, 740\\
Опытная группа: 692, 700, 621, 640, 561, 680, 630\\
Упорядочим: $420_x,\ 490_x,\ 530_x,\ 560_x,\ 561_y,\ 580_x,\ 580_x,\ 600_x,\ 621_y,\ 630_y,\ 640_y,\ 680_y,\ 692_y,\ 700_y,\ 740_x$\\
Итого ранги $y:$ 5, 9, 10, 11, 12, 13, 14. Сумма рангов $74$. Если хотим проверить на уровне значимости $0.05$ столкнёмся с проблемой дискретности распределения, поэтому придётся брать квантиль уровня $0.047$, который равен 71. Попали в критическую область, значит кобальтовые добавки влияют на вес кроликов.\\
$EW_{8,\ 7} = (8 + 7 + 1)\frac{7}{2} = 56,\ \dev W_{8,\ 7} = \frac{7\cdot 8}{12} \cdot (8 + 7 + 1) \approx 74.667$\\
Если брать аппроксимацию гауссовской величиной:
\[W^* = \frac{W - 56}{\sqrt{72}} \sim N(0,\ 1)\]
В наших числах $W^* \approx 2,12$. На уровне значимости 0.95, а критическая область начинается с $1.64$, то есть тут тоже приняли бы альтернативу. 

\begin{center}
    \bf Домашнее задание к 14 марта.
\end{center}
\subsection*{Задача 1}
Наблюдались показания $n = 500$ наугад выбранных часов, выставленных в витринах  часовщиков. Пусть $i$ - номер промежутка от $i$ – го часа до $(i+1)$-го часа, $i = 0,\ 1,\dots,\ 11$.
\[\begin{tabular}{|c|c|c|c|c|c|c|c|c|c|c|c|c|}
        \hline
        $i$   & 0  & 1  & 2  & 3  & 4  & 5  & 6  & 7  & 8  & 9  & 10 & 11 \\
        \hline
        $n_i$ & 41 & 34 & 54 & 39 & 49 & 45 & 41 & 33 & 37 & 41 & 47 & 39 \\
        \hline
    \end{tabular}\]
Согласуются ли эти данные с гипотезой о том, что показания часов распределены равномерно в интервале (0;12)? Уровень значимости принять равным 0.05.
\subsubsection*{Решение}
Нам дана гипотеза $H_0: X_1,\dots,\ X_{500} \sim R(0,\ 12)\Rightarrow F(x) = \begin{cases}
        0,\             & x < 0          \\
        \frac{x}{12},\  & x \in [0,\ 12] \\
        1,\             & x > 12
    \end{cases}$\\
На интервалы уже всё разбили, неизвестных параметров в распределении нет, значит оценивать их не надо. Посчитаем частоты попадания на интервалы:
\[\begin{tabular}{|c|c|c|c|c|c|c|c|c|c|c|c|c|}
        \hline
        $i$        & 0                & 1                & 2                & 3                & 4                & 5                & 6                & 7                & 8                & 9                & 10               & 11               \\
        \hline
        $\hat p_i$ & $\frac{41}{500}$ & $\frac{34}{500}$ & $\frac{54}{500}$ & $\frac{39}{500}$ & $\frac{49}{500}$ & $\frac{45}{500}$ & $\frac{41}{500}$ & $\frac{33}{500}$ & $\frac{37}{500}$ & $\frac{41}{500}$ & $\frac{47}{500}$ & $\frac{39}{500}$ \\
        \hline
    \end{tabular}\]
При справедливости $H_0$ вероятность попадания на все интервалы должна быть одинаковой (так как интервалы поровну делят область некоторого равномерного распределения), то есть $\forall i = \overline{0,\ 11}\quad p_i^{(0)} = \frac{1}{12}$\\
По критерию хи-квадрат:
\[\hat \chi^2 = \sum_{i = 0}^{11} \frac{500}{p_{i}^{(0)}} {\left( \hat p_i - p^{(0)}_i \right)}^2\Big|_{H_0} \sim \chi^2\left( 11 \right)\]
Доверительным интервалом будет $(0,\ \chi^2_{0.95,\ 11}) = (0,\ 19.675)$. Наша оценка примерно равна 10, значит попала в доверительную область.
\subsubsection*{Ответ}
Согласуются.
\subsection*{Задача 2}
В некоторой компании работает $n = 500$ продавцов, на каждого из которых может поступить жалоба. За последний месяц на 275 продавцов жалоб не поступало, на 150 поступило по одной жалобе, на 50 – по две жалобы, на остальных – три или более жалоб. С помощью критерия хи-квадрат проверьте гипотезу о том, что количество жалоб на продавца есть случайная величина подчиняющаяся распределению Пуассона со средним значением одна жалоба в месяц. Уровень значимости считать равным 0.05.
\[\begin{tabular}{|c|c|c|c|c|}
        \hline
        $i$   & 0   & 1   & 2  & 3+ \\
        \hline
        $n_i$ & 275 & 150 & 50 & 25 \\
        \hline
    \end{tabular}\]
\subsubsection*{Решение}
Нужно проверить гипотезу $H_0: X_1,\dots,\ X_{500} \sim \Pi(1)\Rightarrow P(X_i = k) = \frac{e^{-1} 1^k}{k!} = \frac{1}{k!\cdot e}$\\
Разбили на 4 интервала (для продавцов с 0, 1, 2, 3+ жалобами), неизвестных параметров в распределении нет, значит ничего не надо оценивать. Посчитаем частоты:
\[\begin{tabular}{|c|c|c|c|c|}
        \hline
        $i$        & 0                 & 1                 & 2                & 3+               \\
        \hline
        $\hat p_i$ & $\frac{275}{500}$ & $\frac{150}{500}$ & $\frac{50}{500}$ & $\frac{25}{500}$ \\
        \hline
    \end{tabular}\]
При справедливости $H_0$ они должны выглядеть так:
\[\begin{tabular}{|c|c|c|c|c|}
        \hline
        $i$         & 0             & 1             & 2              & 3+                 \\
        \hline
        $p_i^{(0)}$ & $\frac{1}{e}$ & $\frac{1}{e}$ & $\frac{1}{2e}$ & $1 - \frac{5}{2e}$ \\
        \hline
    \end{tabular}\]
По критерию хи-квадрат должно выполняться:
\[\hat \chi^2 = \sum_{i = 0}^{3} \frac{500}{p_i^{(0)}}{\left( \hat p_i - p_i^{(0)}\right)}^2\Big|_{H_0} \sim \chi^2\left( 3 \right)\]
Доверительная область будет $(0,\ \chi^2_{0.95,\ 3}) = (0,\ 7.814)$. Оценка получилась 76.21, походу мимо.
\subsubsection*{Ответ}
Гипотеза отвергается.

\subsection*{Задача 3}
Имеется набор данных «ирисы Фишера». Эти данные собраны ботаником Эдгаром Андерсоном. Они включают длину и ширину чашелистиков, длину и ширину лепестков трёх видов ирисов (setosa, versocolor и виргинский). Выберем из этих данных случайным образом по 10 измерений длин чашелистиков цветов вида setosa и цветов вида versocolor.\\
Длина (в мм) чашелистиков у выбранных цветов вида setosa:
\[X = (5.1,\ 4.9,\ 4.7,\ 4.6,\ 5.0,\ 5.4,\ 4.6,\ 4.4,\ 4.8,\ 4.8)\Rightarrow \overline{X} = 4.83\]
Длина (в мм) чашелистиков у цветов вида versocolor:
\[Y = (5.7,\ 6.3,\ 4.9,\ 6.6,\ 5.2,\ 5.0,\ 5.9,\ 6.0,\ 5.6,\ 5.8)\Rightarrow \overline{Y} = 5.7\]
Можно ли считать, опираясь на эти данные, что длина чашелистиков у цветов вида versocolor в среднем больше, чем у цветов вида setosa? Решите данную задачу в ситуации, когда:
\begin{enumerate}
    \item[а)] предполагается, что наблюдаемый признак имеет гауссовское распределение; 
    \item[б)] нет предположения о виде распределения наблюдаемого признака. 
\end{enumerate}
\subsubsection*{Решение а)}
$X \sim N(m_1,\ \sigma^2_1),\ Y \sim N(m_2,\ \sigma^2_2)$. Дисперсии неизвестны, могут быть различны. Хотим проверить гипотезу $H_0: m_1 = m_2$ против $H_1: m_2 > m_1$
\[T(x,\ y) = \frac{\overline{Y} - \overline{X} - (m_2 - m_1)}{\sqrt{\dev\left( \overline{Y} - \overline{X} \right)}} = \frac{\overline{Y} - \overline{X} - (m_2 - m_1)}{\sqrt{\frac{\sigma_1^2}{n_1} + \frac{\sigma_2^2}{n_2}}}\Bigg|_{H_0} = \frac{\overline{Y} - \overline{X}}{\sqrt{\frac{\sigma_1^2}{n_1} + \frac{\sigma_2^2}{n_2}}}\]
Дисперсии не знаем, оценим их:
\[\sigma_1^2 \to \tilde S^2_1 = \frac{1}{n_1 - 1} \sum_{i = 1}^{n_1} {\left( x_i - \overline{X} \right)}^2 \approx 0.082\]
\[\sigma_2^2 \to \tilde S^2_2 = \frac{1}{n_2 - 1} \sum_{i = 1}^{n_2} {\left( y_i - \overline{Y} \right)}^2 \approx 0.3\]
При справедливости $H_0$ должны получить распределение Стьюдента $t(n_1 + n_2 - 2)$. Посчитаем значение статистики:
\[T(x,\ y) \approx \frac{5.7 - 4.83}{\sqrt{\frac{0.082}{9} + \frac{0.3}{9}}} \approx 4.223\]
Доверительный интервал $(-\infty,\ t_{18,\ 0.95}) = (-\infty,\ 1.734)$. Попали в критическую область, значит принимаем $H_1$, то есть длина чашелистиков у versocolor действительно в среднем больше.
\subsubsection*{Решение б)}
$X\sim F(t),\ Y\sim F(t - \theta)$, проверяем гипотезу $H_0: \theta = 0$ против $H_1: \theta > 0$\\
Про дисперсии ничего не знаем, поэтому надо пользоваться ранговым критерием Вилкоксона.\\
Ранжируем данные:
\[\underset{1}{\underbrace{4.4_X}},\ \underset{2.5}{\underbrace{4.6_Y,\ 4.6_X}},\ \underset{4}{\underbrace{4.7_X}},\ \underset{5.5}{\underbrace{4.8_X,\ 4.8_X}},\ \underset{7.5}{\underbrace{4.9_X,\ 4.9_Y}},\ \underset{9.5}{\underbrace{5.0_X,\ 5.0_Y}},\]
\[\underset{11}{\underbrace{5.1_X}},\ \underset{12}{\underbrace{5.2_Y}},\ \underset{13}{\underbrace{5.4_X}},\ \underset{14}{\underbrace{5.6_Y}},\ \underset{15}{\underbrace{5.7_Y}},\ \underset{16}{\underbrace{5.8_Y}},\ \underset{17}{\underbrace{5.9_Y}},\ \underset{18}{\underbrace{6.0_Y}},\ \underset{19}{\underbrace{6.3_Y}},\ \underset{20}{\underbrace{6.6_Y}}\]
Ранги $y$: 2.5, 7.5, 9.5, 12, 14, 15, 16, 17, 18, 19, 20. Сумма получается 150.5. Если размеры равны, то ожидается
\[EW_{10,\ 10}\Big|_{H_0} = (10 + 10 + 1)\cdot 5 = 105,\quad \dev W_{10,\ 10}\Big|_{H_0} = \frac{100}{12}(10 + 10 + 1) = \frac{2100}{12} = 175\]
Следующая штука должна иметь стадартное гауссовское распределение:
\[W^* = \frac{W_{10,\ 10} - EW_{10,\ 10}}{\sqrt{\dev W_{10,\ 10}}} = \frac{150.5 - 105}{\sqrt{175}} \approx 3.439\]
$Z_{0.95} = 1.64$, доверительный интервал $(-\infty,\ 1.64)$, значит попали в критическую, то есть $Y$ в среднем больше, результаты совпали.
\subsubsection*{Ответ}
По обоим методам можно так считать.

\subsection*{Задача 4}
Деятельность отделения банка характеризуется некоторым показателем Х. Для проверки была случайным образом сделана выборка 10 однотипных отделений банка. Показатель Х у этих отделений составил:
\[Y = (258,\ 588,\ 477,\ 577,\ 619,\ 614,\ 641,\ 543,\ 517,\ 593)\]
После экономического кризиса показатель Х у 9 случайным образом выбранных отделений банка составил:
\[Z = (537,\ 398,\ 256,\ 440,\ 376,\ 524,\ 527,\ 589,\ 479)\]
Можно ли считать, опираясь на эти данные, что экономический кризис привёл к снижению показателя Х. Уровень значимости принять равным 0.05.
\subsubsection*{Решение}
$Y \sim F(t),\ Z \sim F(t - \theta)$, гипотеза $H_0: \theta = 0 $, против $H_1: \theta < 0$\\
Про распределение и его дисперсию мы ничего не знаем, поэтому воспользуемся критерием Вилкоксона.\\
Сумма рангов $Z$ получилась 65.
\[EW_{10,\ 9}\Big|_{H_0} = (10 + 9 + 1)\cdot \frac{9}{2} = 45,\quad \dev W_{10,\ 9} \Big|_{H_0} = \frac{90}{12}(20) = 150\]
Построим статистику (она, кстати, номральной гауссовской должна быть):
\[W^* = \frac{65 - 45}{\sqrt{150}} \approx 1.633\]
На удивление, реально попали в $(-\infty,\ 1.64)$, то есть принимаем гипотезу о равенстве. 
\subsection*{Задача 5}
Показать, что если $X\sim F(t)$, а $Y\sim F(t - \theta)$, то $H_0: \theta < 0 \Rightarrow EY < EX$\\
Распишем по определению:
\begin{equation*}
    \begin{aligned}
        EX &= \int\limits_{-\infty}^{+\infty} tf(t)\, dt\\
        EY &= \int\limits_{-\infty}^{+\infty} tf(t - \theta)\, dt = \left< \begin{aligned}
            z &= t - \theta\\
            dt &= dz
        \end{aligned} \right> = \int\limits_{-\infty}^{+\infty} (z + \theta) f(z)\, dz = \int\limits_{-\infty}^{+\infty} zf(z)\, dz + \theta \underset{=1}{\underbrace{\int\limits_{-\infty}^{+\infty} f(z)\, dz}} = EX + \theta\\
        EY &= EX + \theta\, \Big|_{H_0} < EX + 0 = EX
    \end{aligned}
\end{equation*}
Доказали.
\begin{center}
    \bf Семинар 14 марта
\end{center}
\subsection*{Справка}
$X_1,\dots,\ X_m \sim F(t - \mu)$ и $Y_1,\dots,\ Y_n \sim F\left( \frac{t - \mu}{\Delta} \right),\ \Delta > 0$\\
Проверяем $H_0: \Delta = 1$\\
Если нам известно, что\\
$X_1,\dots,\ X_m \sim N(m_1,\ \sigma_1^2)\\
Y_1,\dots,\ Y_n \sim N(m_2,\ \sigma_2^2)$\\
Проверяем гипотезу $H_0: \sigma_1^2 = \sigma_2^2 = \sigma^2$
\[T(x,\ y) = \frac{\frac{1}{m - 1} \sum_{i = 1}^{m}(x_i - \overline{X})^2}{ \frac{1}{n - 1} \sum_{i = 0}^{n} (y_i - \overline{Y})^2 } \sim F(m - 1,\ n - 1)\]
Против гипотезы $H_1: \sigma_1 < \sigma_2$:\\
Смотрим на отношение выборочных дисперсий $\frac{S_y^2}{S_x^2}\Big|_{H_0} \sim F(n - 1,\ m - 1)$. Критическая область справа.\\
Против гипотезы $H_2: \sigma_1 > \sigma_2$:\\
Смотрим на отношение выборочных дисперсий $\frac{S_x^2}{S_y^2}\Big|_{H_0} \sim F(m - 1,\ n - 1)$. Критическая область также справа (однако мы поменяли числитель со знаменателем).\\
Против гипотезы $H_3: \sigma_1 \neq \sigma_2$:\\
Также ставим большее в знаменатель, получаем соответствующее распределение Фишера. Критическая область также справа, границей будет квантиль фишера уровня $1 - \alpha/2$.
\subsection*{Задача}
$X_1,\dots,\ X_{29} \sim N(m_1,\ \sigma^2_1)\\
Y_1,\dots,\ Y_{16} \sim N(m_2,\ \sigma^2_2)\\
H_0: m_1 = m_2$ против $H_1: m_1 > m_2$.\\
Дано $\overline{X} = 2.5,\ \overline{Y} = 2.06,\ S_{x}^2 = 0.67,\ S_{y}^2 = 0.42 \Rightarrow \tilde S_x^2 = \frac{29}{28} S_x^2 = 0.694,\ \tilde S_y^2 = \frac{16}{15} S_y^2 = 0.448$\\
Нужно проверить гипотезу о том, что дисперсии одинаковы $H_0: \sigma_1 = \sigma_2 = \sigma$ против $H_1: \sigma_1 \neq \sigma_2$\\
Возьмём статистику $T(x,\ y) = \frac{\tilde S_x^2}{\tilde S_y^2} = \frac{0.694}{0.448} < t_{28,\ 15,\ 0.95} \Rightarrow$ принимаем $H_0$
\subsection*{Задача}
Есть станок, который производит детали с параметрами, распределёнными по неизвестному распределению с неизвестным (но одинаковым) матожиданием и дисперсией. Его наладили, нужно проверить гипотезу о том, что дисперсия уменьшилась.\\
До наладки было:
\[X = (52.4,\ 56.1,\ 48.6,\ 46.5,\ 46,\ 42.2,\ 48.8,\ 56.6,\ 59.8,\ 49.7,\ 51.6)\]
После наладки стало:
\[Y = (49.3,\ 47.7,\ 52.8,\ 48.3,\ 49.1,\ 46.4,\ 47,\ 52,\ 51.5,\ 51.2,\ 49.8)\]
Предполагаем, что $X \sim F(t - \mu),\ Y\sim F\left( \frac{y - \mu}{\Delta} \right)$\\
Проверяем гипотезу $H_0: \Delta = 1$ против $H_1: \Delta < 1$.\\
Применяем Ансари-Брейли.
$A_{11,\ 11} = 53,\ EA_{11,\ 11} = 66,\ \dev A_{11,\ 11} = 29.49$\\
$A^* = \frac{53 - 66}{\sqrt{29.49}} \approx -2.5$\\
Граница критической области находится в точке $Z_{0.05} = -1.64\Rightarrow$ попали в критическую область, то есть дисперсия действительно уменьшилась.
\begin{center}
    \bf ДЗ к семинару 21 марта
\end{center}
\subsection*{Задача 1}
Имеется таблица температур в двух городах за $n = m = 13$ лет.
\[\begin{tabular}{|c|c|c|}
    \hline
    Год & Саратов & Алатырь\\
    \hline
    1891 & $-19.2$ & $-21.8$\\
    \hline
    1892 & $-14.8$ & $-15.4$\\
    \hline
    1893 & $-19.6$ & $-20.8$\\
    \hline
    1894 & $-11.1$ & $-11.3$\\
    \hline
    1895 & $-9.4$ & $-11.6$\\
    \hline
    1896 & $-16.9$ & $-19.2$\\
    \hline
    1897 & $-13.7$ & $-13.0$\\
    \hline
    1899 & $-4.9$ & $-7.4$\\
    \hline
    1911 & $-13.9$ & $-15.1$\\
    \hline
    1912 & $-9.4$ & $-14.4$\\
    \hline
    1913 & $-8.3$ & $-11.1$\\
    \hline
    1914 & $-7.9$ & $-10.5$\\
    \hline
    1915 & $-5.3$ & $-7.2$\\
    \hline
\end{tabular}\]

Необходимо проверить равенство дисперсий температур в этих городах. Вообще про распределение мы ничего не знаем, поэтому стоило бы сразу использовать критерий Ансари-Бредли, но это требуется во второй задаче, то есть здесь, скорее всего, подразумевается критерий Фишера. Для его применения мы должны предположить, что температуры распределены нормально, например $N(m_1,\ \sigma_1^2),\ N(m_2,\ \sigma_2^2)$ для температуры Саратова и Алатыря соответственно.\\
\subsubsection*{Решение}
Нужно проверить гипотезу:
\[H_0: \sigma_1^2 = \sigma_2^2 = \sigma^2,\ \text{здесь $\sigma^2$ --- некое число, введённое для удобства обозначения}\]
Альтернативой будет $H_1: \sigma_1^2 \neq \sigma_2^2$
Запишем статистику $T$:
\[T(x,\ y) = \frac{\frac{1}{m - 1} \sum_{i = 1}^{m} {\left( x_i - \overline{X} \right)}^2}{\frac{1}{n - 1} \sum_{i = 1}^n {\left( y_i - \overline{Y} \right)}^2} \Bigg|_{H_0} \sim F(m - 1,\ n - 1)\]
Числителем здесь является $\tilde S_x^2$, знаменателем --- $\tilde S_y^2$. Посчитаем их:
\begin{equation*}
    \begin{aligned}
        \overline{X} &\approx -11.88 \\
        \tilde S_x^2 &= \frac{1}{12} \sum_{i}^{13} {\left(x_i - \overline{X}\right)}^2 \approx 23.99\\
        \overline{Y} &\approx -13.75\\
        \tilde S_y^2 &= \frac{1}{12} \sum_{i}^{13} {\left(y_i - \overline{Y}\right)}^2 \approx 21.76
    \end{aligned}
\end{equation*}
$\tilde S_x^2$ получилась больше, значит статистику мы не меняем (если бы $\tilde S_y^2$ оказался больше, то пришлось бы рассматривать $\frac{1}{T(x,\ y)}$).\\
Статистика должна иметь распределение $F(12,\ 12)$, квантиль уровня значимости $0.05$ для такого распределения равен $2.69$, а критическая область находится справа, то есть доверительный интервал: $(-\infty,\ 2.69)$.\\
Посчитаем значение нашей статистики:
\[T(x,\ y) = \frac{23.99}{21.76} \approx 1.1\]
Попали в доверительную область, значит принимаем $H_0$.
\subsubsection*{Ответ}
Дисперсии действительно можно считать одинаковыми
\subsection*{Задача 2}
Решить предыдущую задачу, пользуясь критерием Ансари-Бредли.
\subsubsection*{Решение}
Полагаем, что выборка $X_1,\dots,\ X_{13} \sim F(t - \mu)$ и выборка $Y_1,\dots,\ Y_{13} \sim F\left( \frac{t - \mu}{\Delta} \right),\ \Delta > 0$ и уровень доверия $\alpha = 0.05$\\
Критерий Ансари-Бредли требует $F(\mu) = 0.5$ (в условии задачи есть подсказка на этот счёт, нужно центрировать данные выборочной медианой).\\
Теперь нужно проверить гипотезу $H_0: \Delta = 1$ против $H_1: \Delta \neq 1$\\
Согласно критерию следует рассмотреть статистику:
\[A_{m,\ n} = \sum_{i = 1}^{m + n} \left( \frac{m + n + 1}{2} - \left| R_i - \frac{m + n + 1}{2} \right| \right)\]
С нашими числами будет выглядеть так:
\[A_{13,\ 13} = \sum_{i = 1}^{26} \left( \frac{27}{2} - \left| R_i - \frac{27}{2} \right| \right) = 83\]
Здесь $R_i$ --- ранг $X_i$ в объединённой выборке \big(в общем случае берём ту, распределение которой мы считаем $F(t - \mu)$\big).\\
На лекции говорилось, что есть таблица точных значений квантилей этой статистики при $n + m \leq 20$ (не наш случай), поэтому придётся апроксимировать гауссовским распределением:
\[A^* = \frac{A_{m,\ n} - EA_{m,\ n}}{\sqrt{\dev A_{m,\ n}}}\bigg|_{H_0} \sim N(0,\ 1)\]
Формулы для математического ожидания и дисперсии статистики известны:
\begin{equation*}
    \begin{aligned}
        EA_{m,\ n} &= \begin{cases}
            \frac{m(m + n + 2)}{4},\ & m + n \underset{2}{\equiv} 0\\
            \frac{m{(m + n + 1)}^2}{4(m + n)},\ & m + n \underset{2}{\equiv} 1\\
        \end{cases}\\
        \dev A_{m,\ n} &= \begin{cases}
            \frac{mn(m + n + 2)(m + n - 2)}{48(m + n - 1)},\ & m + n \underset{2}{\equiv} 0\\
            \frac{mn\left({(m + n)}^2 + 3\right)(m + n + 1)}{48 {(m + n)}^2},\   & m + n \underset{2}{\equiv} 1
        \end{cases}
    \end{aligned}
\end{equation*}
В нашем случае $m + n = 13 + 13 = 26 \underset{2}{\equiv} 0 \Rightarrow $ используем соответствующие формулы:
\begin{equation*}
    \begin{aligned}
        EA_{13,\ 13} &= \frac{13(13 + 13 + 2)}{4} = \frac{13\cdot 28}{4} = 13\cdot 7 = 91\\
        \dev A_{13,\ 13} &= \frac{(13\cdot 13)(13 + 13 + 2)(13 + 13 - 2)}{48(13 + 13 - 1)} = \frac{169\cdot 28 \cdot 24}{48\cdot 25} \approx 94.64
    \end{aligned}
\end{equation*}
Теперь можно посчитать $A^*$:
\[A^* = \frac{83 - 91}{\sqrt{94.64}} \approx -0.822\]
Доверительный интервал: $(Z_{0.025},\ Z_{0.975}) = (-1.96,\ 1.96)$. Статистика попала в доверительный интервал, значит принимается $H_0$
\subsubsection*{Ответ}
По этому критерию принимаем $\Delta = 1$, а $\Delta^2 = \frac{\dev Y}{\dev X}\Rightarrow \dev Y = \dev X$
\subsection*{Задача 3}
За последние 5 лет выборочная дисперсия доходности актива А составила 0.04, а выборочная дисперсия доходности актива Б составила 0.05. Есть ли основание утверждать (на уровне значимости 0.05), что вложения в актив А менее рискованны, чем вложения в актив Б? Предполагается, что доходности активов являются гауссовскими СВ.
\subsubsection*{Решение}
Полагаем, что $X \sim N(m_1,\ \sigma_x^2)$ --- выборка доходностей предприятия А, $Y$ --- выборка доходностей предприятия Б. Нам дано $S_x^2 = 0.04$ и $S_y^2 = 0.05$ (дана обычная выборочная дисперсия, про несмещённость не сказано).\\
Сформулируем гипотезы для проверки:
\[H_0: \sigma_x = \sigma_y\ \text{против}\ H_1: \sigma_x < \sigma_y\]
Если верна $H_0$, то вложения в любой из активов влекут за собой одинаковые риски. Принятие $H_1$ будет означать, что вкладываться в актив $A$ безопаснее.

Для всех известных критериев оценки дисперсии необходимо знать размер выборок. В нашем случае предположу $n = m = 5$ (доходность замерялась ежегодно).

В нашем случае мы сравниваем дисперсии гауссовских величин (в условии написано считать их таковыми), поэтому пользоваться будем критерием Фишера. В этом критерии необходима несмещённая выборочная дисперсия, поэтому стоит получить её из данной нам смещённой:
\begin{equation*}
    \begin{aligned}
        \tilde S^2_x &= \frac{n}{n - 1} S_x^2 = \frac{5}{4}\cdot 0.04 = 0.05\\
        \tilde S^2_y &= \frac{m}{m - 1} S_y^2 = \frac{5}{4}\cdot 0.05 = 0.0625
    \end{aligned}
\end{equation*}
$\tilde S^2_y$ больше, значит нужно рассматривать статистику:
\[T(x,\ y) = \frac{\tilde S^2_y}{\tilde S^2_x}\bigg|_{H_0} \sim F(m - 1,\ n - 1) \sim F(4,\ 4)\]
Подставив числа в статистику, получаем:
\[T(x,\ y) = \frac{5}{4}\]
Квантилем уровня $0.95$ для $F(4,\ 4)$ является 6.4, значит доверительным интервалом будет $(0,\ 6.4)$. Статистика попала в доверительный интервал, значит принимаем $H_0$.
\subsubsection*{Ответ}
Оснований утверждать о меньшем количестве риска нет.
\subsection*{Задача 4}
Два завода изготавливают электролампы одинакового типа. Из продукции завода №1 случайным образом выбрано 10 ламп, из продукции завода №2 --- 12 ламп.  Испытания по длительности горения ламп (в часах) следующие:
\begin{center}
    \begin{minipage}{0.8\textwidth}
        Для завода №1: 1243, 1238, 1253, 1243, 1254, 1260, 1251, 1246, 1255, 1237.\\
        Для завода №2: 1244, 1255, 1258, 1266, 1249, 1257, 1260, 1247, 1256, 1271, 1252, 1259.
   \end{minipage}
\end{center}
Проверьте гипотезу об однородности двух выборок, применяя критерий Колмогорова-Смирнова. (Указание. Квантиль уровня $1-0.049$ распределения статистики Колмогорова-Смирнова для выборок объёма 10 и 12 равна $\frac{33}{60}$).
\subsubsection*{Решение}
Будем работать с гипотезами:
\[H_0: \forall t\quad F(t) = G(t),\ \text{против}\ H_1: \exists t\quad F(t) \neq G(t)\]
Статистикой будет:
\[D_{m,\ n} = \max\limits_{1 \leq i \leq m + n} \left| \hat F_{m}(z_i) - \hat G_{n}(z_i) \right|\]
В этой статистике $z_i$ --- $i$-ый элемент объединённой выборки. Точный квантиль этой статистики дан в указании к заданию и равен $\frac{33}{60}$. То есть доверительным интервалом будет $\left(0,\ \frac{33}{60}\right) = \left(0,\ 0.55\right)$.

Для применения критерия Колмогорова-Смирнова необходимо построить эмпирические функции распределения для обоих заводов. Для этого стоит упорядочить обе выборки:
\begin{center}
    \begin{minipage}{0.8\textwidth}
        Для завода №1: 1237, 1238, 1243, 1243, 1246, 1251, 1253, 1254, 1255, 1260.\\
        Для завода №2: 1244, 1247, 1249, 1252, 1255, 1256, 1257, 1258, 1259, 1260, 1266, 1271.
   \end{minipage}
\end{center}
А теперь рисуем:
\[\begin{tikzpicture}
    \begin{axis}[
        xlabel={$x$},
        ylabel={Эмпирическая ФР},
        xmin=1234, xmax=1275,
        ymin=0, ymax=1.05,
        xtick={1235, 1240, 1245, 1250, 1255, 1260, 1265, 1270, 1275},
        ytick={0,0.1,0.3,0.5,0.7,0.9,0.2,0.4,0.6,0.8,1.0},
        minor x tick num=4,
        minor grid style={dashed, gray!50},
        legend pos=south east,
        width=18cm,
        height=8cm,
        grid=both,
    ]

    \addplot[blue, const plot, jump mark left, very thick] coordinates {
        (1234, 0)
        (1237, 0.1)
        (1238, 0.2)
        (1243, 0.4)
        (1246, 0.5)
        (1251, 0.6)
        (1253, 0.7)
        (1254, 0.8)
        (1255, 0.9)
        (1260, 1.0)
        (1275, 1)
        };
        \addlegendentry{Завод №1: $\hat F_{10}(x)$}
        
        \addplot[green, const plot, jump mark left, very thick] coordinates {
        (1234, 0)
        (1244, 0.0833)
        (1247, 0.1667)
        (1249, 0.25)
        (1252, 0.3333)
        (1255, 0.4167)
        (1256, 0.5)
        (1257, 0.5833)
        (1258, 0.6667)
        (1259, 0.75)
        (1260, 0.8333)
        (1266, 0.9167)
        (1271, 1.0)
        (1275, 1)
    };
    \addlegendentry{Завод №2: $\hat G_{12}(x)$}

    \end{axis}
\end{tikzpicture}\]
По графику видно, что наибольшее различие между функциями находится в точке $x = 1255$:
\[D_{10,\ 12} = \left| \frac{9}{10} - \frac{5}{12} \right| = 0.483\]
Попали в доверительный интервал, значит принимаем $H_0$ и говорим, что распределения одинаковы.
\subsubsection*{Ответ}
Гипотеза однородности принимается по критерию Колмогорова-Смирнова.
\subsection*{Задача 5}
Пусть выборка $X_1,\dots,\ X_n$ порождена случайной величиной $X$ с непрерывным распределением $F(t - \mu)$, а выборка $Y_1,\dots,\ Y_n$ --- случаной величиной $Y$ с распределением $F\left( \frac{t - \mu}{\Delta} \right),\ \Delta > 0$. Предполагается, что $\dev X < \infty$ и выполняется равенство $\displaystyle \int\limits_{-\infty}^{+\infty} t f(t)\, dt = 0$. Показать, что из справедливости $H_0: \Delta < 1$ следует неравенство $\dev X > \dev Y$.
\subsubsection*{Решение}
Чтобы посчитать дисперсии, нужно сначала посчитать математические ожидания, а для подсчёта матожиданий нужны функции плотностей вероятности:
\begin{equation*}
    \begin{aligned}
        f_{x}(t) &= \left( F(t - \mu) \right)' = f(t - \mu)\cdot\left( t - \mu \right)' = f(t - \mu)\\
        f_{y}(t) &= \left( F\left(\frac{t - \mu}{\Delta}\right) \right)' = f\left(\frac{t - \mu}{\Delta}\right)\cdot\left(\frac{t - \mu}{\Delta}\right)' = \frac{1}{\Delta}\cdot f\left(\frac{t - \mu}{\Delta}\right)\\
    \end{aligned}
\end{equation*}
Теперь можно считать математические ожидания:
\begin{equation*}
    \begin{aligned}
        EX &= \int\limits_{-\infty}^{+\infty} tf_{x}(t)\, dt = \int\limits_{-\infty}^{+\infty} t f(t - \mu)\, dt = \left< \begin{aligned}
            a &= t - \mu\\
            da &= dt
        \end{aligned} \right> = \int\limits_{-\oo}^{+\oo} (a + \mu) f(a)\, da =\\
        &= \underset{=0,\ \text{по усл.}}{\underbrace{\int\limits_{-\oo}^{+\oo} af(a)\, da}} + \mu \underset{=1}{\underbrace{\int\limits_{-\oo}^{+\oo} f(a)\, da}} = \mu\\
        EY &= \int\limits_{-\infty}^{+\infty} tf_{y}(t)\, dt \int\limits_{-\infty}^{+\infty} \frac{t}{\Delta} f\left( \frac{t - \mu}{\Delta} \right)\, dt = \left< \begin{aligned}
            z &= \frac{t - \mu}{\Delta}\\
            dz &= \frac{dt}{\Delta}
        \end{aligned} \right> = \int\limits_{-\infty}^{+\infty} \frac{\Delta \cdot z + \mu}{\Delta} f(z)\Delta \, dz =\\
        &= \Delta\int\limits_{-\infty}^{+\infty} zf(z)\, dz + \mu\int\limits_{-\infty}^{+\infty} f(z)\, dz = \mu
    \end{aligned}
\end{equation*}
И, наконец, дисперсии:
\begin{equation*}
    \begin{aligned}
        \dev X &= \int\limits_{-\infty}^{+\infty} {\left(t - EX\right)}^2 f_{x}(t)\, dt = \int\limits_{-\infty}^{+\infty} {\left( t - \mu \right)}^2 f(t - \mu)\, dt = \left< \begin{aligned}
            a &= t - \mu\\
            da &= dt
        \end{aligned} \right> = \int\limits_{-\infty}^{+\infty} a^2 f(a)\, da\\
        \dev Y &= \int\limits_{-\infty}^{+\infty} {\left( t - EY\right)}^2 f_{y}(t)\, dt = \int\limits_{-\infty}^{+\infty} {\left( t - \mu \right)}^2 \frac{1}{\Delta} f\left( \frac{t - \mu}{\Delta} \right)\, dt = \int\limits_{-\infty}^{+\infty} \Delta^2{\left( \frac{t - \mu}{\Delta} \right)}^2 f\left( \frac{t - \mu}{\Delta} \right) \frac{dt}{\Delta} = \\
        &= \left< \begin{aligned}
            z &= \frac{t - \mu}{\Delta}\\
            dz &= \frac{dt}{\Delta}
        \end{aligned} \right> = \int\limits_{-\infty}^{+\infty} \Delta^2 z^2 f(z)\, dz = \Delta^2 \int\limits_{-\infty}^{+\infty}  z^2 f(z)\, dz = \Delta^2 \dev X
    \end{aligned}
\end{equation*}
Получаем соотношение: $\dev Y = \Delta^2 \dev X\Rightarrow \dfrac{\dev Y}{\dev X} = \Delta^2\Big|_{H_0} < 1\Rightarrow \dfrac{\dev Y}{\dev X} < 1 \Rightarrow \dev X > \dev Y$, ч. т. д.
\begin{center}
    \bf Семинар 21 марта
\end{center}
\subsection*{Задача 1}
Случанйым образом выбраны 14 человек одного возраста, проживающих в одном городе, но имеющих различный уровень образования. Их опрашивли про доход:\\
Неполное среднее: 37, 19, 26, 42 $\Rightarrow\overline{X}_{\bullet\, 1} = 31$\\
Среднее специальное: 47, 39, 52, 41, 51 $\Rightarrow\overline{X}_{\bullet\, 2} = 46$\\
Высшее: 64, 78, 59, 71, 63 $\Rightarrow\overline{X}_{\bullet\, 3} = 67$\\
$\overline{X}_N = 49.2$\\

Проверить гипотезу о том, что средние доходы в группах одинаковые, против альтернативы о том, что доходы неодинаковые.\\
В этой задаче фактор: образование. Уровни фактора: неполное среднее, среднее специальное, высшее.\\
\subsubsection*{Первый случай}
\begin{equation*}
    \begin{aligned}
        x_{i\, j} = \theta + \tau_j + \varepsilon_{i\, j},\ j = 1,\ 2,\ 3
    \end{aligned}
\end{equation*}
Предполагаем, что $\varepsilon_{i\, j} \sim N(0,\ \sigma^2)$\\
Проверяем $H_0: \tau_1 = \tau_2 = \tau_3 = 0$ против $H_1: \exists j: \tau_j \neq 0$
\subsubsection*{Решение}
$SS_{\text{ур. ф.}} = \sum_{j = 1}^{n} n_j{\left( \overline{X}_j - \overline{X}_N \right)}^2 = 4\cdot 18.2^2 + 5\cdot 3.2^2 + 5\cdot 18.2^2 \approx 2\,960$\\
$SS_{\text{случ.}} = \sum_{j = 1}^{k} \sum_{i = 1}^{n_j} {\left( x_{i\, j} - \overline{X}_{\bullet\, j} \right)}^2 = 12^2 + 6^2 + 5^2 + 11^2 + 5^2 + 6^2 + 1^2 + 7^2 + 5^2 + 3^2 + 11^2 + 8^2 + 4^2 + 4^2 = 688$\\
Берём статистику:
\[\hat f = \frac{\frac{1}{k - 1} SS_{\text{ур. ф.}}}{ \frac{1}{N - k} SS_{\text{случ.}}} = \frac{\frac{3032}{2}}{\frac{688}{11}} = \frac{1480}{62.56}\approx 23\]
При справедливости $H_0$:
\[\hat f \Big|_{H_0} \sim F(2,\ 11)\]
Квантиль $F_{0.05,\ 2,\ 11} = 3.98$, критическая область справа. Статистика попала в критическую область, значит принимается альтернатива.
\subsubsection*{Второй случай}
Не пользуемся предположение о гауссовости распределения. Будем применять критерий Краскела-Уоллиса.
\[x_{i\, j} = \theta_j + \varepsilon_{i\, j}\]
Необходимо посчитать средний ранг для каждого уровня фактора:\\
Неполное среднее: $r_{\bullet\, 1} = 3$\\
Среднее специальное: $r_{\bullet\, 2} = 6.6$\\
Высшее: $r_{\bullet\, 3} = 12$\\
Статистика будет выглядеть вот так:
\[H = \frac{12}{N(N + 1)} \sum_{j = 1}^{3} n_j {\left( \overline{r}_{\bullet\, j} - \frac{N + 1}{2} \right)}^2 = \frac{12}{14\cdot 15}\left( 4\cdot 4.5^2 + 5\cdot 0.9^2 + 5\cdot 4.5^2 \right) \approx 10.65\]
При справедливости $H_0$ для этой статистики справедливо:
\[H\Big|_{H_0} \sim \chi^2(2)\]
Квантиль $\chi^2_{2\, 0.95} = 5.99$, значит попали в критическую область, значит здесь тоже принимаем $H_2$.
\subsubsection*{Третий случай}
Проверяем гипотезу $H_0: \theta_1 = \theta_2 = \theta_3$ против $H_1: \theta_1 \leq \theta_2 \leq \theta_3$, где хотя бы одно неравенство строгое.\\
Введём функции:
\[\varphi(y,\ z) = \begin{cases}
    1,\ & y < z\\
    0.5,\ & y = z\\
    0,\ & y > z
\end{cases},\quad U_{l,\ m} = \sum_{i = 1}^{n_l} \sum_{j = 1}^{n_m} \varphi(x_{i\, l},\ x_{i\, m})\]
Возьмём статистику:
\[J = \sum_{1 \leq L < m \leq k} U_{l,\ m}\]
$U_{1,\ 2} = 5 + 5 + 5 + 3 = 18\\
U_{1,\ 3} = 5\cdot 4 = 20\\
U_{2,\ 3} = 5\cdot 5 = 25$, теперь можем посчитать статистику $J = 18 + 20 + 25 = 63$\\
Тогда справедиво:
\[J^* = \frac{J - EJ}{\sqrt{\dev J}}\bigg|_{H_0} \sim N(0,\ 1)\]
Из лекции берём формулы для математического ожидания и дисперсии:
\begin{equation*}
    \begin{aligned}
        EJ &= \frac{1}{4} \left( N^2 - \sum_{i = 1}^{k} n_i^2 \right) = 32.5\\
        \dev J &= \frac{1}{72} \left( N^2 (2N + 3) - \sum_{i = 1}^{k} n_j^2 (2n_j + 3) \right) \approx 73 \Rightarrow \sqrt{\dev J} \approx 8.5
    \end{aligned}
\end{equation*}
Тогда значение центрированной и нормированной статистики:
\[J^* = \frac{63 - 32.5}{8.5} \approx 3.6\]
Квантиль $Z_{0.95} = 1.64$, то есть статистика попала в критическую область. 

\begin{center}
    \bf ДЗ к 4 апреля
\end{center}
\subsection*{Задача 1}
Изучается влияние денежного стимулирования на производительность труда. Шести однородным группам, по 5 человек, раздали задачи одинаковой сложности. Задачи были выданы каждому члену группы независимо от остальных. Группы различаются только по денежному вознаграждению за каждую решенную задачу. Величина вознаграждения зависит от номера группы: чем больше номер группы, тем больше вознаграждение. Каждой группе известна цена вознаграждения за решенную задачу. В таблице представлено количество решенных задач каждым членом группы.
\[\begin{tabular}{|c|c|c|c|c|c|}
    \hline
    Группа 1 & Группа 2 & Группа 3 & Группа 4 & Группа 5 & Группа 6\\
    \hline
    10 & 8  & 12 & 12 & 24 & 19\\
    \hline
    11 & 10 & 17 & 15 & 16 & 18\\
    \hline
    9  & 16 & 14 & 16 & 22 & 27\\
    \hline
    13 & 13 & 9  & 16 & 18 & 25\\
    \hline
    7  & 12 & 16 & 19 & 20 & 24\\
    \hline
\end{tabular}\]
Влияет ли вознаграждение на количество решенных задач?\\
Решите данную задачу, предполагая, что:
\begin{enumerate}
    \item[а)] все наблюдения имеют нормальное распределение с одинаковыми дисперсиями;
    \item[б)] наблюдения имеют некоторые неизвестные непрерывные распределения, которые могут различаться только математическим ожиданием;
    \item[в)] наблюдения имеют некоторые неизвестные непрерывные распределения, которые могут различаться только математическим ожиданием, и имеется априорное предположение о том, что с ростом вознаграждения растёт количество решённых задач.
\end{enumerate}
\subsubsection*{Решение а)}
В этом случае можно воспользоваться критерием Фишера.\\
Полагаем, что каждое наблюдение $x_{i\, j}$ (числа в таблице выше) может быть представлено в виде:
\[x_{i\, j} = \theta + \tau_{j} + \varepsilon_{i\, j} \]
Где:
\begin{equation*}
    \begin{aligned}
        &\theta - \text{некоторое неизвестное общее среднее}\\
        &\tau_{j} - \text{отклонение, зависящее от фактора}\\
        &\varepsilon_{i\, j} - \text{случайное отклонение}
    \end{aligned}
\end{equation*}
Полагаем, что $\varepsilon_{i\, j} \sim N(0,\ \sigma^2)$ (дисперсия везде одинаковая).\\
Критерий Фишера проверяет гипотезу $H_0: \tau_1 = \dots = \tau_k = 0$ против $H_1: \exists i: \tau_{i} \neq 0$. В нашем случае $k = 6$, $N = 30$ (общее количество испытаний со всеми уровнями фактора).\\
Теперь для подсчёта статистики необходимо подсчитать $SS_{\text{случ.}}$ и $SS_{\text{ур. ф.}}$:
\begin{equation*}
    \begin{aligned}
        & \frac{SS_{\text{случ.}}}{\sigma^2} = \sum_{j = 1}^{k} \sum_{i = 1}^{n_j} {\left( \frac{x_{i\, j} - \overline{X}_{\bullet\, j}}{\sigma} \right)}^2\\
        & \frac{SS_{\text{ур. ф.}}}{\sigma^2} = \sum_{j = 1}^k n_j {\left( \frac{\overline{X}_{\bullet\, j} - \overline{X}_N}{\sigma} \right)}^2
    \end{aligned}
\end{equation*}
Здесь:
\begin{equation*}
    \begin{aligned}
        & \overline{X}_N = \frac{1}{N} \sum_{j = 1}^{k} \sum_{i = 1}^{n_j} x_{i\, j} = 15.6\\
        & \overline{X}_{\bullet\, 1} = \frac{1}{n_1}\sum_{i = 1}^{n_1} x_{i\, 1} = 10\\
        & \overline{X}_{\bullet\, 2} = \frac{1}{n_2}\sum_{i = 1}^{n_2} x_{i\, 2} = 11.8\\
        & \overline{X}_{\bullet\, 3} = \frac{1}{n_3}\sum_{i = 1}^{n_3} x_{i\, 3} = 13.6\\
        & \overline{X}_{\bullet\, 4} = \frac{1}{n_1}\sum_{i = 1}^{n_4} x_{i\, 4} = 15.6\\
        & \overline{X}_{\bullet\, 5} = \frac{1}{n_5}\sum_{i = 1}^{n_5} x_{i\, 5} = 20\\
        & \overline{X}_{\bullet\, 6} = \frac{1}{n_6}\sum_{i = 1}^{n_6} x_{i\, 6} = 22.6
    \end{aligned}
\end{equation*}
Теперь можно подставить ($\sigma$ здесь не нужна):
\begin{equation*}
    \begin{aligned}
        & SS_{\text{случ.}} = \sum_{j = 1}^{k} \sum_{i = 1}^{n_j} {\left( x_{i\, j} - \overline{X}_{\bullet\, j} \right)}^2 = 224.4\\
        & SS_{\text{ур. ф.}} = \sum_{j = 1}^k n_j {\left( \overline{X}_{\bullet\, j} - \overline{X}_N \right)}^2 = 590.8
    \end{aligned}
\end{equation*}
$SS_{\text{ур. ф.}}$ оказалось больше, тогда должно быть справедливо:
\[\frac{N - k}{k - 1} \cdot \frac{SS_{\text{ур. ф.}}}{SS_{\text{случ.}}} \sim F(k - 1,\ N - k) \sim F(5,\ 24)\]
Нужный нам квантиль уровня $0.95$ этого распределения равен примерно $2.621$, наша статистика же имеет значение:
\[\frac{24}{5}\cdot \frac{590.8}{224.4} \approx 12.6\]
Критической областью здесь будет $(2.621,\ +\oo)$, то есть уверенно находимся в критической областью, значит $H_0$ опровергается.
\subsubsection*{Решение задачи 2}
Сразу же разберёмся с контрастами в этой задаче. Нам нужно построить три оценки для контрастов:
\begin{equation*}
    \begin{aligned}
        \gamma_1 = \theta_2 - \theta_1\Rightarrow c_2 = 1,\ c_1 = -1\\
        \gamma_2 = \theta_4 - \theta_1\Rightarrow c_4 = 1,\ c_1 = -1\\
        \gamma_3 = \theta_6 - \theta_1\Rightarrow c_6 = 1,\ c_1 = -1
    \end{aligned}
\end{equation*}
На лекции давалась следующая оценка контраста:
\[\hat \gamma = \sum_{j = 1}^{k} c_j \cdot \overline{X}_{\bullet\, j}\]
То есть в нашем случае:
\begin{equation*}
    \begin{aligned}
        &\hat \gamma_1 = \overline{X}_{\bullet\, 2} - \overline{X}_{\bullet\, 1} = 1.8\\
        &\hat \gamma_2 = \overline{X}_{\bullet\, 4} - \overline{X}_{\bullet\, 1} = 5.6\\
        &\hat \gamma_3 = \overline{X}_{\bullet\, 6} - \overline{X}_{\bullet\, 1} = 12.6
    \end{aligned}
\end{equation*}
Также из лекции возьмём:
\[\hat \gamma \sim N\left( \gamma,\ \sigma^2 \sum_{j = 1}^{k} \frac{c_j^2}{n_j} \right)\]
Замечание: индекс здесь опущен, так как это верно для всех подобных оценок.\\
В нашем случае $\displaystyle \sum_{j = 1}^{k} \frac{c_j^2}{n_j} = \frac{2}{6}$. Строить доверительные интервалы гауссовских величин мы умеем:
\[\frac{\hat \gamma - \gamma}{\hat \sigma \sqrt{\sum_{j = 1}^{k} \frac{c_j^2}{n_j}}} = \frac{\hat \gamma - \gamma}{\hat \sigma \sqrt{\frac{2}{6}}} \sim t(N - k) \sim t(24)\]
Здесь $\hat \sigma = \frac{1}{N - k} SS_{\text{случ.}} = \frac{224.4}{24} \approx 9.35$.\\
Теперь можно построить интервал в общем виде:
\[P\left(\hat \gamma -  t_{0.975,\ 24} \hat \sigma\sqrt{\sum_{j = 1}^{k} \frac{c_j^2}{n_j}} < \gamma < \hat \gamma +  t_{0.975,\ 24} \hat \sigma\sqrt{\sum_{j = 1}^{k} \frac{c_j^2}{n_j}}\right) = 0.95\]
Подсмотрели $t_{0.975,\ 24} \approx 2.064$.\\
Посчитаем интервалы для каждого контраста:
\begin{enumerate}
    \item Для $\gamma_1$:
    \[P\left( -9.342 < \gamma < 12.942 \right) = 0.95\]
    \item Для $\gamma_2$:
    \[P\left( -5.542 < \gamma < 16.742 \right) = 0.95\]
    \item Для $\gamma_3$:
    \[P\left( 1.458 < \gamma < 23.742 \right) = 0.95\]
\end{enumerate}
\subsubsection*{Решение б)}
Знаем только, что функции распределения величин непрерывны и могут различаться только матожиданием. Значит будем применять критерий Краскела-Уоллиса.\\
Проверяем гипотезу $H_0: \theta_1 = \dots = \theta_k = \theta$ против $H_1: \exists i: \theta_i \neq \theta$\\
Нам понадобится таблица рангов:
\[\begin{tabular}{|c|c|c|c|c|c|}
    \hline
    Группа 1 & Группа 2 & Группа 3 & Группа 4 & Группа 5 & Группа 6\\
    \hline
5.5 & 2 & 9 & 9 & 27.5 & 23.5\\
\hline
7 & 5.5 & 20 & 14 & 17 & 21.5\\
\hline
3.5 & 17 & 13 & 17 & 26 & 30\\
\hline
11.5 & 11.5 & 3.5 & 17 & 21.5 & 29\\
\hline
1 & 9 & 17 & 23.5 & 25 & 27.5\\
\hline
\end{tabular}\]
Пользуясь этими рангами, составляем статистику:
\[H = \frac{12}{N(N + 1)} \sum_{j = 1}^{k} n_j {\left( \overline{r}_{\bullet\, j} - \frac{N + 1}{2}\right)}^2\]
Посчитаем средние ранги по столбцам:
\begin{equation*}
    \begin{aligned}
        & \overline{r}_{\bullet\, 1} = \frac{1}{n_1} \sum_{i = 1}^{n_1} r_{i\, 1} = 5.7\\
        & \overline{r}_{\bullet\, 2} = \frac{1}{n_2} \sum_{i = 1}^{n_2} r_{i\, 2} = 9\\
        & \overline{r}_{\bullet\, 3} = \frac{1}{n_3} \sum_{i = 1}^{n_3} r_{i\, 3} = 12.5\\
        & \overline{r}_{\bullet\, 4} = \frac{1}{n_4} \sum_{i = 1}^{n_4} r_{i\, 4} = 16.1\\
        & \overline{r}_{\bullet\, 5} = \frac{1}{n_5} \sum_{i = 1}^{n_5} r_{i\, 5} = 23.4\\
        & \overline{r}_{\bullet\, 6} = \frac{1}{n_6} \sum_{i = 1}^{n_6} r_{i\, 6} = 26.3\\
    \end{aligned}
\end{equation*}
Подставим подсчитанные значения в статистику и получим:
\[H \approx 21.077\]
При справедливости $H_0$ должно выполняться:
\[H \sim \chi^2 (k - 1) \sim \chi^2(5)\]
Нужный нам квантиль $\chi^2_{0.95,\ 5} \approx 11.07$
Критическая область справа, статистика туда попала, значит в $H_0$ мы не верим.
\subsubsection*{Решение в)}
В этом случае будем пользоваться критерием Джонкхиера.\\
Проверяем гипотезу $H_0: \theta_1 = \dots = \theta_k = \theta$ против альтернативы $H_1: \theta_1 \leq \dots \theta_k$, где хотя бы одно неравенство строгое.\\
Для подсчёта статистики понадобится ввести функцию:
\[\varphi(y,\ z) = \begin{cases}
    1,\ & y < z\\
    0.5,\ & y = z\\
    0,\ & y > z
\end{cases}\]
С её помощью зададим ещё одну функцию:
\[U_{l,\ m} = \sum_{ i = 1}^{n_l} \sum_{j = 1}^{n_m} \varphi\left( x_{i\, j},\ x_{j\, m} \right)\]
Статистика в данном критерии выглядит следующим образом:
\[J = \sum_{1 \leq l < m \leq k} U_{l,\ m}\]
Посчитаем функции $U$:
\begin{equation*}
    \begin{aligned}
        & U_{0\, 1} = 17.0\\
        & U_{0\, 2} = 20.5\\
        & U_{1\, 2} = 17.0\\
        & U_{0\, 3} = 24\\
        & U_{1\, 3} = 20.5\\
        & U_{2\, 3} = 16.5\\
        & U_{0\, 4} = 25\\
        & U_{1\, 4} = 24.5\\
        & U_{2\, 4} = 23.5\\
        & U_{3\, 4} = 22.0\\
        & U_{0\, 5} = 25\\
        & U_{1\, 5} = 25\\
        & U_{2\, 5} = 25\\
        & U_{3\, 5} = 23.5\\
        & U_{4\, 5} = 18.0\\
    \end{aligned}
\end{equation*}
Сумма этого добра будет статистикой $J$:
\[J = 327\]
Из лекции известно, что при $\min(n_1,\dots,\ n_k) \to \oo$ справедливо:
\[J^* = \frac{J - EJ}{\sqrt{\dev J}} \sim N(0,\ 1)\]
Поскольку не все люди могут пальцами одной руки посчитать 5, мы будем пользоваться этой апроксимацией. Из лекции знаем следующие прикольные факты:
\[\begin{cases}
    EJ = \frac{1}{4} {\left( N^2 - \sum\limits_{i = 1}^{k} n_i^2 \right)} = 187.5\\
    \dev J = \frac{1}{72} \left( N^2 (2N + 3) - \sum\limits_{i = 1}^{k} n_i^2(2n_i + 3) \right) \approx 760.417\Rightarrow \sqrt{\dev J} \approx 27.576
\end{cases}\] 
Подставляем:
\[ J^* = \frac{327 - 187.5}{27.576} \approx 5.059 \]
Критическая область здесь $(Z_{0.95},\ +\infty)$, знаем $Z_{0.95} \approx 1.64$, то есть находимся глубоко в критической области и снова не верим в $H_0$. 

\begin{center}
    \bf Семинар 4 апреля
\end{center}
Работаем с величинами, измерямые в номинальных шкалах.
\[\begin{tabular}{|c|c|c|c|c|}
    \hline
    $A\setminus B$ & $B_1$ & $\dots$ & $B_k$ & \\
    \hline
    $A_1$ & $n_{1\, 1}$ & $\dots$ & $n_{1\, k}$ & $n_{1\, \bullet}$\\
    \hline
    $\vdots$ & $\vdots$ & $n_{i\, j}$ & $\vdots$ & $n_{i\, \bullet}$ \\
    \hline
    $A_m$ & $n_{m\, 1}$ & $\dots$ & $n_{m\, k}$ & $n_{m\, \bullet}$\\
    \hline
    & $n_{\bullet\, 1}$ & $\dots $ & $n_{\bullet\, k}$ & $n$\\
    \hline
\end{tabular}\]
Проверяем $H_0: P(A = A_i,\ B = B_j) = p_{i\, \bullet} \cdot p_{j\, \bullet}$ против $H_1: P(\dots) \neq p_{i\, \bullet} \cdot p_{\bullet\, j}$.\\
Используется статистика
\[ \hat \chi^2 = \sum_{i = 1}^{m} \sum_{j = 1}^{k} \frac{n {\left( n_{i\, j} - \frac{n_{i\, \bullet} n_{\bullet\, j}}{n} \right)}^2 }{n_{i\, \bullet}\, n_{\bullet\, j}} \]
Для этой статистики справедливо:
\[ \chi^2 \Big|_{H_0} \sim \chi^2 \big( (k - 1)(m - 1) \big) \]
\subsection*{Задача}
Исследуем влияние вакцины от холеры. Было 1630 привитых человек, из них 5 человек заболело. 1033 непривитых человека, из них заболело 11. Получаем два признака $A$ --- привит (П) / не привит (НП) и $B$ --- заболел (Б) / не заболел (НБ). Составим табилцу:
\[\begin{tabular}{|c|c|c|c|}
    \hline
    $\text{П} \setminus \text{Б}$ & Б & НБ & \\
    \hline
    П &  5 & 1625 & 1630\\
    \hline
    НП & 11 & 1022 & 1033\\
    \hline
     & 16 & 2647 & 2663\\
    \hline
\end{tabular}\]
Гипотезы $H_0,\ H_1$ написаны выше.
Для таблиц $2\times 2$ есть простая формула:
\[\hat \chi^2 = \frac{n{(n_{1\, 1}n_{2\, 2} - n_{1\, 2} n_{2\, 1})}^2}{n_{1\, \bullet}\, n_{2\, \bullet}\, n_{\bullet\, 1}\, n_{\bullet\, 2}} \approx 6.02\]
Знаем, что должно выполняться:
\[\hat \chi^2 \Big|_{H_0} \sim \chi^2(1)\]
Разделять доверительную и критическую область здесь будет квантиль $\chi^2_{0.95,\ 1} \approx 3.84$. То есть статистика попала в критическую область.\\
Посчитаем коэффициенты контингенции и Юла соответственно:
\[\begin{cases}
    \Phi = \frac{n_{1\, 1}n_{2\, 2} - n_{1\, 2} n_{2\, 1}}{\sqrt{n_{1\, \bullet}\, n_{2\, \bullet}\, n_{\bullet\, 1}\, n_{\bullet\, 2}}} \approx -0.048\\
    Q = \frac{n_{1\, 1}n_{2\, 2} - n_{1\, 2} n_{2\, 1}}{n_{1\, 1}n_{2\, 2} + n_{1\, 2} n_{2\, 1}} \approx -0.56
\end{cases}\]
\subsection*{Задача}
Собраны данные по ангийским ВУЗам. Рассматриваются несколько факультетов и пол студента. Имеется ли зависимость между полом студента и выбранной специализацией?
\[\begin{tabular}{|c|c|c|c|}
    \hline
    $\text{спец.} \setminus \text{пол}$ & М & Ж & \\
    \hline
    Искусствоведение &  197 & 223 & 420\\
    \hline
    Естественные науки & 168 & 92 & 260\\
    \hline
    Соц.-эк. науки & 115 & 105 & 220\\
    \hline
     & 480 & 420 & 900\\
    \hline
\end{tabular}\]
Гипотезы не меняются. Строим статистику:
\[\hat \chi^2 = \sum_{i = 1}^{3} \sum_{j = 1}^{2} \frac{n { \left( n_{i\, j} - \frac{n_{i\, \bullet} n_{\bullet\, j}}{n} \right) }^2}{n_{i\, \bullet} n_{\bullet\, j}} \approx 20.36\]
Разделять области будет $\chi^2_{0.95,\ 2} \approx 5.99$, то есть попали в критическую область.\\
Посчитаем коэффициенты Пирсона и Краммера соответственно:
\[\begin{cases}
    P = \sqrt{\frac{\hat \chi^2}{ \hat \chi^2 + n}}\approx 0.148\\
    C = \sqrt{\frac{\hat \chi^2}{n\cdot \min\{ (k - 1),\ (m - 1) \}}} \approx 0.15
\end{cases}\]
То есть зависимость между полом и направлением имеется, но она не очень большая.
\end{document}