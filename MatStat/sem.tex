\documentclass[12pt, a4paper]{article}
\usepackage[utf8]{inputenc}
\usepackage[T2A]{fontenc}
\usepackage[russian]{babel}
\usepackage{amsfonts}
\usepackage{amsmath}
\usepackage{mathabx}
\usepackage{graphicx}
\usepackage{hyperref}
\usepackage{listings}
\usepackage{color}
\usepackage[margin=0.25in]{geometry}
\usepackage{pgfplots}
\pgfplotsset{width=10cm,compat=1.9}
\usepackage{nicefrac}
\graphicspath{../Images}
\usepackage{tikz}
\usepackage{cancel}

\usepackage{fontspec}
\setmainfont{Times New Roman}

\title{Математическая статистика.}
\author{Андрей Тищенко \href{https://t.me/AndrewTGk}{@AndrewTGk}}
\date{2024/2025}

\newcommand{\dev}{\mathcal{D}}
\newcommand{\cov}{\operatorname{cov}}

\begin{document}
\maketitle
\begin{center}
    Семинар 10 января
\end{center}
\subsection*{Задача 1}
$x_1,\dots,\ x_n\sim F_{\xi}(x)$, найти функцию распределения для $X_{(n)},\ X_{(1)}$\\
$F_{X_{(n)}}(x) = P(X_{(n)} \leq x) = P(X_{(1)} \leq x,\dots,\ X_{(n)} \leq x) = P(X_1 \leq x,\dots,\ X_n \leq x) =\\
=P(X_1\leq x) \dots P(X_n \leq x) =(F_{\xi}(x))^{n}$\\
$F_{X_{(1)}}(x) = P(X_{(1)} \leq x) = 1 - P(X_{(1)} > x) = 1 - P(X_{(1)} > x,\dots,\ X_{(n)} > x) =\\
= 1 - P(X_1 > x,\dots,\ X_n > x) = 1 - P(X_1 > x)\dots P(X_n > x) = 1 - \big(1 - F_{\xi}(x)\big)^n$
\subsection*{Задача 2}
$x_1,\dots,\ x_n \sim R(0,\ 1)$. Найти $EX_{(n)},\ EX_{(1)}$.\\
$F_{X_{(n)}}(x) = \big( F_{\xi}(x) \big)^n$\\
$f_{X_{(n)}}(x) = \big( F_{X_{(n)}}(x) \big)' = n\big(F_{\xi}(x)\big)^{n - 1}\cdot f_{\xi}(x)$\\
$F_{\xi}(x) = \begin{cases}
    0,\ x < 0\\
    x,\ x\in [0,\ 1]\\
    1,\ x > 1
\end{cases}$\\
Подставим в предыдущее уравнение:
\[f_{X_{(n)}} = \begin{cases}
    0,\ x < 0\\
    nx^{n - 1},\ x\in [0,\ 1]\\
    0,\ x > 1
\end{cases}\]
$EX_{(n)} = \displaystyle\int\limits_{-\infty}^{+\infty} xf_{X_{(n)}}(x)\, dx = \int\limits_0^1 xnx^{n - 1}\, dx = n\int\limits_0^1 x^n\, dx = \frac{n}{n + 1}$\\
Посчитаем для $X_{(1)}$:\\
$F_{X_{(1)}}(x) = 1 - (1 - F_{\xi}(x))^n\\
f_{X_{(1)}}(x) = \big(F_{X_{(1)}}(x)\big)' = n(1 - F_{\xi}(x))^{n - 1} \big(F_{\xi}(x)\big)' = n(1 - F_{\xi}(x))^{n - 1} f_{\xi}(x) = \begin{cases}
    0,\ x < 0\\
    n(1 - x)^{n - 1},\ 0 \leq x \leq 1\\
    0,\ x > 1
\end{cases}$\\
$EX_{(1)} = \displaystyle \int\limits_0^1 xn(1 - x)^{n - 1}\, dx = n\int\limits_0^1 x(1 - x)^{n - 1}\, dx = \left< \begin{matrix}
    t = 1 - x\\
    x = 1 - t
\end{matrix}\right> = -n\int\limits_1^0 (1 - t)t^{n - 1}\, dt = n\int\limits_0^1(1 - t)t^{n - 1}\, dt =\\
= n\int\limits_0^1 t^{n - 1}\, dt - n\int\limits_0^1 t^{n}\, dt = 1 - \frac{n}{n + 1}$
\subsection*{Задача 3}
$\overline{x} = \frac{1}{n}\sum_{i = 1}^{n} x_i$\\
$E\overline{x} = E\left( \frac{1}{n} \sum_{i = 1}^{n} x_i \right) = \frac{1}{n} \sum_{i = 1}^{n} E(x_i) = Ex_i\\
\mathcal{D}(\overline{x}) = \mathcal{D}\left( \frac{1}{n} \sum_{i = 1}^{n} x_i \right) = \frac{1}{n^2} \sum_{i = 1}^{n} \mathcal{D}x_i = \frac{\mathcal{D}x_i}{n}$\\
Посчитаем выборочную дисперсию:
\[S^2 = \frac{1}{n} \sum_{i = 1}^{n}(x_i - \overline{x})^2\]
$ES^2 = E\left( \frac{1}{n} \sum_{i = 1}^{n} (x_i - \overline{x})^2 \right) = \frac{1}{n}\sum_{i = 1}^{n} E(x_i - \overline{x})^2 = \mathcal{D}(x_1 - \overline{x}) = \mathcal{D}(x_1) + \mathcal{D}(\overline{x}) - 2\cov(x_1,\ \overline{x}) = \frac{(n + 1)\dev(x_1)}{n} -\\- 2\cov(x_1,\ \overline{x})\\
\cov (x_1,\ \overline{x}) = \cov(x_1,\ \frac{1}{n} \sum_{i = 1}^{n} x_i) = \frac{1}{n}\cov (x_1,\ \sum_{i = 1}^{n} x_i) = \frac{1}{n} \cov (x_1,\ x_1) = \frac{\dev(x_1)}{n}$\\
Тогда 
\[ES^2 = \frac{(n + 1)\dev(x_1)}{n} - \frac{2\dev(x_1)}{n} = \dev(x_1)\left(1 - \frac{1}{n}\right)\]
Несмещённая выборачная дисперсия (её математическое ожидание равняется дисперсии $x_1$):
\[\tilde{S}^2 = \frac{1}{n - 1} \sum_{i = 1}^{n} (x_i - \overline{x})^2\]

Посчитаем дисперсию $S^2$:
\[\dev\left( x_1 - \frac{1}{n} \sum_{i = 1}^{n} x_i \right) = \dev\left( \frac{(n - 1)x_1}{n} \right) + \dev\left( \frac{1}{n} \sum_{i = 2}^n x_i \right) = \frac{(n - 1)^2}{n^2}\dev(x_1) + \frac{n - 1}{n^2}\dev(x_1) =\]
\[= \dev(x_1)\left( \frac{(n - 1)(n - 1 + 1)}{n^2}\right) = \dev (x_1) \frac{n - 1}{n}\]
\begin{center}
    Семинар 17 января.
\end{center}
$T(x_1, x_2, \dots, x_n) = \sqrt{\frac{\pi}{2}}\frac{1}{n}\sum \limits_{i=1}^n |x_i - m|$, $x_i \sim N(m, \theta^2)$

$\displaystyle E T(x_1, x_2, \dots, x_n) = \sqrt{\frac{\pi}{2}}\frac{1}{n}\sum \limits_{i=1}^n E |x_i - m| = \sqrt{\frac{\pi}{2}}\ E |x_1 - m| = \sqrt{\frac{\pi}{2}}\ \int \limits_{-\infty}^{+\infty} {|x - m| \frac{1}{\sqrt{2\pi}\theta} e^{-\frac{(x-m)^2}{2\theta^2}} dx}$ 

Заменим $\frac{x-m}{\theta}$ на $y$ 

$\displaystyle\frac{\theta}{2}\int \limits_{-\infty}^{+\infty}|y|\cdot e^{\frac{-y^2}{2}} dy  = \theta \int \limits_{0}^{+\infty} y\cdot e^{\frac{-y^2}{2}} dy = \theta (1-0) = \theta$


\[\frac{1}{n}\sum \limits_{i=1}^{n} \sqrt{\frac{\pi}{2}} \ |x_i-m| \xrightarrow[n\to+\infty]{\text{п.\ н.}}  E\sqrt{\frac{\pi}{2}} \ |x_i-m|\]

\subsection*{Задача}
$X = (X_1,\dots,\ X_n),\ X_i\sim R(0,\ \theta)$\\
$\hat{\theta} = X_{(n)},$ доказать $ \lim\limits_{n\to\infty} EX_{(n)} = \theta\\
F_{X_{(n)}} (x) = (F_{X_i}(x))^n = \left(\frac{x}{\theta}\right)^n\\
f_{X_{(n)}} (x) = \frac{d F_{X_{(n)}}}{dx} = \frac{nx^{n - 1}}{\theta}\\
\displaystyle E X_{(n)} = \int \limits _0^{\theta} \frac{n x^n}{\theta^n}\, dx = \left.\frac{n x^{n + 1}}{(n + 1)\theta^n} \right|_0^{\theta} = \frac{n}{n + 1}\theta \xrightarrow[n\to\infty]{} \theta$. То есть смещённая, но асимптотически несмещённая.\\
Докажем состоятельность, хотим:
\[\forall \varepsilon > 0\quad P(|\hat{\theta} - \theta| < \varepsilon) \xrightarrow[n\to\infty]{} 1\]
\[P(-\varepsilon < X_{(n)} - \theta < \varepsilon) = F_{X_(n)} (\varepsilon + \theta) - F_{X_{(n)}}(\theta - \varepsilon) = 1 - \left( \frac{\theta - \varepsilon}{\theta} \right)^n \xrightarrow[n\to\infty]{} 1\]
\subsection*{Задача}
$I_n(\theta) = E\left( \frac{\delta \ln f(x,\ \theta)}{\delta \theta} \right)^2,\ I_n(\theta) = nI_1(\theta),\ x_1,\dots,\ x_n\sim N(\theta,\ \sigma^2)$.\\
$f(x,\ \theta) = \frac{1}{\sqrt{2\pi}\sigma} e^{-\frac{(x - \theta)^2}{2\sigma^2}}$\\
$\ln f(x,\ \theta) = \ln \left( \frac{1}{\sqrt{2\pi}\sigma }e^{-\frac{(x - \theta)^2}{2\sigma^2}} \right) = -\frac{(x - \theta)^2}{2\sigma^2} + \ln \frac{1}{\sqrt{2\pi}\sigma}\\
\frac{\delta\ln f(x,\ \theta)}{\delta\theta} = -\frac{2(x - \theta)}{2\sigma^2}\cdot (-1) = \frac{x - \theta}{\sigma^2}\\
E\left( \frac{x - \theta}{\sigma^2} \right)^2 = \frac{1}{\sigma^4} E(x - \theta)^2 = \frac{1}{\sigma^4} \sigma^2 = \frac{1}{\sigma^2} = I_1(\theta)\\
\dev \hat{\theta} \geq \frac{1}{nI_1(\theta)} = \frac{\sigma^2}{n} =\mathcal{D}\overline{x}$
\begin{center}
    Семинар 24 января
\end{center}
\subsection*{Задача 4 ДЗ}
$\hat{K}_{xy} = \frac{1}{n} \sum_{i = 1}^{n} (x_i - \overline{X})(y_i - \overline{Y}) = \frac{1}{n} \sum_{i = 1}^{n} (x_i - \overline{X} + Ex_1 - Ex_1)(y_i - \overline{Y} + Ey_1 - Ey_1)$\\
$E\hat{K}_{xy} = E\frac{1}{n} \sum_{i = 1}^{n} \left( (x_i - Ex_1) - (\overline{X} - Ex_1) \right) \left( (y_1 - Ey_1) - (\overline{Y} - Ey_1) \right) =\\
= E\left( (x_i - Ex_1) - (\overline{X} - Ex_1) \right) \cdot \left( (y_1 - Ey_1) - (\overline{Y} - Ey_1) \right) = E\left( (x_1 - Ex_1)(y_1 - Ey_1) + (x_1 - Ex_1)(\overline{Y} - Ey_1) +\\
+(\overline{X} - Ex_1)(y_1 - Ey_1) + (\overline{X} - Ex_1)(\overline{Y} - Ey_1) \right) = \\
=\cov(x,\ y) - \frac{1}{n}\cov(x,\ y) - \frac{1}{n}\cov(x,\ y) + \frac{1}{n}\cov(x,\ y) $
\subsection*{Задача 5 ДЗ}
Решал у доски, всем gl.
\subsection*{Задача 1}
$X_1,\dots,\ X_n \sim \Pi(\theta)$. Проверить, что оценка $\hat{\theta} = \overline{X}$ является R-эффективной.\\
$E\hat{\theta} = E\frac{1}{n} \sum_{i = 1}^{n} x_i = Ex_1 = \theta\\
\dev \frac{1}{n} \sum_{i = 1}^{n} x_i = \frac{1}{n}\theta\\
P(\xi = x_1) = \frac{e^{-\theta}\theta^{x_1}}{x_1!}$. Логарифмируем:
\[\ln \frac{e^{-\theta}\theta^{x_1}}{x_1!} = {-\theta + x_1 \ln \theta - \ln x_1!}\]
Возьмём частную производную:
\[\frac{\delta(-\theta + x_1 \ln \theta - \ln x_1!)}{\delta\theta} = -1 + \frac{x_1}{\theta}\]
Возьмём матожидание квадрата этой величины:
\[E(-1 + \frac{x_1}{\theta})^2 = \frac{1}{\theta^2} E(x_1 - \theta)^2 = \frac{\dev x_1}{\theta^2} = \frac{1}{\theta}\Rightarrow I_n(\theta) = \frac{n}{\theta}\]
Попробуем самостоятельно подогнать оценку:\\
$U(x,\ \theta) = \sum_{i = 1}^{n} -1 + \frac{x_1}{\theta} = \frac{1}{\theta} \sum_{i = 1}^{n} (x_i - \theta) = \frac{1}{\theta}\left(-n\theta + \sum_{i = 1}^{n} \frac{x_i}{n}\right) = \frac{n}{\theta}\left( \sum_{i = 1}^{n} \left( \frac{x_i}{n}\right) - \theta \right)\\
\hat{\theta} - \theta = a(\theta)U(x,\ \theta)\Rightarrow a(\theta) = \frac{\theta}{n},\ \hat{\theta} = \sum_{i = 1}^{n} \frac{x_i}{n}$
\begin{center}
    ДЗ
\end{center}
\subsubsection*{Задача 1}
$X_1,\dots,\ X_n\sim N(\theta, \sigma^2)\Rightarrow \forall i = \overline{1,\ n}\quad f(x_i,\ \theta) = \frac{1}{\sqrt{2\pi}\sigma} e^{-\frac{(x - \theta)^2}{2\sigma^2}}$\\
\[\ln f(x_i,\ \theta) = \ln \frac{1}{\sqrt{2\pi}\sigma} - \frac{(x - \theta)^2}{2\sigma^2} = \ln \frac{1}{\sqrt{2\pi}\theta} - \frac{x^2}{2\sigma^2} + \frac{\theta x}{\sigma^2} - \frac{\theta^2}{2\sigma^2}\Rightarrow \frac{\delta}{\delta \theta} f(x_i,\ \theta) = \frac{x}{\sigma} - \frac{\theta}{\sigma^2}\]
\[U(x,\ \theta) = \sum\limits_{i = 1}^{n}\left( \frac{x_i}{\sigma} - \frac{\theta}{\sigma^2} \right)\]
По критерию эффективности хотим:
\[\hat{\theta} - \theta = \alpha(x)U(x,\ \theta)\]
Преобразуем: $U(x,\ \theta) = \left(\sum\limits_{i = 1}^{n} \frac{x_i}{\sigma}\right) - \frac{n\theta}{\sigma^2}\Rightarrow \underset{\alpha(\sigma)}{\underbrace{\frac{\sigma^2}{n}}} U(x,\ \theta) = \underset{\hat{\theta}}{\underbrace{\left(\frac{1}{n}\sum\limits_{i = 1}^{n} \sigma x_i \right)}} - \theta$
\subsubsection*{Задача 2}
$X_1,\dots,\ X_n\sim N(m,\ \theta)\Rightarrow f(x_i,\ \theta) = \frac{1}{\sqrt{2\pi\theta}} e^{-\frac{(x - m)^2}{2\theta}}$\\
\[\ln f(x,\ \theta) = \ln \frac{1}{\sqrt{2\pi}} - \frac{1}{2}\ln \theta - \frac{(x - m)^2}{2\theta}\Rightarrow \frac{\delta}{\delta \theta} f(x,\ \theta) = -\frac{1}{2\theta} + \frac{(x - m)^2}{2\theta^2}\]
Применим критерий эффективности:
\[U(x,\ \theta) = \sum\limits_{i = 1}^{n} \left( \frac{(x - m)^2}{2\theta^2} - \frac{1}{2\theta} \right) = \sum\limits_{i = 1}^{n} \left( \frac{(x - m)^2 - \theta}{2\theta^2}\right) = \frac{1}{2\theta^2} \sum\limits_{i = 1}^{n} \big((x - m)^2 - \theta\big) = \]
\[= \frac{1}{2\theta^2} \left( \sum\limits_{i = 1}^{n} \big((x - m)^2\big) - n\theta\right) = \frac{n}{2\theta^2} \left( \frac{1}{n} \sum\limits_{i = 1}^{n} \big((x - m)^2\big) - \theta\right)\Rightarrow \underset{\alpha(\theta)}{\underbrace{\frac{2\theta^2}{n}}} U(x,\ \theta) = \underset{\hat{\theta}}{\underbrace{\left(\frac{1}{n} \sum\limits_{i = 1}^{n} (x - m)^2\right)}} - \theta\]
\subsubsection*{Задача 3}
$X_1,\dots,\ X_n \sim G(\theta)\Rightarrow Ex = \frac{1}{\theta}$. Проверить оценку $\hat{\theta} = \frac{1}{\overline{X}}$ на несмещённость.\\
Хотим $E\hat{\theta} = \theta$. Попробуем по определению:
\[E\hat{\theta} = E \frac{n}{\sum\limits_{i = 1}^{n} x_i} = n E\frac{1}{\sum\limits_{i = 1}^{n} x_i}?\]
Попробуем решить через функцию правдободобия:
\[L(x_1,\dots,\ x_n,\ \theta) = \prod_{i = 1}^{n} P(\xi = x_i,\ \theta) = \prod_{i = 1}^{n} (1 - \theta)^{x_i - 1}\theta \approx f(x,\ \theta)\]
\[\ln f(x_i,\ \theta) = \ln \big( (1 - \theta)^{x_i - 1} \theta \big) = (x_i - 1)\ln (1 - \theta) + \ln \theta\]
\[\frac{\delta}{\delta \theta} \ln f(x,\ \theta) = \frac{1}{\theta} - \frac{x_i - 1}{1 - \theta} = \frac{1 - \theta - \theta x_i + \theta}{\theta - \theta^2} = \frac{1 - \theta x_i}{\theta - \theta^2}\]
Применим критерий эффективности:
\[U(x,\ \theta) = \sum_{i = 1}^{n} \frac{1 - \theta x_i}{\theta - \theta^2} = \frac{1}{\theta - \theta^2} \left( n - \theta \sum_{i = 1}^{n} x_i \right) = \frac{n}{\theta - \theta^2} \left( 1 - \frac{\theta}{n} \sum_{i = 1}^{n} \right) = \frac{n\overline{X}}{\theta - \theta^2} \left( \frac{1}{\overline{X}} - \theta \right)\]
Значит $\frac{1}{\overline{X}}$ является R-эффективной, то есть несмещённой.
\subsubsection*{Задача 4}
$X_1,\dots,\ X_n \sim Bi(k,\ \theta)$. Показать, что $\hat{\theta}  = \frac{\overline{X}}{k}$ R-эффективная.\\
Посчитаем функцию правдободобия:
\[L(x_1,\dots,\ x_n,\ \theta) = \prod_{i = 1}^{n} P(\xi = x_i,\ \theta) = \prod_{i = 1}^{n} C_n^k\theta^{x_i}\cdot (1 - \theta)^{k - x_i} \approx f(x_i,\ \theta)\]
\[\ln f(x_i,\ \theta) = \ln\frac{n!}{k!(n - k)!} + x_i \ln \theta + (k - x_i)\ln (1 - \theta)\]
\[\frac{\delta}{\delta \theta} \ln f(x_i,\ \theta) = \frac{x_i}{\theta} + \frac{x_i - k}{1 - \theta} = \frac{x_i - \theta x_i + \theta x_i - \theta k}{\theta - \theta^2} = \frac{x_i - \theta k}{\theta - \theta^2}\]
Применим критерий эффективности:
\[U(x,\ \theta) = \sum_{i = 1}^{n} \frac{x_i - \theta k}{\theta - \theta^2} = \frac{1}{\theta - \theta^2}\left(-n\theta k + \sum_{i = 1}^{n} x_i\right) = \frac{nk}{\theta - \theta^2}\left( \frac{1}{nk} \sum_{i = 1}^{n} (x_i) - \theta \right) = \frac{nk}{\theta - \theta^2} \left( \frac{\overline{X}}{k} - \theta \right)\]
Получается, что $\frac{\overline{X}}{k}$ является R-эффективной (тут также выполнена 5 задача, так как я применил критерий эффективности в доказательстве).

\end{document}