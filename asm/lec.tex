\documentclass[12pt, letterpaper, twoside]{article}
\usepackage[T2A]{fontenc}
\usepackage{amsfonts}
\usepackage{amsmath}
\usepackage{mathabx}
\usepackage{graphicx}
\usepackage{tikz}

\title{Лекции по ассембелру модуль 4}
\author{Андрей Тищенко}
\date{2023/2024 гг.}

\begin{document}
    \maketitle
    \[\textbf{Лекция 6 апреля.}\]   
    \[\text{Организационные вопросы}\]
    См. в презентации (в телеграмм канале).
    \[\text{Мотивация}\]
    \begin{enumerate}
        \item[Проблема 1.] Свойства чисел. Свойства математических абстракций при выполнении вычислительной техникой несколько меняются.
        \item[Проблема 2.] Отладка. Где размещаются переменные, в каком порядке, что происходит при попадании в UB или в реализационно зависимое поведение. Как сделать отладчик?
        \item[Проблема 3.] Производительность. Понятие сложности с точки зрения математики. 
        \item[Проблема 4.] Безопасность. Что на самом деле выполняется компьютером. Зачастую проблемы безопасности = ошибки в коде. Сейчас продвинутые эксполиты мимикрируют под ошибки в коде.   
    \end{enumerate}
    \[\text{История первых компьютеров}\]
    Смотри презентацию.
    \[\text{Принципы фон Неймана}\]
    \begin{enumerate}
        \item[1.] Двоичное кодирование информации.\\
        В дополнительном коде данные хранятся в виде $\displaystyle\vec{x} =  2^{n - 1}x_{n - 1} + \sum_{i = 0}^{n - 2} 2^{i}x_{i}$
        \item[2.] Неразличимость команд и данных.\\
        Команды хранятся в памяти в виде:\\
        Код операции, $\text{Операнд}_1$, $\text{Операнд}_2$
        \item[3.] Адресуемость памяти.
        \item[4.] Последовательное выполнение команд. 
    \end{enumerate}
    \[\text{(Модельный) цикл работы ЭВМ}\]
    \begin{enumerate}
        \item[1.] Извлечение инструкции из памяти. Используя текущее положение счётчика команд, процессор излекает некоторое количество байт из памяти и помещает их в буфер команд.
        \item[2.] Декодирование команды. Процессор просматривает содержимое буфера команд и определеяет код операции и ее операнды. Длина декодированной команды прибавляется к текущему значению счётчика команд.
        \item[3.] Загрузка операндов. Извлекаются значения операндов. Если операнд размещен в ячейках памяти - вычисляется исполнительный адрес.
        \item[4.] Выполнение операции над данными.
        \item[5.] Запись результата. Результат может быть записан в том числе и в счётчиках команд для изменения естественного порядка выполнения. Возвращаемся в пункт 1.\\
        Интересный факт: взятие данных из регистра происходит быстрее, чем из памяти, потому что из регистра до процессора ток идёт быстрее. 
    \end{enumerate}
    \[\text{Промежуточные итоги}\]
    \begin{enumerate}
        \item Регистр, имя регистра, значение регистра, разрядность регистра, машинное слово.
        \item Память ячейка памяти, адрес ячейки памяти, время доступа к памяти.
        \item Цикл, тактовая частота, счётчик команд (указывает на очередную команду).
        \item Машинная команда, код операции, операнд, адреность команды (сколько операндов в команде. 2 операнда - двухадресная, 1 - одноадресная, 0 - безадресная, 3 - трёхадресная), исполнительный (действительный адрес)
        \item Способ адресации: прямая, непосредственная, косвенная.
    \end{enumerate}
    
\end{document}